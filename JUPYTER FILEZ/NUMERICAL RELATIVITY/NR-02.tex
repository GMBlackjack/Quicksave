% Based on http://nbviewer.jupyter.org/github/ipython/nbconvert-examples/blob/master/citations/Tutorial.ipynb , authored by Brian E. Granger
    % Declare the document class
    \documentclass[landscape,letterpaper,10pt,english]{article}


    \usepackage[breakable]{tcolorbox}
    \usepackage{parskip} % Stop auto-indenting (to mimic markdown behaviour)
    

    % Basic figure setup, for now with no caption control since it's done
    % automatically by Pandoc (which extracts ![](path) syntax from Markdown).
    \usepackage{graphicx}
    % Maintain compatibility with old templates. Remove in nbconvert 6.0
    \let\Oldincludegraphics\includegraphics
    % Ensure that by default, figures have no caption (until we provide a
    % proper Figure object with a Caption API and a way to capture that
    % in the conversion process - todo).
    \usepackage{caption}
    \DeclareCaptionFormat{nocaption}{}
    \captionsetup{format=nocaption,aboveskip=0pt,belowskip=0pt}

    \usepackage{float}
    \floatplacement{figure}{H} % forces figures to be placed at the correct location
    \usepackage{xcolor} % Allow colors to be defined
    \usepackage{enumerate} % Needed for markdown enumerations to work
    \usepackage{geometry} % Used to adjust the document margins
    \usepackage{amsmath} % Equations
    \usepackage{amssymb} % Equations
    \usepackage{textcomp} % defines textquotesingle
    % Hack from http://tex.stackexchange.com/a/47451/13684:
    \AtBeginDocument{%
        \def\PYZsq{\textquotesingle}% Upright quotes in Pygmentized code
    }
    \usepackage{upquote} % Upright quotes for verbatim code
    \usepackage{eurosym} % defines \euro

    \usepackage{iftex}
    \ifPDFTeX
        \usepackage[T1]{fontenc}
        \IfFileExists{alphabeta.sty}{
              \usepackage{alphabeta}
          }{
              \usepackage[mathletters]{ucs}
              \usepackage[utf8x]{inputenc}
          }
    \else
        \usepackage{fontspec}
        \usepackage{unicode-math}
    \fi

    \usepackage{fancyvrb} % verbatim replacement that allows latex
    \usepackage{grffile} % extends the file name processing of package graphics
                         % to support a larger range
    \makeatletter % fix for old versions of grffile with XeLaTeX
    \@ifpackagelater{grffile}{2019/11/01}
    {
      % Do nothing on new versions
    }
    {
      \def\Gread@@xetex#1{%
        \IfFileExists{"\Gin@base".bb}%
        {\Gread@eps{\Gin@base.bb}}%
        {\Gread@@xetex@aux#1}%
      }
    }
    \makeatother
    \usepackage[Export]{adjustbox} % Used to constrain images to a maximum size
    \adjustboxset{max size={0.9\linewidth}{0.9\paperheight}}

    % The hyperref package gives us a pdf with properly built
    % internal navigation ('pdf bookmarks' for the table of contents,
    % internal cross-reference links, web links for URLs, etc.)
    \usepackage{hyperref}
    % The default LaTeX title has an obnoxious amount of whitespace. By default,
    % titling removes some of it. It also provides customization options.
    \usepackage{titling}
    \usepackage{longtable} % longtable support required by pandoc >1.10
    \usepackage{booktabs}  % table support for pandoc > 1.12.2
    \usepackage{array}     % table support for pandoc >= 2.11.3
    \usepackage{calc}      % table minipage width calculation for pandoc >= 2.11.1
    \usepackage[inline]{enumitem} % IRkernel/repr support (it uses the enumerate* environment)
    \usepackage[normalem]{ulem} % ulem is needed to support strikethroughs (\sout)
                                % normalem makes italics be italics, not underlines
    \usepackage{mathrsfs}
    

    
    % Colors for the hyperref package
    \definecolor{urlcolor}{rgb}{0,.145,.698}
    \definecolor{linkcolor}{rgb}{.71,0.21,0.01}
    \definecolor{citecolor}{rgb}{.12,.54,.11}

    % ANSI colors
    \definecolor{ansi-black}{HTML}{3E424D}
    \definecolor{ansi-black-intense}{HTML}{282C36}
    \definecolor{ansi-red}{HTML}{E75C58}
    \definecolor{ansi-red-intense}{HTML}{B22B31}
    \definecolor{ansi-green}{HTML}{00A250}
    \definecolor{ansi-green-intense}{HTML}{007427}
    \definecolor{ansi-yellow}{HTML}{DDB62B}
    \definecolor{ansi-yellow-intense}{HTML}{B27D12}
    \definecolor{ansi-blue}{HTML}{208FFB}
    \definecolor{ansi-blue-intense}{HTML}{0065CA}
    \definecolor{ansi-magenta}{HTML}{D160C4}
    \definecolor{ansi-magenta-intense}{HTML}{A03196}
    \definecolor{ansi-cyan}{HTML}{60C6C8}
    \definecolor{ansi-cyan-intense}{HTML}{258F8F}
    \definecolor{ansi-white}{HTML}{C5C1B4}
    \definecolor{ansi-white-intense}{HTML}{A1A6B2}
    \definecolor{ansi-default-inverse-fg}{HTML}{FFFFFF}
    \definecolor{ansi-default-inverse-bg}{HTML}{000000}

    % common color for the border for error outputs.
    \definecolor{outerrorbackground}{HTML}{FFDFDF}

    % commands and environments needed by pandoc snippets
    % extracted from the output of `pandoc -s`
    \providecommand{\tightlist}{%
      \setlength{\itemsep}{0pt}\setlength{\parskip}{0pt}}
    \DefineVerbatimEnvironment{Highlighting}{Verbatim}{commandchars=\\\{\}}
    % Add ',fontsize=\small' for more characters per line
    \newenvironment{Shaded}{}{}
    \newcommand{\KeywordTok}[1]{\textcolor[rgb]{0.00,0.44,0.13}{\textbf{{#1}}}}
    \newcommand{\DataTypeTok}[1]{\textcolor[rgb]{0.56,0.13,0.00}{{#1}}}
    \newcommand{\DecValTok}[1]{\textcolor[rgb]{0.25,0.63,0.44}{{#1}}}
    \newcommand{\BaseNTok}[1]{\textcolor[rgb]{0.25,0.63,0.44}{{#1}}}
    \newcommand{\FloatTok}[1]{\textcolor[rgb]{0.25,0.63,0.44}{{#1}}}
    \newcommand{\CharTok}[1]{\textcolor[rgb]{0.25,0.44,0.63}{{#1}}}
    \newcommand{\StringTok}[1]{\textcolor[rgb]{0.25,0.44,0.63}{{#1}}}
    \newcommand{\CommentTok}[1]{\textcolor[rgb]{0.38,0.63,0.69}{\textit{{#1}}}}
    \newcommand{\OtherTok}[1]{\textcolor[rgb]{0.00,0.44,0.13}{{#1}}}
    \newcommand{\AlertTok}[1]{\textcolor[rgb]{1.00,0.00,0.00}{\textbf{{#1}}}}
    \newcommand{\FunctionTok}[1]{\textcolor[rgb]{0.02,0.16,0.49}{{#1}}}
    \newcommand{\RegionMarkerTok}[1]{{#1}}
    \newcommand{\ErrorTok}[1]{\textcolor[rgb]{1.00,0.00,0.00}{\textbf{{#1}}}}
    \newcommand{\NormalTok}[1]{{#1}}

    % Additional commands for more recent versions of Pandoc
    \newcommand{\ConstantTok}[1]{\textcolor[rgb]{0.53,0.00,0.00}{{#1}}}
    \newcommand{\SpecialCharTok}[1]{\textcolor[rgb]{0.25,0.44,0.63}{{#1}}}
    \newcommand{\VerbatimStringTok}[1]{\textcolor[rgb]{0.25,0.44,0.63}{{#1}}}
    \newcommand{\SpecialStringTok}[1]{\textcolor[rgb]{0.73,0.40,0.53}{{#1}}}
    \newcommand{\ImportTok}[1]{{#1}}
    \newcommand{\DocumentationTok}[1]{\textcolor[rgb]{0.73,0.13,0.13}{\textit{{#1}}}}
    \newcommand{\AnnotationTok}[1]{\textcolor[rgb]{0.38,0.63,0.69}{\textbf{\textit{{#1}}}}}
    \newcommand{\CommentVarTok}[1]{\textcolor[rgb]{0.38,0.63,0.69}{\textbf{\textit{{#1}}}}}
    \newcommand{\VariableTok}[1]{\textcolor[rgb]{0.10,0.09,0.49}{{#1}}}
    \newcommand{\ControlFlowTok}[1]{\textcolor[rgb]{0.00,0.44,0.13}{\textbf{{#1}}}}
    \newcommand{\OperatorTok}[1]{\textcolor[rgb]{0.40,0.40,0.40}{{#1}}}
    \newcommand{\BuiltInTok}[1]{{#1}}
    \newcommand{\ExtensionTok}[1]{{#1}}
    \newcommand{\PreprocessorTok}[1]{\textcolor[rgb]{0.74,0.48,0.00}{{#1}}}
    \newcommand{\AttributeTok}[1]{\textcolor[rgb]{0.49,0.56,0.16}{{#1}}}
    \newcommand{\InformationTok}[1]{\textcolor[rgb]{0.38,0.63,0.69}{\textbf{\textit{{#1}}}}}
    \newcommand{\WarningTok}[1]{\textcolor[rgb]{0.38,0.63,0.69}{\textbf{\textit{{#1}}}}}


    % Define a nice break command that doesn't care if a line doesn't already
    % exist.
    \def\br{\hspace*{\fill} \\* }
    % Math Jax compatibility definitions
    \def\gt{>}
    \def\lt{<}
    \let\Oldtex\TeX
    \let\Oldlatex\LaTeX
    \renewcommand{\TeX}{\textrm{\Oldtex}}
    \renewcommand{\LaTeX}{\textrm{\Oldlatex}}
    % Document parameters
    % Document title
    \title{NR-02}
    
    
    
    
    
% Pygments definitions
\makeatletter
\def\PY@reset{\let\PY@it=\relax \let\PY@bf=\relax%
    \let\PY@ul=\relax \let\PY@tc=\relax%
    \let\PY@bc=\relax \let\PY@ff=\relax}
\def\PY@tok#1{\csname PY@tok@#1\endcsname}
\def\PY@toks#1+{\ifx\relax#1\empty\else%
    \PY@tok{#1}\expandafter\PY@toks\fi}
\def\PY@do#1{\PY@bc{\PY@tc{\PY@ul{%
    \PY@it{\PY@bf{\PY@ff{#1}}}}}}}
\def\PY#1#2{\PY@reset\PY@toks#1+\relax+\PY@do{#2}}

\@namedef{PY@tok@w}{\def\PY@tc##1{\textcolor[rgb]{0.73,0.73,0.73}{##1}}}
\@namedef{PY@tok@c}{\let\PY@it=\textit\def\PY@tc##1{\textcolor[rgb]{0.24,0.48,0.48}{##1}}}
\@namedef{PY@tok@cp}{\def\PY@tc##1{\textcolor[rgb]{0.61,0.40,0.00}{##1}}}
\@namedef{PY@tok@k}{\let\PY@bf=\textbf\def\PY@tc##1{\textcolor[rgb]{0.00,0.50,0.00}{##1}}}
\@namedef{PY@tok@kp}{\def\PY@tc##1{\textcolor[rgb]{0.00,0.50,0.00}{##1}}}
\@namedef{PY@tok@kt}{\def\PY@tc##1{\textcolor[rgb]{0.69,0.00,0.25}{##1}}}
\@namedef{PY@tok@o}{\def\PY@tc##1{\textcolor[rgb]{0.40,0.40,0.40}{##1}}}
\@namedef{PY@tok@ow}{\let\PY@bf=\textbf\def\PY@tc##1{\textcolor[rgb]{0.67,0.13,1.00}{##1}}}
\@namedef{PY@tok@nb}{\def\PY@tc##1{\textcolor[rgb]{0.00,0.50,0.00}{##1}}}
\@namedef{PY@tok@nf}{\def\PY@tc##1{\textcolor[rgb]{0.00,0.00,1.00}{##1}}}
\@namedef{PY@tok@nc}{\let\PY@bf=\textbf\def\PY@tc##1{\textcolor[rgb]{0.00,0.00,1.00}{##1}}}
\@namedef{PY@tok@nn}{\let\PY@bf=\textbf\def\PY@tc##1{\textcolor[rgb]{0.00,0.00,1.00}{##1}}}
\@namedef{PY@tok@ne}{\let\PY@bf=\textbf\def\PY@tc##1{\textcolor[rgb]{0.80,0.25,0.22}{##1}}}
\@namedef{PY@tok@nv}{\def\PY@tc##1{\textcolor[rgb]{0.10,0.09,0.49}{##1}}}
\@namedef{PY@tok@no}{\def\PY@tc##1{\textcolor[rgb]{0.53,0.00,0.00}{##1}}}
\@namedef{PY@tok@nl}{\def\PY@tc##1{\textcolor[rgb]{0.46,0.46,0.00}{##1}}}
\@namedef{PY@tok@ni}{\let\PY@bf=\textbf\def\PY@tc##1{\textcolor[rgb]{0.44,0.44,0.44}{##1}}}
\@namedef{PY@tok@na}{\def\PY@tc##1{\textcolor[rgb]{0.41,0.47,0.13}{##1}}}
\@namedef{PY@tok@nt}{\let\PY@bf=\textbf\def\PY@tc##1{\textcolor[rgb]{0.00,0.50,0.00}{##1}}}
\@namedef{PY@tok@nd}{\def\PY@tc##1{\textcolor[rgb]{0.67,0.13,1.00}{##1}}}
\@namedef{PY@tok@s}{\def\PY@tc##1{\textcolor[rgb]{0.73,0.13,0.13}{##1}}}
\@namedef{PY@tok@sd}{\let\PY@it=\textit\def\PY@tc##1{\textcolor[rgb]{0.73,0.13,0.13}{##1}}}
\@namedef{PY@tok@si}{\let\PY@bf=\textbf\def\PY@tc##1{\textcolor[rgb]{0.64,0.35,0.47}{##1}}}
\@namedef{PY@tok@se}{\let\PY@bf=\textbf\def\PY@tc##1{\textcolor[rgb]{0.67,0.36,0.12}{##1}}}
\@namedef{PY@tok@sr}{\def\PY@tc##1{\textcolor[rgb]{0.64,0.35,0.47}{##1}}}
\@namedef{PY@tok@ss}{\def\PY@tc##1{\textcolor[rgb]{0.10,0.09,0.49}{##1}}}
\@namedef{PY@tok@sx}{\def\PY@tc##1{\textcolor[rgb]{0.00,0.50,0.00}{##1}}}
\@namedef{PY@tok@m}{\def\PY@tc##1{\textcolor[rgb]{0.40,0.40,0.40}{##1}}}
\@namedef{PY@tok@gh}{\let\PY@bf=\textbf\def\PY@tc##1{\textcolor[rgb]{0.00,0.00,0.50}{##1}}}
\@namedef{PY@tok@gu}{\let\PY@bf=\textbf\def\PY@tc##1{\textcolor[rgb]{0.50,0.00,0.50}{##1}}}
\@namedef{PY@tok@gd}{\def\PY@tc##1{\textcolor[rgb]{0.63,0.00,0.00}{##1}}}
\@namedef{PY@tok@gi}{\def\PY@tc##1{\textcolor[rgb]{0.00,0.52,0.00}{##1}}}
\@namedef{PY@tok@gr}{\def\PY@tc##1{\textcolor[rgb]{0.89,0.00,0.00}{##1}}}
\@namedef{PY@tok@ge}{\let\PY@it=\textit}
\@namedef{PY@tok@gs}{\let\PY@bf=\textbf}
\@namedef{PY@tok@gp}{\let\PY@bf=\textbf\def\PY@tc##1{\textcolor[rgb]{0.00,0.00,0.50}{##1}}}
\@namedef{PY@tok@go}{\def\PY@tc##1{\textcolor[rgb]{0.44,0.44,0.44}{##1}}}
\@namedef{PY@tok@gt}{\def\PY@tc##1{\textcolor[rgb]{0.00,0.27,0.87}{##1}}}
\@namedef{PY@tok@err}{\def\PY@bc##1{{\setlength{\fboxsep}{\string -\fboxrule}\fcolorbox[rgb]{1.00,0.00,0.00}{1,1,1}{\strut ##1}}}}
\@namedef{PY@tok@kc}{\let\PY@bf=\textbf\def\PY@tc##1{\textcolor[rgb]{0.00,0.50,0.00}{##1}}}
\@namedef{PY@tok@kd}{\let\PY@bf=\textbf\def\PY@tc##1{\textcolor[rgb]{0.00,0.50,0.00}{##1}}}
\@namedef{PY@tok@kn}{\let\PY@bf=\textbf\def\PY@tc##1{\textcolor[rgb]{0.00,0.50,0.00}{##1}}}
\@namedef{PY@tok@kr}{\let\PY@bf=\textbf\def\PY@tc##1{\textcolor[rgb]{0.00,0.50,0.00}{##1}}}
\@namedef{PY@tok@bp}{\def\PY@tc##1{\textcolor[rgb]{0.00,0.50,0.00}{##1}}}
\@namedef{PY@tok@fm}{\def\PY@tc##1{\textcolor[rgb]{0.00,0.00,1.00}{##1}}}
\@namedef{PY@tok@vc}{\def\PY@tc##1{\textcolor[rgb]{0.10,0.09,0.49}{##1}}}
\@namedef{PY@tok@vg}{\def\PY@tc##1{\textcolor[rgb]{0.10,0.09,0.49}{##1}}}
\@namedef{PY@tok@vi}{\def\PY@tc##1{\textcolor[rgb]{0.10,0.09,0.49}{##1}}}
\@namedef{PY@tok@vm}{\def\PY@tc##1{\textcolor[rgb]{0.10,0.09,0.49}{##1}}}
\@namedef{PY@tok@sa}{\def\PY@tc##1{\textcolor[rgb]{0.73,0.13,0.13}{##1}}}
\@namedef{PY@tok@sb}{\def\PY@tc##1{\textcolor[rgb]{0.73,0.13,0.13}{##1}}}
\@namedef{PY@tok@sc}{\def\PY@tc##1{\textcolor[rgb]{0.73,0.13,0.13}{##1}}}
\@namedef{PY@tok@dl}{\def\PY@tc##1{\textcolor[rgb]{0.73,0.13,0.13}{##1}}}
\@namedef{PY@tok@s2}{\def\PY@tc##1{\textcolor[rgb]{0.73,0.13,0.13}{##1}}}
\@namedef{PY@tok@sh}{\def\PY@tc##1{\textcolor[rgb]{0.73,0.13,0.13}{##1}}}
\@namedef{PY@tok@s1}{\def\PY@tc##1{\textcolor[rgb]{0.73,0.13,0.13}{##1}}}
\@namedef{PY@tok@mb}{\def\PY@tc##1{\textcolor[rgb]{0.40,0.40,0.40}{##1}}}
\@namedef{PY@tok@mf}{\def\PY@tc##1{\textcolor[rgb]{0.40,0.40,0.40}{##1}}}
\@namedef{PY@tok@mh}{\def\PY@tc##1{\textcolor[rgb]{0.40,0.40,0.40}{##1}}}
\@namedef{PY@tok@mi}{\def\PY@tc##1{\textcolor[rgb]{0.40,0.40,0.40}{##1}}}
\@namedef{PY@tok@il}{\def\PY@tc##1{\textcolor[rgb]{0.40,0.40,0.40}{##1}}}
\@namedef{PY@tok@mo}{\def\PY@tc##1{\textcolor[rgb]{0.40,0.40,0.40}{##1}}}
\@namedef{PY@tok@ch}{\let\PY@it=\textit\def\PY@tc##1{\textcolor[rgb]{0.24,0.48,0.48}{##1}}}
\@namedef{PY@tok@cm}{\let\PY@it=\textit\def\PY@tc##1{\textcolor[rgb]{0.24,0.48,0.48}{##1}}}
\@namedef{PY@tok@cpf}{\let\PY@it=\textit\def\PY@tc##1{\textcolor[rgb]{0.24,0.48,0.48}{##1}}}
\@namedef{PY@tok@c1}{\let\PY@it=\textit\def\PY@tc##1{\textcolor[rgb]{0.24,0.48,0.48}{##1}}}
\@namedef{PY@tok@cs}{\let\PY@it=\textit\def\PY@tc##1{\textcolor[rgb]{0.24,0.48,0.48}{##1}}}

\def\PYZbs{\char`\\}
\def\PYZus{\char`\_}
\def\PYZob{\char`\{}
\def\PYZcb{\char`\}}
\def\PYZca{\char`\^}
\def\PYZam{\char`\&}
\def\PYZlt{\char`\<}
\def\PYZgt{\char`\>}
\def\PYZsh{\char`\#}
\def\PYZpc{\char`\%}
\def\PYZdl{\char`\$}
\def\PYZhy{\char`\-}
\def\PYZsq{\char`\'}
\def\PYZdq{\char`\"}
\def\PYZti{\char`\~}
% for compatibility with earlier versions
\def\PYZat{@}
\def\PYZlb{[}
\def\PYZrb{]}
\makeatother


    % For linebreaks inside Verbatim environment from package fancyvrb.
    \makeatletter
        \newbox\Wrappedcontinuationbox
        \newbox\Wrappedvisiblespacebox
        \newcommand*\Wrappedvisiblespace {\textcolor{red}{\textvisiblespace}}
        \newcommand*\Wrappedcontinuationsymbol {\textcolor{red}{\llap{\tiny$\m@th\hookrightarrow$}}}
        \newcommand*\Wrappedcontinuationindent {3ex }
        \newcommand*\Wrappedafterbreak {\kern\Wrappedcontinuationindent\copy\Wrappedcontinuationbox}
        % Take advantage of the already applied Pygments mark-up to insert
        % potential linebreaks for TeX processing.
        %        {, <, #, %, $, ' and ": go to next line.
        %        _, }, ^, &, >, - and ~: stay at end of broken line.
        % Use of \textquotesingle for straight quote.
        \newcommand*\Wrappedbreaksatspecials {%
            \def\PYGZus{\discretionary{\char`\_}{\Wrappedafterbreak}{\char`\_}}%
            \def\PYGZob{\discretionary{}{\Wrappedafterbreak\char`\{}{\char`\{}}%
            \def\PYGZcb{\discretionary{\char`\}}{\Wrappedafterbreak}{\char`\}}}%
            \def\PYGZca{\discretionary{\char`\^}{\Wrappedafterbreak}{\char`\^}}%
            \def\PYGZam{\discretionary{\char`\&}{\Wrappedafterbreak}{\char`\&}}%
            \def\PYGZlt{\discretionary{}{\Wrappedafterbreak\char`\<}{\char`\<}}%
            \def\PYGZgt{\discretionary{\char`\>}{\Wrappedafterbreak}{\char`\>}}%
            \def\PYGZsh{\discretionary{}{\Wrappedafterbreak\char`\#}{\char`\#}}%
            \def\PYGZpc{\discretionary{}{\Wrappedafterbreak\char`\%}{\char`\%}}%
            \def\PYGZdl{\discretionary{}{\Wrappedafterbreak\char`\$}{\char`\$}}%
            \def\PYGZhy{\discretionary{\char`\-}{\Wrappedafterbreak}{\char`\-}}%
            \def\PYGZsq{\discretionary{}{\Wrappedafterbreak\textquotesingle}{\textquotesingle}}%
            \def\PYGZdq{\discretionary{}{\Wrappedafterbreak\char`\"}{\char`\"}}%
            \def\PYGZti{\discretionary{\char`\~}{\Wrappedafterbreak}{\char`\~}}%
        }
        % Some characters . , ; ? ! / are not pygmentized.
        % This macro makes them "active" and they will insert potential linebreaks
        \newcommand*\Wrappedbreaksatpunct {%
            \lccode`\~`\.\lowercase{\def~}{\discretionary{\hbox{\char`\.}}{\Wrappedafterbreak}{\hbox{\char`\.}}}%
            \lccode`\~`\,\lowercase{\def~}{\discretionary{\hbox{\char`\,}}{\Wrappedafterbreak}{\hbox{\char`\,}}}%
            \lccode`\~`\;\lowercase{\def~}{\discretionary{\hbox{\char`\;}}{\Wrappedafterbreak}{\hbox{\char`\;}}}%
            \lccode`\~`\:\lowercase{\def~}{\discretionary{\hbox{\char`\:}}{\Wrappedafterbreak}{\hbox{\char`\:}}}%
            \lccode`\~`\?\lowercase{\def~}{\discretionary{\hbox{\char`\?}}{\Wrappedafterbreak}{\hbox{\char`\?}}}%
            \lccode`\~`\!\lowercase{\def~}{\discretionary{\hbox{\char`\!}}{\Wrappedafterbreak}{\hbox{\char`\!}}}%
            \lccode`\~`\/\lowercase{\def~}{\discretionary{\hbox{\char`\/}}{\Wrappedafterbreak}{\hbox{\char`\/}}}%
            \catcode`\.\active
            \catcode`\,\active
            \catcode`\;\active
            \catcode`\:\active
            \catcode`\?\active
            \catcode`\!\active
            \catcode`\/\active
            \lccode`\~`\~
        }
    \makeatother

    \let\OriginalVerbatim=\Verbatim
    \makeatletter
    \renewcommand{\Verbatim}[1][1]{%
        %\parskip\z@skip
        \sbox\Wrappedcontinuationbox {\Wrappedcontinuationsymbol}%
        \sbox\Wrappedvisiblespacebox {\FV@SetupFont\Wrappedvisiblespace}%
        \def\FancyVerbFormatLine ##1{\hsize\linewidth
            \vtop{\raggedright\hyphenpenalty\z@\exhyphenpenalty\z@
                \doublehyphendemerits\z@\finalhyphendemerits\z@
                \strut ##1\strut}%
        }%
        % If the linebreak is at a space, the latter will be displayed as visible
        % space at end of first line, and a continuation symbol starts next line.
        % Stretch/shrink are however usually zero for typewriter font.
        \def\FV@Space {%
            \nobreak\hskip\z@ plus\fontdimen3\font minus\fontdimen4\font
            \discretionary{\copy\Wrappedvisiblespacebox}{\Wrappedafterbreak}
            {\kern\fontdimen2\font}%
        }%

        % Allow breaks at special characters using \PYG... macros.
        \Wrappedbreaksatspecials
        % Breaks at punctuation characters . , ; ? ! and / need catcode=\active
        \OriginalVerbatim[#1,codes*=\Wrappedbreaksatpunct]%
    }
    \makeatother

    % Exact colors from NB
    \definecolor{incolor}{HTML}{303F9F}
    \definecolor{outcolor}{HTML}{D84315}
    \definecolor{cellborder}{HTML}{CFCFCF}
    \definecolor{cellbackground}{HTML}{F7F7F7}

    % prompt
    \makeatletter
    \newcommand{\boxspacing}{\kern\kvtcb@left@rule\kern\kvtcb@boxsep}
    \makeatother
    \newcommand{\prompt}[4]{
        {\ttfamily\llap{{\color{#2}[#3]:\hspace{3pt}#4}}\vspace{-\baselineskip}}
    }
    

    
% Start the section counter at -1, so the Table of Contents is Section 0
   \setcounter{section}{-2}
% Prevent overflowing lines due to hard-to-break entities
    \sloppy
    % Setup hyperref package
    \hypersetup{
      breaklinks=true,  % so long urls are correctly broken across lines
      colorlinks=true,
      urlcolor=urlcolor,
      linkcolor=linkcolor,
      citecolor=citecolor,
      }

    % Slightly bigger margins than the latex defaults
    \geometry{verbose,tmargin=0.5in,bmargin=0.5in,lmargin=0.5in,rmargin=0.5in}


\begin{document}
    
    \maketitle
    
    

    
    \hypertarget{numerical-relativity-problems-chapter-2-the-31-deconposition-of-einsteins-equations}{%
\section{Numerical Relativity Problems Chapter 2: The 3+1 Deconposition
of Einstein's
Equations}\label{numerical-relativity-problems-chapter-2-the-31-deconposition-of-einsteins-equations}}

\hypertarget{authors-gabriel-m-steward}{%
\subsection{Authors: Gabriel M
Steward}\label{authors-gabriel-m-steward}}

    https://github.com/zachetienne/nrpytutorial/blob/master/Tutorial-Template\_Style\_Guide.ipynb

Link to the Style Guide. Not internal in case something breaks.

    \hypertarget{nrpy-source-code-for-this-module}{%
\subsubsection{\texorpdfstring{ NRPy+ Source Code for this
module:}{ NRPy+ Source Code for this module:}}\label{nrpy-source-code-for-this-module}}

None, save the pdf conversion at the bottom of this document.

\hypertarget{introduction}{%
\subsection{Introduction:}\label{introduction}}

Now we take a look into ``so how do we actually go about doing this?''
via Numerical Relativity by Baumgarte and Shapiro.

\hypertarget{other-optional}{%
\subsection{\texorpdfstring{ Other
(Optional):}{ Other (Optional):}}\label{other-optional}}

In order to fascilitate learning, whenever the opportunity arises Sympy
will be used.

\hypertarget{note-on-notation}{%
\subsubsection{Note on Notation:}\label{note-on-notation}}

Any new notation will be brought up in the notebook when it becomes
relevant.

\hypertarget{citations}{%
\subsubsection{Citations:}\label{citations}}

{[}1{]}
https://physics.stackexchange.com/questions/79157/square-bracket-notation-for-anti-symmetric-part-of-a-tensor
(3-way and up antisymmetry formula)

{[}2{]}
https://profoundphysics.com/christoffel-symbols-a-complete-guide-with-examples/
(Christoffel Symbols)

    \hypertarget{table-of-contents}{%
\section{Table of Contents}\label{table-of-contents}}

\[\label{toc}\]

\hyperref[p1]{Problem 1} (Initial Satisfaction of Constraints)

\hyperref[p2]{Problem 2} (Adjusting Electromagnetism, incomplete)

\hyperref[p3]{Problem 3} (Gauge adjustments)

\hyperref[p4]{Problem 4} (Coulomb Gauge)

\hyperref[p5]{Problem 5} (Rotation Free)

\hyperref[p6]{Problem 6} (Spacelike Entirely)

\hyperref[p7]{Problem 7} (Projected Tensors, incorrect)

\hyperref[p8]{Problem 8} (3D Derivatives, incomplete)

\hyperref[p9]{Problem 9} (The Inconsistent Product Rule)

\hyperref[p10]{Problem 10} (Rotation Free Returns)

\hyperref[p11]{Problem 11} (The Extrinsic Curvature Vanishes)

\hyperref[p12]{Problem 12} (Spacelike Acceleration, incomplete)

\hyperref[p13]{Problem 13} (Acceleration Lapse, incomplete)

\hyperref[p14]{Problem 14} (Schwartzchild Acceleration)

Appendix A Break Begin

\hyperref[pa1]{Problem A1} (Equivalence of Covariant and Partial
Derivatives in the Lie Derivative)

\hyperref[pa2]{Problem A2} (The Product Rule Holds for Lie Derivatives)

\hyperref[pa3]{Problem A3} (Exterior Derivative Commutation)

\hyperref[pa4]{Problem A4} (Lie Dragging of a Secondary Field)

Appendix A Break End

\hyperref[p15]{Problem 15} (Return of the Reimann Tensor and Index
Salad, incomplete)

\hyperref[p16]{Problem 16} (Spatial Vector Funtimes)

34

\hyperref[latex_pdf_output]{PDF} (turn this into a PDF)

    \hypertarget{problem-1-back-to-top}{%
\section{\texorpdfstring{Problem 1 {[}Back to
\hyperref[toc]{top}{]}}{Problem 1 {[}Back to {]}}}\label{problem-1-back-to-top}}

\[\label{P1}\]

\emph{Demonstrate that the constraint equations 2.2, if satisfied
initially, are automatically satisfied at later times when the
gravitational field is evolved by using the dynamical equations 2.3.
Equivalently, show that the relation
\(\partial_t (G^{a0} - 8\pi T^{a0}) = 0\) will be satisfied at the
initial time \(x^0=t\), hence conclude that 2.2 will be satisfied at
\(x^0 = t+\delta t\). Hint: use the Bianchi identities together with the
equations of energy-momentum conservation to evaluate
\(\nabla_b(G^{ab}-8\pi T^{ab})\) at \(x^0=t\)}

    2.2: \(G^{a0} = 8\pi T^{a0}\)

2.3: \(G^{ij} = 8\pi T^{ij}\)

1.23 (The Bianchi Identities):
\(\nabla_e R_{abcd} + \nabla_d R_{abec} + \nabla_c R_{adbe} = 0\)

And naturally energy-momentum is conserved. Though, as we know, energy
is rather hard to define in relativity\ldots{}

    Note that it at first seems like we are being asked to solve it
different ways, but the ``equivalently'' really is just giving us a hint
on where to go and what to do and why. Naturally if the time derivative
of a quantity is zero, that quantity is not going to change over an
infinitesimal adjustment (\(\delta t\) in our formulation). So let's
just start working it out the way the hint suggests.

\[ \nabla_b(G^{ab}-8\pi T^{ab}) \]

\[ = \nabla_bG^{ab}- \nabla_b8\pi T^{ab} \]

\[ = g_{bf}\nabla_bG^{a}_f - g_{bf}8\pi \nabla_b T^{a}_f \]

    Now here's the thing, on page 6 we note that as a consequence of the
Branchi identities, the covariant divergence of G vanishe AND so does T,
via 1.34. Thus we have two terms that are zero. Specifically, zero
\emph{always}. Now, granted, we haven't proven WHY the Branchi
identities demand this, but it does give us the rather obvious case of

\[ = 0 \]

That said, this is not what we are looking for, that's the ENTIRE
covariant derivative. What WE want is to show that JUST the temporal
portions go to zero as well. So let's back up a bit, all the way to the
General Relativity book, which will henceforth be refered to as GR.
GR6.97 actually gives ust he end result of contracting the Bianchi
identities.

\[ \nabla_u(2R^u_l-\delta^u_kR) = 0 \]

Save for a the placement of hte 2 this is identical to the formulation
for the G in terms of R. Which is WHY we can say covariant derivatives
of G are always zero.

\[ 0 = \nabla_jG^{ij}- \nabla_j8\pi T^{ij} + \nabla_0G^{a0}- \nabla_08\pi T^{a0} + \nabla_bG^{0b}- \nabla_b8\pi T^{0b} - \nabla_0G^{00} + \nabla_08\pi T^{00} \]

What we have here is spatial components, temporal component column,
temporal component row, and then a subtraction of the 00 case which was
counted twice. The symmetry of G and T does not allow us to simplify as
the derivative on identical components relies on a potentially different
variable.

    We also note from GR that the Bianchi identities also imply the
covariant derivative of T is zero, and this specifically \emph{IS} the
local conservation of energy and momentum. From our earlier work we now
expand the covariant derivative out, yet we know it must be zero still.

    \[ 0 = g_{bf}\left(\partial_bG^{a}_f+\Gamma^a_{db}G^{d}_f-\Gamma^d_{fb}G^{a}_d\right) - g_{bf}8\pi \left(\partial_bT^{a}_f+\Gamma^a_{db}T^{d}_f-\Gamma^d_{fb}T^{a}_d\right) \]

Since we're looking for a time relation, might as well make the
covariant derivative time specifically.

\[ \Rightarrow 0 = g_{tf}\left(\partial_tG^{a}_f+\Gamma^a_{dt}G^{d}_f-\Gamma^d_{ft}G^{a}_d\right) - g_{tf}8\pi \left(\partial_tT^{a}_f+\Gamma^a_{dt}T^{d}_f-\Gamma^d_{ft}T^{a}_d\right) \]

We can arrange terms together and divide out the metric\ldots{}

\[ \Rightarrow 0 = \partial_tG^{a}_f - 8\pi\partial_tT^{a}_f+\Gamma^a_{dt}G^{d}_f-\Gamma^d_{ft}G^{a}_d-8\pi\Gamma^a_{dt}T^{d}_f+8\pi\Gamma^d_{ft}T^{a}_d \]

    Here's the curious thing about this--the Christoffel symbols on both
sides match with flipped signs, and the signs for G and T are in the
same place. We have NOT applied any restrictions to G or T, their
indeces are still free, so they hit everything. Remember that in general
\[G^{ab} = 8\pi T^{ab}\]. Which means that since the Christoffel symbols
match, the four trailing terms will CANCEL!

\[ \Rightarrow 0 = \partial_t(G^{a}_f - 8\pi T^{a}_f ) \]

Which is what we sought. Which is far more trivial than we made it by
overthinking. This kind of HAS to be true, however the subtraction used
on the Christoffel symbols would not apply here, as the derivatives are
being taken and the derivative of 0 need not actually be 0.

We know that the conditions of Einstein's equations are satisfied
initially, and we have just seen that with said conditions the temporal
derivative is zero, thus so long as Einstein's equations are satisfied
the evolution is valid.

Which is good, but remember that such assumptions vanish at cosmological
scales due to the Cosmological Constant.

    \hypertarget{problem-2-back-to-top}{%
\section{\texorpdfstring{Problem 2 {[}Back to
\hyperref[toc]{top}{]}}{Problem 2 {[}Back to {]}}}\label{problem-2-back-to-top}}

\[\label{P2}\]

\emph{Show that the evolution equations 2.11 and 2.12 preserve the
constraint 2.5. i.e., show that}

\[ \frac{\partial}{\partial t} \mathcal{C}_E = 0 \]

    2.5: \$\mathcal{C}\_E = D\_i E\^{}i - 4\pi \rho = 0 \$

2.11: \$ \partial\_t A\_i = -E\_i - D\_i\Phi \$

2.12: \$ \partial\_t E\_i = D\_iD\^{}jA\_j - D\textsuperscript{jD\_jA}i
- 4\pi j\_i \$

    OKAY so there are pages upon pages of scratch work here that have gone
absolutely nowhere, time to move on.

    \hypertarget{problem-3-back-to-top}{%
\section{\texorpdfstring{Problem 3 {[}Back to
\hyperref[toc]{top}{]}}{Problem 3 {[}Back to {]}}}\label{problem-3-back-to-top}}

\[\label{P3}\]

\emph{Show that a transformation to a new ``tilded'' gauge according to}

\[ \tilde\Phi = \Phi - \frac{\partial \Lambda}{\partial t} \]

\[ \tilde{A}_i = A_i + D_i \Lambda \]

\emph{leaves the physical fields \(E^i\) and \(B^i\) unchanged}

We do note that to be a gauge transformation the shifts have to be
sufficiently small, so we know that the additional terms are small in
comparison to the original ones. Thus we consider \(\Lambda\) and its
derivative to be small.

    Let's go with a relation that has E in it:

\[ \partial_t A_i = -E_i -D_i \Phi \]

\[ \Rightarrow E_i = - \partial_t A_i -D_i \Phi \]

So if we replace our adjustments with our substitutions this BETTER be
the same.

\[ \Rightarrow E_i = - \partial_t A_i - \partial_t D_i \Lambda - D_i \Phi + D_i \partial_t \Lambda \]

\[ \Rightarrow E_i = - \partial_t A_i- D_i \Phi \]

So yes E has to be the same.

    But what of B? Well, we can kind of think of this logically. E and B are
intrisically linked, if E changes, B changes, if E stays the same, B
stays the same. Thus, if one is unchanged, both are unchanged. 2.8
allows one to be calculated from the other.

    \hypertarget{problem-4-back-to-top}{%
\section{\texorpdfstring{Problem 4 {[}Back to
\hyperref[toc]{top}{]}}{Problem 4 {[}Back to {]}}}\label{problem-4-back-to-top}}

\[\label{P4}\]

\emph{In the so-called radiation, Coulomb, or transverse gauge, the
divergence (or longitudinal) part of A is chosen to vanish.
\(D_iA^i = 0\). So that \(A_i\) is purely transverse. Show that in this
gague \(\Phi\) plays the role of a Coulomb potential,
\(D^iD_i\Phi = -4\pi \rho\) and that the vector potential \(A_i\)
satisfies a simple inhomogeneous wave equation}

\[ \square A_i = -\partial^2_t A_i + D^j D_j A_i = -4\pi j_i + D_i(\partial_t\Phi) \]

    This appears to be a problem in two parts, first we have to show that
the potential is as always, knowing only that the physical covariant
derivative for A always goes to zero. This knowledge does not
immediately remove most restrictions, however--after all, derivatives of
zero can still be something other than zero. The bane of our existence.

NOTE: there was a section here about j=0. It does not.

    What DOES help us solve it is 2.5, which, when reworded, states that
\(D_iE^i = 4\pi \rho\). With this knowledge, we can apply a covariant
derivative on 2.11

\[ D^i \partial_t A_i = -D^iE_i - D^i D_i \Phi\]

    The last term is the one we want. The left-hand side looks
unevaluatable, but since the divergence vanishes, it better vanish
equally at all times, thus the derivative of zero is zero in this case.
This leaves us with:

\[ \Rightarrow D^i D_i \Phi = -D^iE_i \]

\[ \Rightarrow D^i D_i \Phi = -4\pi \rho \]

Which is what it should be.

    And now the second part. We need to show:

\[ \square A_i = -\partial^2_t A_i + D^j D_j A_i = -4\pi j_i + D_i(\partial_t\Phi) \]

Which is a relation of the square laplacian. The left hand side andt he
middle follow automatically. However, the last step does not. Also it
seems to imply that the current, j, is not zero, like we determined
before.

    From the middle step, we can expand the time derivative of \(A_i\) via
2.11

\[ -\partial^2_t A_i + D^j D_j A_i \]

\[ = -\partial_t (-E_i - D_i \Phi) + D^j D_j A_i \]

\[ = \partial_tE_i + D_i \partial_t \Phi + D^j D_j A_i \]

With this we ave one of the terms we want. We can use 2.12 for the time
derivative of E to get\ldots{}

\[ = D_iD^jA_j - D^jD_jA_i - 4\pi j_i + D_i \partial_t \Phi + D^j D_j A_i \]

The divergence vanishes, and two terms cancel, leaving us with:

\[ = - 4\pi j_i + D_i \partial_t \Phi \]

Which is what we sought.

    \hypertarget{problem-5-back-to-top}{%
\section{\texorpdfstring{Problem 5 {[}Back to
\hyperref[toc]{top}{]}}{Problem 5 {[}Back to {]}}}\label{problem-5-back-to-top}}

\[\label{P5}\]

\emph{Show that the normalized 1-form \(\omega_a = \alpha\Omega_a\) is
rotation-free \(\omega_{[a} \nabla_b \omega_{c]} = 0\)}

    So first of all we need to be careful with the notation-- the brackets
refer to the antisymmetric portion of the tensor. in this case it would
be the tensor composed of all these things smashed together, which is a
three-way antisymmetric tensor. We use \hyperref[1]{1} to grab the
three-index antisymmetry formula.

\[ \omega_{[a} \nabla_b \omega_{c]} \]

\[ = \alpha^2 \Omega_{[a} \nabla_b \Omega_{c]} \]

\[ = \alpha^2 \nabla_{[a}t \nabla_b \nabla_{c]}t \]

2.20 lets us know that \(\nabla_{[a}\nabla_{b]}t = 0\). Perhaps we can
make use of this. This fact may be rather obvious since derivative order
can be changed at will, but the extra t we have in the middle here
complicates things now. The antisymmetric expansion gives us:

\[ = \alpha^2 \frac{1}{3!} \left[ \nabla_{a}t \nabla_b \nabla_{c}t + \nabla_{c}t \nabla_a \nabla_{b}t + \nabla_{b}t \nabla_c \nabla_{a}t - \nabla_{a}t \nabla_c \nabla_{b}t - \nabla_{b}t \nabla_a \nabla_{c}t - \nabla_{c}t \nabla_b \nabla_{a}t \right]\]

We note that every term cancels because the last two \(\nabla\)s can be
shuffled at will, meaning a + term will always cancel with a - term.
Thus, it goes to zero.

Now as for what exactly this means physically\ldots{} what DOES
rotation-free mean? Presumably it means it's immune to rotation or lacks
the capacity to rotate, but the physical intuition is not coming today.

    \hypertarget{problem-6-back-to-top}{%
\section{\texorpdfstring{Problem 6 {[}Back to
\hyperref[toc]{top}{]}}{Problem 6 {[}Back to {]}}}\label{problem-6-back-to-top}}

\[\label{P6}\]

\emph{Show that \(\gamma^a_b v^b\), where \(v^b\) is an arbitrary
spacetime vector, is purely spatial.}

Just so we're clear, \(v^b = (t,x,y,z).\) The final result of what we
get here better not have any time in it.

2.30 gives us the projection operator,
\(\gamma^a_b = \delta^a_b + n^an_b\)

    All together, we get the result

\[ \delta^a_b v^b + n^an_bv^b \]

Notably we only sum over b, a is what our final vector is goign to be.

\[ = v^a + n^an_bv^b \]

This initial projection just makes use of the dirac delta, only the term
that matches the index will survive. As for the other term, however, the
terms of the sum don't just automatically vanish.

    The way to prove something is spatial in this case is to prove that
nothing about it lies along the normal vector to time, which is to say,
we apply \(n^a\) to it again and see where we end up:

\[ n^av^a + n^an^an_bv^b \]

In essence, this better equal zero!

    Since we are trying to equal zero here, we can pull out a metric from
every term to lower the index of a. We actually pull the dirac delta
back out since it'll be useful now.

\[ g^{aa}n_a \delta^a_b v^b + g^{aa}n_an^an_bv^b \]

\[ = g^{aa}n_b v^b - g^{aa}n_bv^b = 0 \]

Where the last step is \(n^an_a = -1\). And we're done!

    \hypertarget{problem-7-back-to-top}{%
\section{\texorpdfstring{Problem 7 {[}Back to
\hyperref[toc]{top}{]}}{Problem 7 {[}Back to {]}}}\label{problem-7-back-to-top}}

\[\label{P7}\]

\emph{Show that for the second rank tensor \(T_{ab}\) we have}

\[ T_{ab} = \perp T_{ab} - n_an^c \perp T_{cb} - n_bn^c \perp T_{ac} + n_an_bn^cn^dT_{cd} \].

    Curiously, this seems to be a case of working backward. We have
projections of the tensor, we need to reclaim the original tensor with
them. (Or think of it as putting the tensor in terms of its
projections\ldots{} sorta. The motivation is a bit unclear). We are
warned in teh book to use the projection symbolw ith some care since it
only applies to the free indices of the tensor that it operates on.

First step, let's expand all the projections via 2.31.

\[ T_{ab} = \gamma^c_a\gamma^d_b T_{cd} - n_an^c \gamma^a_c \gamma^d_b T_{ad} - n_bn^c \gamma^d_a \gamma^b_c T_{bd} + n_an_bn^cn^dT_{cd} \].

    Now we recall from 2.24 that every normal vector \emph{contains} a
metric. The best part is we can choose exactly what form that metric
takes for maximum simplification. More specifically, we choose indeces
such that we end up with forms akin to
\(g^{ab} g_{bc} = g^a_c = \delta^a_c\) and deltas are really easy to
make vanish. Furthermore, we played with indeces and found:

\[ n^a = -g^{ab}\omega_b \Rightarrow n_c = -g_{ac}g^{ab}\omega_b = -\delta^b_c \omega_b = -\omega_c\]

Which makes this quite a bit simpler. Watch the negative sign! (Which
doesn't show up here since we always end up with two of them)

This turns out to not be all that helpful but it IS useful to knnow.
Instead of doing this, let's just shuffle the index salad to make all
the T terms match.

    \[ T_{ab} = \gamma^c_a\gamma^d_b T_{cd} - n_cn^a \gamma^c_a \gamma^d_b T_{cd} - n_cn^b \gamma^d_a \gamma^c_b T_{cd} + n_an_bn^cn^dT_{cd} \]
\[ = \left( \gamma^c_a\gamma^d_b - n_cn^a \gamma^c_a \gamma^d_b - n_cn^b \gamma^d_a \gamma^c_b + n_an_bn^cn^d \right) T_{cd} \]

    Index salad: a and b are entirely interchangeable and arbitrary.

\[ = \left( \gamma^c_a\gamma^d_b - n_cn^a \gamma^c_a \gamma^d_b - n_cn^b \gamma^d_b \gamma^c_a + n_an_bn^cn^d \right) T_{cd} \]

Somehow, we need to show that the term in the parentheses is equivalent
to \(\delta^c_a\delta^d_b\). It is ALMOST factorable itno a binomial
that can be outright split up, but nto quite. If we could somehow lower
the cd and raise the ab without actually \emph{changing} the final term,
we would get exactly what we want.

However, it sure seems like that is NOT true. In fact in our notes we
ended up proving that, in any specific case, \(n_an^b \neq n_bn^a\).
Same goes for the last term, doing it twice does not ``undo'' the
problem, unless we were in the Euclidean metric, which we most
definitely are not.

    Unless we were in the Euclidean metric.

    \sout{PREVIOUS WORK WE'RE PRETTY SURE IS WRONG}

    Each term in the sum is independent and ends with a particular tensor T.
We can re-arrange the index to put it in ab terms, giving us:

\[ = (\gamma^a_c\gamma^b_d - n_an^c \gamma^a_c \gamma^b_d - n_bn^c \gamma^a_d \gamma^b_c + n_cn_dn^an^b)T_{ab} \]

Now we engage in the practice of Index Salad trying to get the term in
parentheses to reduce to 1. Curious, this almost looks like a binomial
expansion. Only a and b matter for interacting with the Tensor, so we
can shuffle c and d as we wish, getting some nice like terms:

\[ = (\gamma^a_c\gamma^b_d - n_an^c \gamma^a_c \gamma^b_d - n_bn^d \gamma^a_c \gamma^b_d + n_cn_dn^an^b)T_{ab} \]

The indeces on the last term aren't neat, we want up to be down and down
to be up. Unfortunately it does not seem possible to just ``choose
metrics'' that will make everything cancel when we adjust the indeces.

Now we recall from 2.24 that every normal vector \emph{contains} a
metric. The best part is we can choose exactly what form that metric
takes for maximum simplification. More specifically, we choose indeces
such that we end up with forms akin to
\(g^{ab} g_{bc} = g^a_c = \delta^a_c\) and deltas are really easy to
make vanish. Furthermore, we played with indeces and found:

\[ n^a = -g^{ab}\omega_b \Rightarrow n_c = -g_{ac}g^{ab}\omega_b = -\delta^b_c \omega_b = -\omega_c\]

Which makes this quite a bit simpler. Watch the negative sign! (Which
doesn't show up here since we always end up with two of them)

    \[ = (\gamma^a_c\gamma^b_d - n_an^c \gamma^a_c \gamma^b_d - n_bn^d \gamma^a_c \gamma^b_d + n_cn_dn^an^b)T_{ab} \]

\ldots Wait hold on while the relation above appears potentially useful,
we still cannot decomposet his into a square binomial like it looks like
we SHOULD. If we COULD we would end up with something along the lines
of:

\[ = (\gamma^a_c - n_an^c)(\gamma^b_d-n_bn^a)T_{ab}\]

2.30 informs us that the binomials in parentheses are delta functions!
Specifically:

\[ = (\delta^a_c)(\delta^b_d)T_{ab}\]

\[ = T_{cd} = T_{ab}\]

So yes this LOOKS about right. This would all be automatic if we could
prove \(n_an_bn^cn^d = n_cn_dn^an^b\)

Wait, hold on, agh, there's the obvious answer right there. We CAN
shuffle the indeces around, even a and b, because while they WOULD
interact with T, they actually in the end DO NOT. Sure, the two diracs
change the indeces from ab to cd\ldots{} but then we change them right
bac. In effect, there is no change at all to the tensor. Which means
that even if we shuffle indeces around of the summing terms, the result
is still 1. Hah!

\ldots Yes this is a little shaky but it has to be true, though as for
why we can't do it without this step we are not sure.

If we split each n up into its component parts, we \emph{can} get each
to behave, ending up with the ``equality''

\[ \omega_a\omega_b\omega_c\omega_d (-g^{cd})^2 = \omega_a\omega_b\omega_c\omega_d (-g^{ab})^2 \]

Which, if we are allowed to shuffle the indeces, yes is true. IF we are
allowed to shuffle the indeces. It seems like we should due to the order
of operations we have set up, but we are hesitant.

    \hypertarget{problem-8-back-to-top}{%
\section{\texorpdfstring{Problem 8 {[}Back to
\hyperref[toc]{top}{]}}{Problem 8 {[}Back to {]}}}\label{problem-8-back-to-top}}

\[\label{P8}\]

\emph{Show that the 3-dimensional covariant derivative is compatible
with the spatial metric \(\gamma_{ab}\), that is, show that
\(D_a\gamma_{bc} = 0\)}

    Let's write this out carefully. When acting on a scalar, the covariant
derivative follows 2.40: \(D_af = \gamma_a^b \nabla_b f\) For our
purposes replace b with d since we have other indeces around.

Acting on other tensors involves tacking on more \(\gamma\) functions
based on the number of indeces. For instance, in our case:

\[ D_a\gamma_{bc} =  \gamma^b_e \gamma^c_f \gamma_a^d \nabla_d\gamma_{ef} \]

We just need to show that this is in fact zero.

    \ldots Yeah haven't the foggiest idea how to do this.

    \hypertarget{problem-9-back-to-top}{%
\section{\texorpdfstring{Problem 9 {[}Back to
\hyperref[toc]{top}{]}}{Problem 9 {[}Back to {]}}}\label{problem-9-back-to-top}}

\[\label{P9}\]

\emph{Show that for a scalar product \(v^aw_a\) the Leibnitz rule}

\[D_a(v^bw_b) = v^bD_aw_b + w_bD_av^b\]

\emph{Only holds if \(v^a\) and \(v_b\) are purely spatial}

Oho, trying to tell me the product rule is wrong are you? Well\ldots{}
yeah that makes sense.

    Anyway a scalar product is a SCALAR so the 3D derivative becomes

\[ \gamma^c_a \nabla_c (v^bw_b) \]

Now we know the product rule applies to the normal covariant derivative,
so we can use it here.

\[ \gamma^c_a \left[ \nabla_c (v^b) w_b + v^b \nabla_c (w_b) \right] \]

Now, it sure seems like no matter what \(\gamma\) can just be pulled in
and show that the product rule still applies\ldots{} but let's think
about what \(\gamma\) IS. The form we have is 2.30, which is the
projection of a 4-dimensional tensor into a spatial slice. Well, if v
and w are spatial, then the result is obvious: projection does nothing
if it's already projected, so the rule must still hold.

But that's only half of the proof. We need to show that it is NOT true
when v or w or both has a temporal component. This is actually easy to
see: let there be v = (1,x,y,z) and w = (0,x,y,z). When they have their
fancy dot product they just end up with ``\(x^2+y^2+z^2\)'' the temporal
components completley cancel, then everything goes throgh and acts just
fine.

For the side of the product rule where D acts on v alone will remove the
temporal component entirely. However, in the case where D acts on w, the
temporal component in v is still there. Thus when \(\gamma\) goes
through it WILL act on that vector and project it. Which is to say
\emph{a situation may be constructed where \(\gamma\) changes the left
side of the rule without the right, necessarily breaking the equality}.
Essentially what we've done is proof by counterexample, ableit somewhat
generally.

    Put another way, if either w or v has a temporal component, when
\(\gamma\) goes through the vector by necessity must be altered to
remove it, it doesn't matter how exactly. However, on the opposite side,
there is no alteration occuring, that is to say, ``no change''.

The end result of all this would be something akin to:

\[D_a(v^bw_b) = v^{b '} D_aw_b + w_bD_av^b\]

Where the prime vector has been altered in some fashion. This is not a
general rule, this is just our specific case.

How exactly \(\gamma\) acts can be left vague. Good for us!

    \hypertarget{problem-10-back-to-top}{%
\section{\texorpdfstring{Problem 10 {[}Back to
\hyperref[toc]{top}{]}}{Problem 10 {[}Back to {]}}}\label{problem-10-back-to-top}}

\[\label{P10}\]

\emph{Show that the twist \(\omega_{ab}\) has to vanish as a consequence
of \(n^a\) being rotation-free. See \textbf{Problem 5}.}

    What \textbf{Problem 5} actually shows is that \(\omega_a\) is rotation
free. That said, since it is rotation free, and \(n^a\) is constructed
from it and the metric, obviously the same holds for it.

So the question is why does this make the twist vanish? (Such an
excellent name, the Twist\ldots) The twist is given by 2.48.

\[ \omega_{ab} = \gamma_a^c \gamma_b^d \nabla_{[c}n_{d]} \]

And the rotation-free requirement is 2.23.

\[\omega_{[a} \nabla_b \omega_{c]} = 0 = n_{[a} \nabla_b n_{c]} \]

\textbf{Problem 7} can confirm that this is actually a direct
substitution, just with a negative sign. Notably since there are two of
them the signs cancel.

    Note that all \(\omega\) are actually \(\alpha\Omega\) which means we
actuallky have:

\[ \omega_{ab} = \gamma_a^c \gamma_b^d \alpha \nabla_{[c}\Omega_{d]} \]

And from 2.20 we know that that antisymmetric portion goes to zero. Thus
everything goes to zero and the twist vanishes.

\ldots Seems too simple\ldots{}

    \hypertarget{problem-11-back-to-top}{%
\section{\texorpdfstring{Problem 11 {[}Back to
\hyperref[toc]{top}{]}}{Problem 11 {[}Back to {]}}}\label{problem-11-back-to-top}}

\[\label{P11}\]

\emph{Show that the extrinsic curvature of t=constant hypersurfaces of
the Schwarzchild metric 2.35 vanishes.}

    The curvature is given by 2.49

\[ K_{ab} = -\gamma^c_a\gamma^d_b\nabla_c n_d \].

We just need to evaluate this for Schwarzchild geometry. Which we
actually have outlined in other places. The spatial metric is 2.39

\[ \gamma_{ab} = \left( 1 + \frac{M}{2r}\right)^4 diag(0,1,r^2,r^2sin^2\theta) \]

And the normal vector is

\[ n^a = -g^{ab}\omega_b = \frac{1+M/2r}{1-M/2r}(1,0,0,0) \]

Now, these are not exactly in the right forms. But we can alter the
curvature equation to \emph{make} it the right forms!

    \[ K_{ab} = - g^{cf}g^{dg}g_{dh} \gamma_{fa}\gamma_{gb}\nabla_c n^h \].

    Now, in OUR metric, these terms only exist in certain locations. The
easiest result is to remove all non-diagonals.

\[ K_{ab} = - g^{ca}g^{db}g_{dh} \gamma_{aa}\gamma_{bb}\nabla_c n^h \].

\[ K_{ab} = - g^{ca} \delta^b_h \gamma_{aa}\gamma_{bb}\nabla_c n^h \].

\[ K_{ab} = - g^{ca} \gamma_{aa}\gamma_{bb}\nabla_c n^b \].

    And that seals it! The only time n exists is when b=t, but when b=t, the
\(\gamma\) does not exist! and the other way around is true as well
meaning that, no matter what, the extrinsic curvature will vanish. Now
at first we wonder why we need to think about why t=const here, what if
t wasn't const? Well. Remember that the hypersurfaces we've been
modeling this entire time have t=const as the assumption the
\emph{entire time}. It's baked in to what we've done above.

    \hypertarget{problem-12-back-to-top}{%
\section{\texorpdfstring{Problem 12 {[}Back to
\hyperref[toc]{top}{]}}{Problem 12 {[}Back to {]}}}\label{problem-12-back-to-top}}

\[\label{P12}\]

\emph{Show that the acceleration \(a_a\) is purely spatial,
\(n^aa_a = 0\)}

    Have worked on this one for QUITE some time, ended up with:

\[ n^aa_a = -\alpha^3 g^{ac} g^{bd} \nabla_c t \nabla_d t \nabla_b \nabla_a t \]

Since t is a scalar function 5.53 from General Relativity can give us:

\[ = -\alpha^3 g^{ac} g^{bd} \partial_c t \partial_d t \partial_b \partial_a t \]

    Which does not seem to have a way to evaluate to zero, at least not
obviously. Taking two derivatives of a scalar function is not guaranteed
to be zero by any means. So what's going on here?

Moving on, too much time spent.

    \hypertarget{problem-13-back-to-top}{%
\section{\texorpdfstring{Problem 13 {[}Back to
\hyperref[toc]{top}{]}}{Problem 13 {[}Back to {]}}}\label{problem-13-back-to-top}}

\[\label{P13}\]

\emph{Show that the acceleration \(a_a\) is related to the lapse
\(\alpha\) according to}

\[ a_a = D_a ln \alpha \]

Given how little luck we had with the previous problem, perhaps it is
unsurpising that this one is unsolved.

    \hypertarget{problem-14-back-to-top}{%
\section{\texorpdfstring{Problem 14 {[}Back to
\hyperref[toc]{top}{]}}{Problem 14 {[}Back to {]}}}\label{problem-14-back-to-top}}

\[\label{P14}\]

\emph{Find the acceleration \(a_a\) for the normal observer 2.38 in
Schwarzchild spacetime.}

The definition of acceleration is \(a_a = n^b \nabla_b n_a\). For the
Schwarzchild metric we would need to grab Christoffels to deal with
this\ldots{} and for the alternative method we would ALSO need them so
egh let's just go grab them from \hyperref[2]{2}. Now that we have them,
we can use 2.38:

\[ n^a = \frac{1+M/2r}{1-M/2r}(1,0,0,0) \]

This means, rather obviously, that only \(n^t\) actually exists.

    The normal observer is the one moving along the normal vector, which
essentially means that the acceleration itself is only happening in the
time component as well. (We can see this since the xyz terms in our
expression all vanish.)

Curiously the sum over the b index also goes to nothing except for t,
which leaves us with

\[ a_t = n^t \nabla_t n_t\].

Of course, the issue is that our vector n isn't in the right index form,
so we have to:

\[ = n^t \nabla_t g_{tt} n^t\].

    Now we happen to just know what the Schwarzchild metric is, so this
isn't an issue. The tt component is \(-\frac{1-M/2r}{1+M/2r}\)
which\ldots{} well would you look at that it reduces the part in the
derivative to -1. And that\ldots{} makes the entire thing zero.

But wait, didn't we say there should be temporal acceleration?

Yes. But remember this is the physical acceleration, there's no temporal
componenent. So\ldots{} Yeah the normal observer has no acceleration.

    \hypertarget{a.1-an-aside-on-lie-derivatives}{%
\section{A.1 An Aside on Lie
Derivatives}\label{a.1-an-aside-on-lie-derivatives}}

The book introduced Lie Derivatives and said to refer to the appendix
and guess what, there are problems back there, so we're going to DO
them.

\hypertarget{problem-a1-back-to-top}{%
\section{\texorpdfstring{Problem A1 {[}Back to
\hyperref[toc]{top}{]}}{Problem A1 {[}Back to {]}}}\label{problem-a1-back-to-top}}

\[\label{PA1}\]

\emph{Show that the expression}

\[ \mathcal{L}_{\textbf X} T^a_b = X^c \nabla_c T^a_b - T^c_b \nabla_c X^a + T^a_c \nabla_b X^c \]

\emph{wehre \(\nabla_a\) denotes a covariant derivative with a symmetric
connection, is equivalent to A.8}

\$A.8: \mathcal{L}\_\{\textbf X\} T\^{}a\_b = X\^{}c \partial\_c
T\^{}a\_b - T\^{}c\_b \partial\_c X\^{}a + T\^{}a\_c \partial\_b X\^{}c
\$

    The symmetric connection merely means theat the Christoffels are
symmetric on their lower indeces. Which\ldots{} well we kind of usually
assume they are but in the GENERAL Lie Derivative case they may not be.
Regardless, what we ultimately need here is to show that every
Christoffel term vanishes, leaving only the partial derivative terms.
For that, we turn to our book on General Relativity. Equations 6.33
through 6.35 give us all the information we need about covariant
derivatives acting on tensors. We already know the partial derivative
terms are prsent in A.8, so we actually want to show that:

\[ 0 = X^c(\Gamma^a_{uc} T^u_b - \Gamma^u_{cb} T^a_u) - T^c_b(\Gamma^a_{uc} X^u) + T^a_c(\Gamma^c_{ub} X^u) \]

    Note that every term is summed over both c and u. WIth some clever index
shuffling, we can arrive at:

\[ \Rightarrow 0 = X^u\Gamma^a_{cu} T^c_b - X^u\Gamma^c_{ub} T^a_c - T^c_b\Gamma^a_{uc} X^u + T^a_c\Gamma^c_{ub} X^u \]

    Which only cancels if the Christoffels in the first and third terms can
shuffle their lower indeces. Which is, in fact, a thing we were given.

    \hypertarget{problem-a2-back-to-top}{%
\section{\texorpdfstring{Problem A2 {[}Back to
\hyperref[toc]{top}{]}}{Problem A2 {[}Back to {]}}}\label{problem-a2-back-to-top}}

\[\label{PA2}\]

*Show that \$\mathcal{L}\_\{\textbf X\}(\nu\^{}a\omega\emph{b) =
\nu\^{}a \mathcal{L}}\{\textbf X\}\omega\_b + \omega\emph{b
\mathcal{L}}\{\textbf X\}(\nu\^{}a) \$*

Ah, our good friend the product rule.

This is not as trivial as it appears to be, for the equation relating
the Lie derivative of a tensor is separate from that of the Lie
derivative of a vector/one form. A.8 is the tensor, A.13 is the vector,
and A.14 is the one form. Let's make the relation explicit.

    \[\mathcal{L}_{\textbf X}(\nu^a\omega_b) = \nu^a \mathcal{L}_{\textbf X}\omega_b + \omega_b \mathcal{L}_{\textbf X}(\nu^a) \]

\[\Rightarrow X^c \nabla_c (\nu^a\omega_b) - (\nu^c\omega_b)\nabla_c X^a + (\nu^a\omega_c)\nabla_b X^c = \nu^a X^c \nabla_c \omega_b + \nu^a \omega_c \nabla_b X^c + \omega_b X^c \nabla_c \nu^a  - \omega_b \nu^c \nabla_c X^a \]

Using the covariant derivative version to make everything match up. Now
covariant derivatives DO obey the product rule.

    \[\Rightarrow X^c \nu^a \nabla_c \omega_b + X^c \omega_b \nabla_c \nu^a- (\nu^c\omega_b)\nabla_c X^a + (\nu^a\omega_c)\nabla_b X^c = \nu^a X^c \nabla_c \omega_b + \nu^a \omega_c \nabla_b X^c + \omega_b X^c \nabla_c \nu^a  - \omega_b \nu^c \nabla_c X^a \]

\[\Rightarrow X^c \nu^a \nabla_c \omega_b + X^c \omega_b \nabla_c \nu^a = \nu^a X^c \nabla_c \omega_b  + \omega_b X^c \nabla_c \nu^a  \]

Which is now obviously true.

    \hypertarget{problem-a3-back-to-top}{%
\section{\texorpdfstring{Problem A3 {[}Back to
\hyperref[toc]{top}{]}}{Problem A3 {[}Back to {]}}}\label{problem-a3-back-to-top}}

\[\label{PA3}\]

\emph{Show that for a p-form \(\tilde{\pmb{\Omega}}\), \$
\mathcal{L}\emph{\{\pmb X\}\tilde{\pmb d}\tilde{\pmb{\Omega}} =
\tilde{\pmb d}\mathcal{L}}\{\pmb X\}\tilde{\pmb{\Omega}} \$}

Now that's a mess of bolding.

    Okay so annoyingly the trick to how to deal with this comes from A.18,
which is in the NEXT SECTION and we didn't even look at it, harumph.
Anyway, the expression above is annoying, let's write it out a little
differently.

\[ \mathcal{L}_{\pmb X} \partial_a \Omega_{bcdef...} \]

Now we know the one-form is just the ``gradient'' (or what we often
think of as the gradient), and all it does is take partial derivatives
of every single component in the p-form.

Well guess what, derivatives commute and can be taken in any order.
However, from a cursory inspection of A.18 it appears constants are
thrown into the middle of everything, so it's not eactly a trivial
result\ldots{} The first term is the problem child--every subsequent
term has the \(\Omega\) out front where the derivative can easily access
it no problem.

But the first term\ldots{}

\[ \mathcal{L}_{\pmb X} \Omega_{bcdef...} = X^o \nabla_o \Omega_{bcdef...} + \text{nice terms} \]

Now if we add the one-form back in,

\[ \partial_a \mathcal{L}_{\pmb X} \Omega_{bcdef...} = \partial_a X^o \nabla_o \Omega_{bcdef...} + \text{nice terms} \]

Now for the other relation, we actually get two terms out since if the
Lie Derivative acts on the entire ``one form'' at once, it produces a
term for every single index, including the index on the exterior
derivative.

\[  \mathcal{L}_{\pmb X} \partial_a \Omega_{bcdef...} = X^o \nabla_o \partial_a \Omega_{bcdef...} + \partial_o \Omega_{bcdef...}\nabla_a X^o + \text{nice terms} \]

Because we didn't have a term that changed the differential index
before. However, all the ``nice terms'' are identical and just go poof,
making the equality:

\[ \partial_a X^o \nabla_o \Omega_{bcdef...} = X^o \nabla_o \partial_a \Omega_{bcdef...} + \partial_o \Omega_{bcdef...}\nabla_a X^o \]

Which we might be able to work with, let's see.

    Expand the left side by the product rule.

\[ \Rightarrow \nabla_o \Omega_{bcdef...} \partial_a X^o + X^o \partial_a \nabla_o \Omega_{bcdef...} = X^o \nabla_o \partial_a \Omega_{bcdef...} + \partial_o \Omega_{bcdef...}\nabla_a X^o \]

Since derivatives commute we have the two terms closes to the equals
sign cancel.

\[ \Rightarrow \nabla_o \Omega_{bcdef...} \partial_a X^o  = \partial_o \Omega_{bcdef...}\nabla_a X^o \]

    Now this is quite promising. In fact, covariant and partial derivatives
are interchangeable\ldots{} IF we have a symmetric affine connection,
that is, that the Christoffels commute. Notably, this assumption has
been \emph{implicit} in all our work so far since we relied on A.11,
A.13, A.14, all of which use the covariant derivative version.

We are not sure this assumption is valid, to be sure\ldots{} HOWEVER!
The \emph{general} form of the Lie Derivative uses partial derivaties,
so if we doubt this step, we can revert to A.8 and use the partial
derivative version. Which means that, even in the GENERAL case\ldots{}

\[ \Rightarrow \partial_o \Omega_{bcdef...} \partial_a X^o  = \partial_o \Omega_{bcdef...}\partial_a X^o \]

Which IS clearly true and what we sought to prove.

    \hypertarget{problem-a4-back-to-top}{%
\section{\texorpdfstring{Problem A4 {[}Back to
\hyperref[toc]{top}{]}}{Problem A4 {[}Back to {]}}}\label{problem-a4-back-to-top}}

\[\label{PA4}\]

\emph{Let \(x^a(\lambda)\) be the integral curves of a vector field
\(x^a\), and let \(Y^a\) be a second vector field. Show that if \(Y^a\)
is \textbf{Lie dragged} along X\^{}a, \(\mathcal{L}_{\pmb X}Y^a = 0\),
then it will connect points of equal \(\lambda\) along the congruence
\(x^a(\lambda)\)}

Okay let's just try to evaluate this, as a vector field IS a vector, in
a way.

\[ \mathcal{L}_{\pmb X}Y^a \]

\[ = X^b \nabla_b Y^a - Y^b \nabla_b X^a \]

\[ = [X,Y]^a \]

So we are actually \emph{given} that this equals zero. We need to take
this zero and, from it, show that Y connects poitns of equal \(\lambda\)
along the given congruence. What we're doing is looking for a situation
where this commutator equals zero. This is obviously only true when the
fields are equal to each other, that is, X=Y. (Or one of them is zero
but that's trivial.)

    This means that Y has the same integral curves as X. Thus we have shown
what we sought.

(Mild confusion\ldots{} we'll see if this leads to any problems later.)

    \hypertarget{problem-15-back-to-top}{%
\section{\texorpdfstring{Problem 15 {[}Back to
\hyperref[toc]{top}{]}}{Problem 15 {[}Back to {]}}}\label{problem-15-back-to-top}}

\[\label{P15}\]

\emph{Following the example of \textbf{Problem 7}, show that the
4-dimensional Reimann tensor \(^{(4)}R_{abcd}\) can be written as:}

\[ ^{(4)}R_{abcd} = \gamma^p_a \gamma^q_b \gamma^r_c \gamma^s_d {}^{(4)}R_{pqrs} - 2\gamma_a^p \gamma_b^q \gamma^r_{[c} n_{d]} n^s {}^{(4)}R_{pqrs} - 2\gamma_c^p\gamma_d^q\gamma^r_{[a}n_{b]}n^s{}^{(4)}R_{pqrs} +  2\gamma_a^p \gamma^r_{[c} n_{d]} n_b n^q n^s {}^{(4)}R_{pqrs} - 2\gamma_b^p\gamma^r_{[c}n_{d]}n_an^qn^s{}^{(4)}R_{pqrs} \]

    Now as we recall \textbf{Problem 7} was annoying and involved some
``questionable'' index shuffling, but we'll see if we can do it agian.
The antisymmetric parts are a bit problematic.. let's examine what they
actually mean.

\[ \gamma^r_{[c} n_{d]} = \frac12 (\gamma^r_c n_d - \gamma^r_d n_c) \]

notably this is actually helpful if we EXPAND, then it'll get rid of all
the 2-coefficients and give us more terms, but all temrs that are more
similar to each other.

\[ ^{(4)}R_{abcd} = \gamma^p_a \gamma^q_b \gamma^r_c \gamma^s_d {}^{(4)}R_{pqrs} - \gamma_a^p \gamma_b^q \gamma^r_{c} n_{d} n^s {}^{(4)}R_{pqrs} + \gamma_a^p \gamma_b^q \gamma^r_{d} n_{c} n^s {}^{(4)}R_{pqrs} - \gamma_c^p\gamma_d^q\gamma^r_{a}n_{b}n^s{}^{(4)}R_{pqrs} + \gamma_c^p\gamma_d^q\gamma^r_{b}n_{a}n^s{}^{(4)}R_{pqrs} + \gamma_a^p \gamma^r_{c} n_{d} n_b n^q n^s {}^{(4)}R_{pqrs} - \gamma_a^p \gamma^r_{d} n_{c} n_b n^q n^s {}^{(4)}R_{pqrs} - \gamma_b^p\gamma^r_{c}n_{d}n_an^qn^s{}^{(4)}R_{pqrs} + \gamma_b^p\gamma^r_{d}n_{c}n_an^qn^s{}^{(4)}R_{pqrs} \]

    \[ \Rightarrow ^{(4)}R_{abcd} = \left(\gamma^p_a \gamma^q_b \gamma^r_c \gamma^s_d  - \gamma_a^p \gamma_b^q \gamma^r_{c} n_{d} n^s + \gamma_a^p \gamma_b^q \gamma^r_{d} n_{c} n^s - \gamma_c^p\gamma_d^q\gamma^r_{a}n_{b}n^s + \gamma_c^p\gamma_d^q\gamma^r_{b}n_{a}n^s + \gamma_a^p \gamma^r_{c} n_{d} n_b n^q n^s  - \gamma_a^p \gamma^r_{d} n_{c} n_b n^q n^s- \gamma_b^p\gamma^r_{c}n_{d}n_an^qn^s + \gamma_b^p\gamma^r_{d}n_{c}n_an^qn^s\right){}^{(4)}R_{pqrs} \]

    So basically we need to show that the things in the parentheses reduce
to nothing more than a ``change the index of R'' situation. Which pretty
clearly means it becomes \(\delta^p_a\delta^q_b\delta^r_c\delta^s_d\)

\[ ^{(4)}R_{abcd} = \left(\gamma^p_a \gamma^q_b \gamma^r_c \gamma^s_d  - \gamma_a^p \gamma_b^q \gamma^r_{c} n_{d} n^s + \gamma_a^p \gamma_b^q \gamma^r_{d} n_{c} n^s - \gamma_c^p\gamma_d^q\gamma^r_{a}n_{b}n^s + \gamma_c^p\gamma_d^q\gamma^r_{b}n_{a}n^s + \gamma_a^p \gamma^r_{c} n_{d} n_b n^q n^s  - \gamma_a^p \gamma^r_{d} n_{c} n_b n^q n^s- \gamma_b^p\gamma^r_{c}n_{d}n_an^qn^s + \gamma_b^p\gamma^r_{d}n_{c}n_an^qn^s\right){}^{(4)}R_{pqrs} \]

    Much like \textbf{Problem 7} even after much monkeying, the relations
just don't seem to be equal.

    \hypertarget{problem-16-back-to-top}{%
\section{\texorpdfstring{Problem 16 {[}Back to
\hyperref[toc]{top}{]}}{Problem 16 {[}Back to {]}}}\label{problem-16-back-to-top}}

\[\label{P16}\]

\emph{Show that}

\[ \nabla_a V^a = \frac1\alpha D_a (\alpha V^a) \]

\emph{For any spatial vector V\^{}a. Hint: One possible derivation uses
2.51 and 2.62; a more elegant aproach starts with the identity A.44}

2.51 is from \textbf{Problem 13} and is \(a_a = D_a ln \alpha\)

2.62 is the long

\[ D_a V^b = \gamma^p_a\gamma^b_q\nabla_pV^q = \gamma^p_a(g^b_q+n_qn^b) \nabla_pV^q = \gamma^p_a \nabla_p V^b - \gamma^p_a n^b V^q \nabla_p n_q = \gamma_a^p \nabla_p V^b - n^bV^e\gamma^p_a\gamma^q_e\nabla_pn_q = \gamma^p_a\nabla_pV^b + n^bV^eK_{ae}\]

Meanwhile A.44 gives us

\[\nabla_a X^a = \frac{1}{\sqrt{|g|}} \partial_a (\sqrt{|g|}X^a)\]

    \hypertarget{addendum-output-this-notebook-to-latex-formatted-pdf-file-back-to-top}{%
\section{\texorpdfstring{Addendum: Output this notebook to
\(\LaTeX\)-formatted PDF file {[}Back to
\hyperref[toc]{top}{]}}{Addendum: Output this notebook to \textbackslash LaTeX-formatted PDF file {[}Back to {]}}}\label{addendum-output-this-notebook-to-latex-formatted-pdf-file-back-to-top}}

\[\label{latex_pdf_output}\]

The following code cell converts this Jupyter notebook into a proper,
clickable \(\LaTeX\)-formatted PDF file. After the cell is successfully
run, the generated PDF may be found in the root NRPy+ tutorial
directory, with filename \url{NR-02.pdf} (Note that clicking on this
link may not work; you may need to open the PDF file through another
means.)

\textbf{Important Note}: Make sure that the file name is right in all
six locations, two here in the Markdown, four in the code below.

\begin{itemize}
\tightlist
\item
  NR-02.pdf
\item
  NR-02.ipynb
\item
  NR-02.tex
\end{itemize}

    \begin{tcolorbox}[breakable, size=fbox, boxrule=1pt, pad at break*=1mm,colback=cellbackground, colframe=cellborder]
\prompt{In}{incolor}{1}{\boxspacing}
\begin{Verbatim}[commandchars=\\\{\}]
\PY{k+kn}{import} \PY{n+nn}{cmdline\PYZus{}helper} \PY{k}{as} \PY{n+nn}{cmd}    \PY{c+c1}{\PYZsh{} NRPy+: Multi\PYZhy{}platform Python command\PYZhy{}line interface}
\PY{n}{cmd}\PY{o}{.}\PY{n}{output\PYZus{}Jupyter\PYZus{}notebook\PYZus{}to\PYZus{}LaTeXed\PYZus{}PDF}\PY{p}{(}\PY{l+s+s2}{\PYZdq{}}\PY{l+s+s2}{NR\PYZhy{}02}\PY{l+s+s2}{\PYZdq{}}\PY{p}{)}
\end{Verbatim}
\end{tcolorbox}

    \begin{Verbatim}[commandchars=\\\{\}]
Created NR-02.tex, and compiled LaTeX file to PDF file NR-02.pdf
    \end{Verbatim}

    \begin{tcolorbox}[breakable, size=fbox, boxrule=1pt, pad at break*=1mm,colback=cellbackground, colframe=cellborder]
\prompt{In}{incolor}{ }{\boxspacing}
\begin{Verbatim}[commandchars=\\\{\}]

\end{Verbatim}
\end{tcolorbox}

    \begin{tcolorbox}[breakable, size=fbox, boxrule=1pt, pad at break*=1mm,colback=cellbackground, colframe=cellborder]
\prompt{In}{incolor}{ }{\boxspacing}
\begin{Verbatim}[commandchars=\\\{\}]

\end{Verbatim}
\end{tcolorbox}


    % Add a bibliography block to the postdoc
    
    
    
\end{document}
