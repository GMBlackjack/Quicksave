% Based on http://nbviewer.jupyter.org/github/ipython/nbconvert-examples/blob/master/citations/Tutorial.ipynb , authored by Brian E. Granger
    % Declare the document class
    \documentclass[landscape,letterpaper,10pt,english]{article}


    \usepackage[breakable]{tcolorbox}
    \usepackage{parskip} % Stop auto-indenting (to mimic markdown behaviour)
    

    % Basic figure setup, for now with no caption control since it's done
    % automatically by Pandoc (which extracts ![](path) syntax from Markdown).
    \usepackage{graphicx}
    % Maintain compatibility with old templates. Remove in nbconvert 6.0
    \let\Oldincludegraphics\includegraphics
    % Ensure that by default, figures have no caption (until we provide a
    % proper Figure object with a Caption API and a way to capture that
    % in the conversion process - todo).
    \usepackage{caption}
    \DeclareCaptionFormat{nocaption}{}
    \captionsetup{format=nocaption,aboveskip=0pt,belowskip=0pt}

    \usepackage{float}
    \floatplacement{figure}{H} % forces figures to be placed at the correct location
    \usepackage{xcolor} % Allow colors to be defined
    \usepackage{enumerate} % Needed for markdown enumerations to work
    \usepackage{geometry} % Used to adjust the document margins
    \usepackage{amsmath} % Equations
    \usepackage{amssymb} % Equations
    \usepackage{textcomp} % defines textquotesingle
    % Hack from http://tex.stackexchange.com/a/47451/13684:
    \AtBeginDocument{%
        \def\PYZsq{\textquotesingle}% Upright quotes in Pygmentized code
    }
    \usepackage{upquote} % Upright quotes for verbatim code
    \usepackage{eurosym} % defines \euro

    \usepackage{iftex}
    \ifPDFTeX
        \usepackage[T1]{fontenc}
        \IfFileExists{alphabeta.sty}{
              \usepackage{alphabeta}
          }{
              \usepackage[mathletters]{ucs}
              \usepackage[utf8x]{inputenc}
          }
    \else
        \usepackage{fontspec}
        \usepackage{unicode-math}
    \fi

    \usepackage{fancyvrb} % verbatim replacement that allows latex
    \usepackage{grffile} % extends the file name processing of package graphics
                         % to support a larger range
    \makeatletter % fix for old versions of grffile with XeLaTeX
    \@ifpackagelater{grffile}{2019/11/01}
    {
      % Do nothing on new versions
    }
    {
      \def\Gread@@xetex#1{%
        \IfFileExists{"\Gin@base".bb}%
        {\Gread@eps{\Gin@base.bb}}%
        {\Gread@@xetex@aux#1}%
      }
    }
    \makeatother
    \usepackage[Export]{adjustbox} % Used to constrain images to a maximum size
    \adjustboxset{max size={0.9\linewidth}{0.9\paperheight}}

    % The hyperref package gives us a pdf with properly built
    % internal navigation ('pdf bookmarks' for the table of contents,
    % internal cross-reference links, web links for URLs, etc.)
    \usepackage{hyperref}
    % The default LaTeX title has an obnoxious amount of whitespace. By default,
    % titling removes some of it. It also provides customization options.
    \usepackage{titling}
    \usepackage{longtable} % longtable support required by pandoc >1.10
    \usepackage{booktabs}  % table support for pandoc > 1.12.2
    \usepackage{array}     % table support for pandoc >= 2.11.3
    \usepackage{calc}      % table minipage width calculation for pandoc >= 2.11.1
    \usepackage[inline]{enumitem} % IRkernel/repr support (it uses the enumerate* environment)
    \usepackage[normalem]{ulem} % ulem is needed to support strikethroughs (\sout)
                                % normalem makes italics be italics, not underlines
    \usepackage{mathrsfs}
    

    
    % Colors for the hyperref package
    \definecolor{urlcolor}{rgb}{0,.145,.698}
    \definecolor{linkcolor}{rgb}{.71,0.21,0.01}
    \definecolor{citecolor}{rgb}{.12,.54,.11}

    % ANSI colors
    \definecolor{ansi-black}{HTML}{3E424D}
    \definecolor{ansi-black-intense}{HTML}{282C36}
    \definecolor{ansi-red}{HTML}{E75C58}
    \definecolor{ansi-red-intense}{HTML}{B22B31}
    \definecolor{ansi-green}{HTML}{00A250}
    \definecolor{ansi-green-intense}{HTML}{007427}
    \definecolor{ansi-yellow}{HTML}{DDB62B}
    \definecolor{ansi-yellow-intense}{HTML}{B27D12}
    \definecolor{ansi-blue}{HTML}{208FFB}
    \definecolor{ansi-blue-intense}{HTML}{0065CA}
    \definecolor{ansi-magenta}{HTML}{D160C4}
    \definecolor{ansi-magenta-intense}{HTML}{A03196}
    \definecolor{ansi-cyan}{HTML}{60C6C8}
    \definecolor{ansi-cyan-intense}{HTML}{258F8F}
    \definecolor{ansi-white}{HTML}{C5C1B4}
    \definecolor{ansi-white-intense}{HTML}{A1A6B2}
    \definecolor{ansi-default-inverse-fg}{HTML}{FFFFFF}
    \definecolor{ansi-default-inverse-bg}{HTML}{000000}

    % common color for the border for error outputs.
    \definecolor{outerrorbackground}{HTML}{FFDFDF}

    % commands and environments needed by pandoc snippets
    % extracted from the output of `pandoc -s`
    \providecommand{\tightlist}{%
      \setlength{\itemsep}{0pt}\setlength{\parskip}{0pt}}
    \DefineVerbatimEnvironment{Highlighting}{Verbatim}{commandchars=\\\{\}}
    % Add ',fontsize=\small' for more characters per line
    \newenvironment{Shaded}{}{}
    \newcommand{\KeywordTok}[1]{\textcolor[rgb]{0.00,0.44,0.13}{\textbf{{#1}}}}
    \newcommand{\DataTypeTok}[1]{\textcolor[rgb]{0.56,0.13,0.00}{{#1}}}
    \newcommand{\DecValTok}[1]{\textcolor[rgb]{0.25,0.63,0.44}{{#1}}}
    \newcommand{\BaseNTok}[1]{\textcolor[rgb]{0.25,0.63,0.44}{{#1}}}
    \newcommand{\FloatTok}[1]{\textcolor[rgb]{0.25,0.63,0.44}{{#1}}}
    \newcommand{\CharTok}[1]{\textcolor[rgb]{0.25,0.44,0.63}{{#1}}}
    \newcommand{\StringTok}[1]{\textcolor[rgb]{0.25,0.44,0.63}{{#1}}}
    \newcommand{\CommentTok}[1]{\textcolor[rgb]{0.38,0.63,0.69}{\textit{{#1}}}}
    \newcommand{\OtherTok}[1]{\textcolor[rgb]{0.00,0.44,0.13}{{#1}}}
    \newcommand{\AlertTok}[1]{\textcolor[rgb]{1.00,0.00,0.00}{\textbf{{#1}}}}
    \newcommand{\FunctionTok}[1]{\textcolor[rgb]{0.02,0.16,0.49}{{#1}}}
    \newcommand{\RegionMarkerTok}[1]{{#1}}
    \newcommand{\ErrorTok}[1]{\textcolor[rgb]{1.00,0.00,0.00}{\textbf{{#1}}}}
    \newcommand{\NormalTok}[1]{{#1}}

    % Additional commands for more recent versions of Pandoc
    \newcommand{\ConstantTok}[1]{\textcolor[rgb]{0.53,0.00,0.00}{{#1}}}
    \newcommand{\SpecialCharTok}[1]{\textcolor[rgb]{0.25,0.44,0.63}{{#1}}}
    \newcommand{\VerbatimStringTok}[1]{\textcolor[rgb]{0.25,0.44,0.63}{{#1}}}
    \newcommand{\SpecialStringTok}[1]{\textcolor[rgb]{0.73,0.40,0.53}{{#1}}}
    \newcommand{\ImportTok}[1]{{#1}}
    \newcommand{\DocumentationTok}[1]{\textcolor[rgb]{0.73,0.13,0.13}{\textit{{#1}}}}
    \newcommand{\AnnotationTok}[1]{\textcolor[rgb]{0.38,0.63,0.69}{\textbf{\textit{{#1}}}}}
    \newcommand{\CommentVarTok}[1]{\textcolor[rgb]{0.38,0.63,0.69}{\textbf{\textit{{#1}}}}}
    \newcommand{\VariableTok}[1]{\textcolor[rgb]{0.10,0.09,0.49}{{#1}}}
    \newcommand{\ControlFlowTok}[1]{\textcolor[rgb]{0.00,0.44,0.13}{\textbf{{#1}}}}
    \newcommand{\OperatorTok}[1]{\textcolor[rgb]{0.40,0.40,0.40}{{#1}}}
    \newcommand{\BuiltInTok}[1]{{#1}}
    \newcommand{\ExtensionTok}[1]{{#1}}
    \newcommand{\PreprocessorTok}[1]{\textcolor[rgb]{0.74,0.48,0.00}{{#1}}}
    \newcommand{\AttributeTok}[1]{\textcolor[rgb]{0.49,0.56,0.16}{{#1}}}
    \newcommand{\InformationTok}[1]{\textcolor[rgb]{0.38,0.63,0.69}{\textbf{\textit{{#1}}}}}
    \newcommand{\WarningTok}[1]{\textcolor[rgb]{0.38,0.63,0.69}{\textbf{\textit{{#1}}}}}


    % Define a nice break command that doesn't care if a line doesn't already
    % exist.
    \def\br{\hspace*{\fill} \\* }
    % Math Jax compatibility definitions
    \def\gt{>}
    \def\lt{<}
    \let\Oldtex\TeX
    \let\Oldlatex\LaTeX
    \renewcommand{\TeX}{\textrm{\Oldtex}}
    \renewcommand{\LaTeX}{\textrm{\Oldlatex}}
    % Document parameters
    % Document title
    \title{CO-13}
    
    
    
    
    
% Pygments definitions
\makeatletter
\def\PY@reset{\let\PY@it=\relax \let\PY@bf=\relax%
    \let\PY@ul=\relax \let\PY@tc=\relax%
    \let\PY@bc=\relax \let\PY@ff=\relax}
\def\PY@tok#1{\csname PY@tok@#1\endcsname}
\def\PY@toks#1+{\ifx\relax#1\empty\else%
    \PY@tok{#1}\expandafter\PY@toks\fi}
\def\PY@do#1{\PY@bc{\PY@tc{\PY@ul{%
    \PY@it{\PY@bf{\PY@ff{#1}}}}}}}
\def\PY#1#2{\PY@reset\PY@toks#1+\relax+\PY@do{#2}}

\@namedef{PY@tok@w}{\def\PY@tc##1{\textcolor[rgb]{0.73,0.73,0.73}{##1}}}
\@namedef{PY@tok@c}{\let\PY@it=\textit\def\PY@tc##1{\textcolor[rgb]{0.24,0.48,0.48}{##1}}}
\@namedef{PY@tok@cp}{\def\PY@tc##1{\textcolor[rgb]{0.61,0.40,0.00}{##1}}}
\@namedef{PY@tok@k}{\let\PY@bf=\textbf\def\PY@tc##1{\textcolor[rgb]{0.00,0.50,0.00}{##1}}}
\@namedef{PY@tok@kp}{\def\PY@tc##1{\textcolor[rgb]{0.00,0.50,0.00}{##1}}}
\@namedef{PY@tok@kt}{\def\PY@tc##1{\textcolor[rgb]{0.69,0.00,0.25}{##1}}}
\@namedef{PY@tok@o}{\def\PY@tc##1{\textcolor[rgb]{0.40,0.40,0.40}{##1}}}
\@namedef{PY@tok@ow}{\let\PY@bf=\textbf\def\PY@tc##1{\textcolor[rgb]{0.67,0.13,1.00}{##1}}}
\@namedef{PY@tok@nb}{\def\PY@tc##1{\textcolor[rgb]{0.00,0.50,0.00}{##1}}}
\@namedef{PY@tok@nf}{\def\PY@tc##1{\textcolor[rgb]{0.00,0.00,1.00}{##1}}}
\@namedef{PY@tok@nc}{\let\PY@bf=\textbf\def\PY@tc##1{\textcolor[rgb]{0.00,0.00,1.00}{##1}}}
\@namedef{PY@tok@nn}{\let\PY@bf=\textbf\def\PY@tc##1{\textcolor[rgb]{0.00,0.00,1.00}{##1}}}
\@namedef{PY@tok@ne}{\let\PY@bf=\textbf\def\PY@tc##1{\textcolor[rgb]{0.80,0.25,0.22}{##1}}}
\@namedef{PY@tok@nv}{\def\PY@tc##1{\textcolor[rgb]{0.10,0.09,0.49}{##1}}}
\@namedef{PY@tok@no}{\def\PY@tc##1{\textcolor[rgb]{0.53,0.00,0.00}{##1}}}
\@namedef{PY@tok@nl}{\def\PY@tc##1{\textcolor[rgb]{0.46,0.46,0.00}{##1}}}
\@namedef{PY@tok@ni}{\let\PY@bf=\textbf\def\PY@tc##1{\textcolor[rgb]{0.44,0.44,0.44}{##1}}}
\@namedef{PY@tok@na}{\def\PY@tc##1{\textcolor[rgb]{0.41,0.47,0.13}{##1}}}
\@namedef{PY@tok@nt}{\let\PY@bf=\textbf\def\PY@tc##1{\textcolor[rgb]{0.00,0.50,0.00}{##1}}}
\@namedef{PY@tok@nd}{\def\PY@tc##1{\textcolor[rgb]{0.67,0.13,1.00}{##1}}}
\@namedef{PY@tok@s}{\def\PY@tc##1{\textcolor[rgb]{0.73,0.13,0.13}{##1}}}
\@namedef{PY@tok@sd}{\let\PY@it=\textit\def\PY@tc##1{\textcolor[rgb]{0.73,0.13,0.13}{##1}}}
\@namedef{PY@tok@si}{\let\PY@bf=\textbf\def\PY@tc##1{\textcolor[rgb]{0.64,0.35,0.47}{##1}}}
\@namedef{PY@tok@se}{\let\PY@bf=\textbf\def\PY@tc##1{\textcolor[rgb]{0.67,0.36,0.12}{##1}}}
\@namedef{PY@tok@sr}{\def\PY@tc##1{\textcolor[rgb]{0.64,0.35,0.47}{##1}}}
\@namedef{PY@tok@ss}{\def\PY@tc##1{\textcolor[rgb]{0.10,0.09,0.49}{##1}}}
\@namedef{PY@tok@sx}{\def\PY@tc##1{\textcolor[rgb]{0.00,0.50,0.00}{##1}}}
\@namedef{PY@tok@m}{\def\PY@tc##1{\textcolor[rgb]{0.40,0.40,0.40}{##1}}}
\@namedef{PY@tok@gh}{\let\PY@bf=\textbf\def\PY@tc##1{\textcolor[rgb]{0.00,0.00,0.50}{##1}}}
\@namedef{PY@tok@gu}{\let\PY@bf=\textbf\def\PY@tc##1{\textcolor[rgb]{0.50,0.00,0.50}{##1}}}
\@namedef{PY@tok@gd}{\def\PY@tc##1{\textcolor[rgb]{0.63,0.00,0.00}{##1}}}
\@namedef{PY@tok@gi}{\def\PY@tc##1{\textcolor[rgb]{0.00,0.52,0.00}{##1}}}
\@namedef{PY@tok@gr}{\def\PY@tc##1{\textcolor[rgb]{0.89,0.00,0.00}{##1}}}
\@namedef{PY@tok@ge}{\let\PY@it=\textit}
\@namedef{PY@tok@gs}{\let\PY@bf=\textbf}
\@namedef{PY@tok@gp}{\let\PY@bf=\textbf\def\PY@tc##1{\textcolor[rgb]{0.00,0.00,0.50}{##1}}}
\@namedef{PY@tok@go}{\def\PY@tc##1{\textcolor[rgb]{0.44,0.44,0.44}{##1}}}
\@namedef{PY@tok@gt}{\def\PY@tc##1{\textcolor[rgb]{0.00,0.27,0.87}{##1}}}
\@namedef{PY@tok@err}{\def\PY@bc##1{{\setlength{\fboxsep}{\string -\fboxrule}\fcolorbox[rgb]{1.00,0.00,0.00}{1,1,1}{\strut ##1}}}}
\@namedef{PY@tok@kc}{\let\PY@bf=\textbf\def\PY@tc##1{\textcolor[rgb]{0.00,0.50,0.00}{##1}}}
\@namedef{PY@tok@kd}{\let\PY@bf=\textbf\def\PY@tc##1{\textcolor[rgb]{0.00,0.50,0.00}{##1}}}
\@namedef{PY@tok@kn}{\let\PY@bf=\textbf\def\PY@tc##1{\textcolor[rgb]{0.00,0.50,0.00}{##1}}}
\@namedef{PY@tok@kr}{\let\PY@bf=\textbf\def\PY@tc##1{\textcolor[rgb]{0.00,0.50,0.00}{##1}}}
\@namedef{PY@tok@bp}{\def\PY@tc##1{\textcolor[rgb]{0.00,0.50,0.00}{##1}}}
\@namedef{PY@tok@fm}{\def\PY@tc##1{\textcolor[rgb]{0.00,0.00,1.00}{##1}}}
\@namedef{PY@tok@vc}{\def\PY@tc##1{\textcolor[rgb]{0.10,0.09,0.49}{##1}}}
\@namedef{PY@tok@vg}{\def\PY@tc##1{\textcolor[rgb]{0.10,0.09,0.49}{##1}}}
\@namedef{PY@tok@vi}{\def\PY@tc##1{\textcolor[rgb]{0.10,0.09,0.49}{##1}}}
\@namedef{PY@tok@vm}{\def\PY@tc##1{\textcolor[rgb]{0.10,0.09,0.49}{##1}}}
\@namedef{PY@tok@sa}{\def\PY@tc##1{\textcolor[rgb]{0.73,0.13,0.13}{##1}}}
\@namedef{PY@tok@sb}{\def\PY@tc##1{\textcolor[rgb]{0.73,0.13,0.13}{##1}}}
\@namedef{PY@tok@sc}{\def\PY@tc##1{\textcolor[rgb]{0.73,0.13,0.13}{##1}}}
\@namedef{PY@tok@dl}{\def\PY@tc##1{\textcolor[rgb]{0.73,0.13,0.13}{##1}}}
\@namedef{PY@tok@s2}{\def\PY@tc##1{\textcolor[rgb]{0.73,0.13,0.13}{##1}}}
\@namedef{PY@tok@sh}{\def\PY@tc##1{\textcolor[rgb]{0.73,0.13,0.13}{##1}}}
\@namedef{PY@tok@s1}{\def\PY@tc##1{\textcolor[rgb]{0.73,0.13,0.13}{##1}}}
\@namedef{PY@tok@mb}{\def\PY@tc##1{\textcolor[rgb]{0.40,0.40,0.40}{##1}}}
\@namedef{PY@tok@mf}{\def\PY@tc##1{\textcolor[rgb]{0.40,0.40,0.40}{##1}}}
\@namedef{PY@tok@mh}{\def\PY@tc##1{\textcolor[rgb]{0.40,0.40,0.40}{##1}}}
\@namedef{PY@tok@mi}{\def\PY@tc##1{\textcolor[rgb]{0.40,0.40,0.40}{##1}}}
\@namedef{PY@tok@il}{\def\PY@tc##1{\textcolor[rgb]{0.40,0.40,0.40}{##1}}}
\@namedef{PY@tok@mo}{\def\PY@tc##1{\textcolor[rgb]{0.40,0.40,0.40}{##1}}}
\@namedef{PY@tok@ch}{\let\PY@it=\textit\def\PY@tc##1{\textcolor[rgb]{0.24,0.48,0.48}{##1}}}
\@namedef{PY@tok@cm}{\let\PY@it=\textit\def\PY@tc##1{\textcolor[rgb]{0.24,0.48,0.48}{##1}}}
\@namedef{PY@tok@cpf}{\let\PY@it=\textit\def\PY@tc##1{\textcolor[rgb]{0.24,0.48,0.48}{##1}}}
\@namedef{PY@tok@c1}{\let\PY@it=\textit\def\PY@tc##1{\textcolor[rgb]{0.24,0.48,0.48}{##1}}}
\@namedef{PY@tok@cs}{\let\PY@it=\textit\def\PY@tc##1{\textcolor[rgb]{0.24,0.48,0.48}{##1}}}

\def\PYZbs{\char`\\}
\def\PYZus{\char`\_}
\def\PYZob{\char`\{}
\def\PYZcb{\char`\}}
\def\PYZca{\char`\^}
\def\PYZam{\char`\&}
\def\PYZlt{\char`\<}
\def\PYZgt{\char`\>}
\def\PYZsh{\char`\#}
\def\PYZpc{\char`\%}
\def\PYZdl{\char`\$}
\def\PYZhy{\char`\-}
\def\PYZsq{\char`\'}
\def\PYZdq{\char`\"}
\def\PYZti{\char`\~}
% for compatibility with earlier versions
\def\PYZat{@}
\def\PYZlb{[}
\def\PYZrb{]}
\makeatother


    % For linebreaks inside Verbatim environment from package fancyvrb.
    \makeatletter
        \newbox\Wrappedcontinuationbox
        \newbox\Wrappedvisiblespacebox
        \newcommand*\Wrappedvisiblespace {\textcolor{red}{\textvisiblespace}}
        \newcommand*\Wrappedcontinuationsymbol {\textcolor{red}{\llap{\tiny$\m@th\hookrightarrow$}}}
        \newcommand*\Wrappedcontinuationindent {3ex }
        \newcommand*\Wrappedafterbreak {\kern\Wrappedcontinuationindent\copy\Wrappedcontinuationbox}
        % Take advantage of the already applied Pygments mark-up to insert
        % potential linebreaks for TeX processing.
        %        {, <, #, %, $, ' and ": go to next line.
        %        _, }, ^, &, >, - and ~: stay at end of broken line.
        % Use of \textquotesingle for straight quote.
        \newcommand*\Wrappedbreaksatspecials {%
            \def\PYGZus{\discretionary{\char`\_}{\Wrappedafterbreak}{\char`\_}}%
            \def\PYGZob{\discretionary{}{\Wrappedafterbreak\char`\{}{\char`\{}}%
            \def\PYGZcb{\discretionary{\char`\}}{\Wrappedafterbreak}{\char`\}}}%
            \def\PYGZca{\discretionary{\char`\^}{\Wrappedafterbreak}{\char`\^}}%
            \def\PYGZam{\discretionary{\char`\&}{\Wrappedafterbreak}{\char`\&}}%
            \def\PYGZlt{\discretionary{}{\Wrappedafterbreak\char`\<}{\char`\<}}%
            \def\PYGZgt{\discretionary{\char`\>}{\Wrappedafterbreak}{\char`\>}}%
            \def\PYGZsh{\discretionary{}{\Wrappedafterbreak\char`\#}{\char`\#}}%
            \def\PYGZpc{\discretionary{}{\Wrappedafterbreak\char`\%}{\char`\%}}%
            \def\PYGZdl{\discretionary{}{\Wrappedafterbreak\char`\$}{\char`\$}}%
            \def\PYGZhy{\discretionary{\char`\-}{\Wrappedafterbreak}{\char`\-}}%
            \def\PYGZsq{\discretionary{}{\Wrappedafterbreak\textquotesingle}{\textquotesingle}}%
            \def\PYGZdq{\discretionary{}{\Wrappedafterbreak\char`\"}{\char`\"}}%
            \def\PYGZti{\discretionary{\char`\~}{\Wrappedafterbreak}{\char`\~}}%
        }
        % Some characters . , ; ? ! / are not pygmentized.
        % This macro makes them "active" and they will insert potential linebreaks
        \newcommand*\Wrappedbreaksatpunct {%
            \lccode`\~`\.\lowercase{\def~}{\discretionary{\hbox{\char`\.}}{\Wrappedafterbreak}{\hbox{\char`\.}}}%
            \lccode`\~`\,\lowercase{\def~}{\discretionary{\hbox{\char`\,}}{\Wrappedafterbreak}{\hbox{\char`\,}}}%
            \lccode`\~`\;\lowercase{\def~}{\discretionary{\hbox{\char`\;}}{\Wrappedafterbreak}{\hbox{\char`\;}}}%
            \lccode`\~`\:\lowercase{\def~}{\discretionary{\hbox{\char`\:}}{\Wrappedafterbreak}{\hbox{\char`\:}}}%
            \lccode`\~`\?\lowercase{\def~}{\discretionary{\hbox{\char`\?}}{\Wrappedafterbreak}{\hbox{\char`\?}}}%
            \lccode`\~`\!\lowercase{\def~}{\discretionary{\hbox{\char`\!}}{\Wrappedafterbreak}{\hbox{\char`\!}}}%
            \lccode`\~`\/\lowercase{\def~}{\discretionary{\hbox{\char`\/}}{\Wrappedafterbreak}{\hbox{\char`\/}}}%
            \catcode`\.\active
            \catcode`\,\active
            \catcode`\;\active
            \catcode`\:\active
            \catcode`\?\active
            \catcode`\!\active
            \catcode`\/\active
            \lccode`\~`\~
        }
    \makeatother

    \let\OriginalVerbatim=\Verbatim
    \makeatletter
    \renewcommand{\Verbatim}[1][1]{%
        %\parskip\z@skip
        \sbox\Wrappedcontinuationbox {\Wrappedcontinuationsymbol}%
        \sbox\Wrappedvisiblespacebox {\FV@SetupFont\Wrappedvisiblespace}%
        \def\FancyVerbFormatLine ##1{\hsize\linewidth
            \vtop{\raggedright\hyphenpenalty\z@\exhyphenpenalty\z@
                \doublehyphendemerits\z@\finalhyphendemerits\z@
                \strut ##1\strut}%
        }%
        % If the linebreak is at a space, the latter will be displayed as visible
        % space at end of first line, and a continuation symbol starts next line.
        % Stretch/shrink are however usually zero for typewriter font.
        \def\FV@Space {%
            \nobreak\hskip\z@ plus\fontdimen3\font minus\fontdimen4\font
            \discretionary{\copy\Wrappedvisiblespacebox}{\Wrappedafterbreak}
            {\kern\fontdimen2\font}%
        }%

        % Allow breaks at special characters using \PYG... macros.
        \Wrappedbreaksatspecials
        % Breaks at punctuation characters . , ; ? ! and / need catcode=\active
        \OriginalVerbatim[#1,codes*=\Wrappedbreaksatpunct]%
    }
    \makeatother

    % Exact colors from NB
    \definecolor{incolor}{HTML}{303F9F}
    \definecolor{outcolor}{HTML}{D84315}
    \definecolor{cellborder}{HTML}{CFCFCF}
    \definecolor{cellbackground}{HTML}{F7F7F7}

    % prompt
    \makeatletter
    \newcommand{\boxspacing}{\kern\kvtcb@left@rule\kern\kvtcb@boxsep}
    \makeatother
    \newcommand{\prompt}[4]{
        {\ttfamily\llap{{\color{#2}[#3]:\hspace{3pt}#4}}\vspace{-\baselineskip}}
    }
    

    
% Start the section counter at -1, so the Table of Contents is Section 0
   \setcounter{section}{-2}
% Prevent overflowing lines due to hard-to-break entities
    \sloppy
    % Setup hyperref package
    \hypersetup{
      breaklinks=true,  % so long urls are correctly broken across lines
      colorlinks=true,
      urlcolor=urlcolor,
      linkcolor=linkcolor,
      citecolor=citecolor,
      }

    % Slightly bigger margins than the latex defaults
    \geometry{verbose,tmargin=0.5in,bmargin=0.5in,lmargin=0.5in,rmargin=0.5in}


\begin{document}
    
    \maketitle
    
    

    
    \hypertarget{compact-objects-problems-chapter-13-compact-x-ray-sources}{%
\section{Compact Objects Problems Chapter 13: Compact X-Ray
Sources}\label{compact-objects-problems-chapter-13-compact-x-ray-sources}}

\hypertarget{authors-gabriel-m-steward}{%
\subsection{Authors: Gabriel M
Steward}\label{authors-gabriel-m-steward}}

    https://github.com/zachetienne/nrpytutorial/blob/master/Tutorial-Template\_Style\_Guide.ipynb

Link to the Style Guide. Not internal in case something breaks.

    \hypertarget{nrpy-source-code-for-this-module}{%
\subsubsection{\texorpdfstring{ NRPy+ Source Code for this
module:}{ NRPy+ Source Code for this module:}}\label{nrpy-source-code-for-this-module}}

None! \ldots well except for the pdf thing at the bottom.

\hypertarget{introduction}{%
\subsection{Introduction:}\label{introduction}}

As if black holes weren't cool enough, now we're going to see how they
emit when interacting with matter.

\hypertarget{other-optional}{%
\subsection{\texorpdfstring{ Other
(Optional):}{ Other (Optional):}}\label{other-optional}}

Placeholder

\hypertarget{note-on-notation}{%
\subsubsection{Note on Notation:}\label{note-on-notation}}

Any new notation will be brought up in the notebook when it becomes
relevant.

\hypertarget{citations}{%
\subsubsection{Citations:}\label{citations}}

{[}1{]} linky (descrip)

    \hypertarget{table-of-contents}{%
\section{Table of Contents}\label{table-of-contents}}

\[\label{toc}\]

\hyperref[p1]{Problem 1} (Planck Density Fun)

\hyperref[latex_pdf_output]{PDF} (turn this into a PDF)

    \hypertarget{problem-1-back-to-top}{%
\section{\texorpdfstring{Problem 1 {[}Back to
\hyperref[toc]{top}{]}}{Problem 1 {[}Back to {]}}}\label{problem-1-back-to-top}}

\[\label{P1}\]

\emph{Construct a density by dimensional analysis out of c, G, and
\(\hbar\). Evaluate numerically this ``Planck density'' at which quantum
gravitational effects would become important.}

c = 3e8 m/s

G = 6.67e-11 \(m^3\) / kg \(s^2\)

\(\hbar\) = 1.055e-34 kg \(m^2\) / s

    The density composed of this would be \(kg / m^3\). Now the method which
we used to find this is something very odd but kind of cool. When
combining various units together in dimensional analysis it is
synonymous with considering how vectors span a space, with each exponent
on a unit being a coefficient. Thus in m s kg units, we have the
dimensionality of c as (1,-1,0), G as (3,-2,-1), and \(\hbar\) as
(2,-1,1). We sought the linear combination that added to (-3,0,1), aka
density. And by solving the resulting system of equations we found it at
5c - 2G - \(\hbar\).

Aka \(\frac{c^5}{G^2\hbar}\).

Which produces roughly 5e96 \(kg/m^3\). Which converts to roughtly 5e93
\(g/cm^3\) which does in fact round to 1e94 as is suggested by the given
answer.

    \hypertarget{problem-2-back-to-top}{%
\section{\texorpdfstring{Problem 2 {[}Back to
\hyperref[toc]{top}{]}}{Problem 2 {[}Back to {]}}}\label{problem-2-back-to-top}}

\[\label{P2}\]

\emph{For an alternative derivation of hte redshift formula, uset he
fact that E is constant along the photon's path to show that}

\[ \frac{\nu_{em}}{\nu_{rec}} = \left( 1-\frac{2M}{r_{em}} \right)^{-1/2} \]

\emph{For a static emitter at r=\(r_{em}\) and a receiver at r
-\textgreater{} \(\infty\). Explain why the event horizon for a
Schwarzchild black hole is sometimes called the ``surface of infinite
redshift.''}

Last question first. See that when r=2M we end up with a 0 solution in
the radical. Which is negative. Which is a division by zero. Oh noes,
that's not good.

In more mathematical terms the limit as r approachs 2M appraoches
infinity for the equation we're trying to solve for, thus infinite
redshift. (realistically photons emitted here can't escape so the
infinite redshift never occurs.)

Now let's actually derive this expression.

    Now E is supposedly the ``energy'' but energy is poorly defined in
relativity, really it's the negative time-component of the four
momentum, as seen in 12.4.8.

\[ -p_t = \left( 1-\frac{2M}{r} \right) \dot t = E \]

Now we know r is changing, this is true, so in order for the overall
situation to be constant either M or \(\dot t\) has to be changing. M is
not, so the rate at which we experience time must be adjusting. In fact,
we can for this rate in both ends: emission is at r, and then we set r
to infinity for the reception. This gives us:

\[ \left( 1-\frac{2M}{r} \right) \dot t  = \dot t' = p'^t\]

    The momentum of a photon is \(h\nu\). Which would seem to imply:

\[ \frac{\nu_{em}}{\nu_{rec}} = \left( 1-\frac{2M}{r_{em}} \right)^{-1} \]

Which has the correct \emph{qualitative} behavior, but we're missing the
root. So where does this come from? What we forgot to do was the
corodinate transform on page 341, which tells us that:

\[ E = \sqrt{1-\frac{2M}{r}} E_{local}\]

    Which does in fact translate directly to what we want. The question of
where does this come from and why wasn't our previous method correct is
rather simple: we didn't consider the coordinate system where the photon
STARTED. Obviously there needed to be a transform. AS we said when we
started, energy is ill-defined, but the values can still be transformed.
12.4.9 gives the above almost directly, as E correlates directly to
\(h\nu\). the h cancels, and the frequencies become the ratio. Tah-dah!

Oh even better 5.3.3 gives the exact relation based on the metrics.

    \hypertarget{problem-3-back-to-top}{%
\section{\texorpdfstring{Problem 3 {[}Back to
\hyperref[toc]{top}{]}}{Problem 3 {[}Back to {]}}}\label{problem-3-back-to-top}}

\[\label{P3}\]

\emph{Show that the same observer at r finds that the tangential
velocity of the particle satisfies}
\[ v^\hat{\phi} = \left( 1 - \frac{2M}{r} \right)^{1/2} \frac{\tilde l}{r\tilde E} \]
\emph{So that \(v^\hat{\phi} \rightarrow 0\) as \(r \rightarrow 2M\)}

    We finally realized what this is: the book gives r and \(\theta\)
results, so we get to find \(\phi\) results. So let's just try to
evaluate it, starting with\ldots{}

\[ v^{\hat\phi} = \frac{p^{\hat\phi}}{p^{\hat t}} \]

Flipping the metric does nothing for a physial coordinate.

\[ = \frac{p_{\hat\phi}}{p^{\hat t}} \]

Bottom is local energy, top can be represented as a dot product.

\[ = \frac{\vec p \cdot \vec e_{\hat\phi}}{E_{local}} \]

The dot product can be converted to the other (non-hat) reference frame.
..and it turns out we're idiots, \(\phi\) was done, \(\theta\) was not
done because it was kept constant. Either way, the conversion is\ldots{}

\[ = \frac{\vec p \cdot \vec e_{\phi}/r}{E_{local}} \]

Now that our vectors have been orthonormalized, the dot product
evaluates.

\[ = \frac{p_\phi/r}{E_{local}} \]

Now we can convert the local to become nonlocal.

\[ = \frac{\sqrt{1-\frac{2M}{r}} p_\phi/r}{E} \]

And that momentum value there just \emph{is} l, the angular momentum,
which becomes\ldots{}

\[ = \frac{\sqrt{1-\frac{2M}{r}} l}{rE} \]

And adding the tilde notation only means dividing by m, so if we do it
on top and the bottom it changes nothing.

\[ = \frac{\sqrt{1-\frac{2M}{r}} \tilde l}{r\tilde E} \]

And we're done!

    \hypertarget{problem-4-back-to-top}{%
\section{\texorpdfstring{Problem 4 {[}Back to
\hyperref[toc]{top}{]}}{Problem 4 {[}Back to {]}}}\label{problem-4-back-to-top}}

\[\label{P4}\]

\emph{a) Show from 12.4.17 that a local observer at r finds that hte
velocity of a radially freely-falling particle released from rest at
infinity is given by}

\[ v^{\hat r} = \sqrt{\frac{2m}{r}} \]

\emph{which has precisely the same form as the Newtonian velocity!}

    12.4.17 is

\[ v^{\hat r} = \sqrt{1 - \frac{1}{\tilde E^2} \left( 1-\frac{2M}{r} \right) \left( 1+\frac{\tilde l^2}{r^2} \right) } \]

    Now, relesed from rest implies quite obviously that the angular momentum
is zero. So that can be taken out right away.

\[ v^{\hat r} = \sqrt{1 - \frac{1}{\tilde E^2} \left( 1-\frac{2M}{r} \right)} \]

    Now, that energy is actually E/m because of the tilde. And, even more
interesting, we start at rest, which means E = m! So it's just\ldots{}
1! And that reduces the expression all the way down to:

\[ v^{\hat r} = \sqrt{\frac{2M}{r}} \]

Success!

    \emph{b) Obtain the same result from 12.4.9, noting that
\(E_{local} = \gamma m.\)}

    12.4.9 is our friend

\[ E = \sqrt{1-\frac{2M}{r}} E_{local}\]

    Now, replace the local energy with the suggested substitution.

\[ E = \sqrt{1-\frac{2M}{r}} \gamma m \]

\(\gamma\) is our other friend \(\frac{1}{\sqrt{1-v^2}}\). The only
velocity we can have is a radial velocity in this situation, so that v
is what we are looking for! And as before E=m.

\[ m = \sqrt{1-\frac{2M}{r}} \gamma m \]
\[ \Rightarrow 1 = \sqrt{1-\frac{2M}{r}} \frac{1}{\sqrt{1-v^2}}\]
\[ \Rightarrow \sqrt{1-v^2} = \sqrt{1-\frac{2M}{r}} \]
\[ \Rightarrow 1-v^2 = 1-\frac{2M}{r} \]
\[ \Rightarrow v^2 = \frac{2M}{r} \]
\[ \Rightarrow v = \sqrt{\frac{2M}{r}} \]

Once again, we have achieved success.

    \hypertarget{problem-5-back-to-top}{%
\section{\texorpdfstring{Problem 5 {[}Back to
\hyperref[toc]{top}{]}}{Problem 5 {[}Back to {]}}}\label{problem-5-back-to-top}}

\[\label{P5}\]

\emph{A particle moves along a geodesic from r and \(\phi\) to r+dr and
\(\phi + d\phi\) in time dt. A local static observer at (r,\(\phi\))
measures the proper length of the particle's path to have increased by
ds(t,\(\theta\),\(\phi\)=const) = \(g^{1/2}_{rr} dr(=d\hat r)\) and
ds(t,r,\(\theta\)=const)=\(g^{1/2}_{\phi\phi} d\phi (=d\hat\phi)\) in
the r and \(\phi\) respectively, during this time; the proper time for
this motion as measured on the observer's clock lasts
\([-ds^2(r,\theta,\phi=const)]^{1/2} = (-g_{00})^{1/2} dt (=d\hat t)\).
{[}Note that \(d\hat t\) for the \textbf{observer} is \textbf{not} equal
to d\(\tau\) appearing, e.g., in 12.4.13-12.4.15 for the particle!{]}
Use the expressions for these measurements together with 12.4.13-12.4.15
to rederive 12.4.17 ad 12.4.18.}

    12.4.17 and 12.4.18, our goals, are:

\[ v^\hat{\phi} = \left( 1 - \frac{2M}{r} \right)^{1/2} \frac{\tilde l}{r\tilde E} \]

\[ v^{\hat r} = \sqrt{1 - \frac{1}{\tilde E^2} \left( 1-\frac{2M}{r} \right) \left( 1+\frac{\tilde l^2}{r^2} \right) } \]

    And the equations 13 through 15 are

\[ \left( \frac{dr}{d\tau} \right)^2 = \tilde E^2 - \left( 1-\frac{2M}{r} \right) \left( 1+\frac{\tilde l^2}{r^2} \right)\]

\[ \frac{d\phi}{d \tau} = \frac{\tilde l}{r^2} \]

\[ \frac{dt}{d\tau} = \frac{\tilde E}{1-\frac{2M}{r}} \]

    The warning is true: \(\hat t\) is related to the proper time as far as
the observer is concerned, \(d\tau\) is the proper time for the actual
particle doing the moving. So, in short, we need to be very meticulous
and careful here.

Anyway, the relations we are given ar ea bit obfuscated, but ammount to:

\[ d\hat r  = \left( 1-\frac{2M}{r} \right)^{-1/2} dr\]
\[ d\hat \phi  = rsin\theta d \phi\]
\[ d\hat t  = \left( 1-\frac{2M}{r} \right)^{1/2} dt\]

Though with our restrictions \(\theta = \pi/2\) and so

\[ d\hat \phi  = r d \phi\]

    Now, take derivatives of everything with respect to \(d\tau\) by
dividing by it, get some results:

\[ \frac{d\hat r}{d \tau}  = \left( 1-\frac{2M}{r} \right)^{-1/2} \frac{1}{d\tau}dr \]
\[ \frac{d\hat \phi}{d\tau}  = r \frac{1}{d\tau}d \phi\]
\[ \frac{d\hat t}{d\tau}  = \left( 1-\frac{2M}{r} \right)^{1/2} \frac{1}{d\tau} dt \]

Which can all be substituted by the suggested expressions.

\[ \frac{d\hat r}{d \tau}  = \left( 1-\frac{2M}{r} \right)^{-1/2} \sqrt{\tilde E^2 - \left( 1-\frac{2M}{r} \right) \left( 1+\frac{\tilde l^2}{r^2} \right)} \]
\[ \frac{d\hat \phi}{d\tau}  = r \frac{\tilde l}{r^2}\]
\[ \frac{d\hat t}{d\tau}  = \left( 1-\frac{2M}{r} \right)^{1/2} \frac{\tilde E}{1-\frac{2M}{r}} \]

    Simplify\ldots{}

\[ \frac{d\hat r}{d \tau}  = \left( 1-\frac{2M}{r} \right)^{-1/2} \sqrt{\tilde E^2 - \left( 1-\frac{2M}{r} \right) \left( 1+\frac{\tilde l^2}{r^2} \right)} \]
\[ \frac{d\hat \phi}{d\tau}  = \frac{\tilde l}{r}\]
\[ \frac{d\hat t}{d\tau}  = \left( 1-\frac{2M}{r} \right)^{-1/2} \tilde E \]

    Now the waning comes into play: the proper time of the particle is NOT
what we are seeking, and the proper time we have here isn't the proper
time as far as the observer is conscerned. Hence, our curious little
poser problem. The VELOCITY is given by:

\[v^{\hat i} = \frac{p^{\hat i}}{p^\hat t} = \frac{d\hat i/d\lambda}{d\hat t/d\lambda} = \frac{d\hat i}{d\hat t}\]

Which is to say a more direct speed.

    Now we actually have an expression with the time coordinate we need, so
how about\ldots{} we do the insane and \emph{divide} by it?

\[ \frac{d\hat r}{d \hat t}  = \left( 1-\frac{2M}{r} \right)^{-1/2} \sqrt{\tilde E^2 - \left( 1-\frac{2M}{r} \right) \left( 1+\frac{\tilde l^2}{r^2} \right)} \frac{1}{\left( 1-\frac{2M}{r} \right)^{-1/2} \tilde E} \]
\[ \frac{d\hat \phi}{d \hat t}  = \frac{\tilde l}{r} \frac{1}{\left( 1-\frac{2M}{r} \right)^{-1/2} \tilde E}\]

    The top one conveniently becomes\ldots{}

\[ \frac{d\hat r}{d \hat t}  = \sqrt{\tilde E^2 - \left( 1-\frac{2M}{r} \right) \left( 1+\frac{\tilde l^2}{r^2} \right)} \frac{1}{\tilde E} \]
\[  = \sqrt{1 - \frac{1}{\tilde E^2}\left( 1-\frac{2M}{r} \right) \left( 1+\frac{\tilde l^2}{r^2} \right)} \]

And would you look at that it's exactly what we wanted. As for the
second one\ldots{}

    \[ \frac{d\hat \phi}{d \hat t}  = \frac{\tilde l}{r} \frac{1}{\left( 1-\frac{2M}{r} \right)^{-1/2} \tilde E}\]
\[  = \frac{\tilde l}{r \tilde E}\left(1-\frac{2M}{r} \right)^{1/2} \]

Which, also, is exactly what we wanted.

    \hypertarget{problem-6-back-to-top}{%
\section{\texorpdfstring{Problem 6 {[}Back to
\hyperref[toc]{top}{]}}{Problem 6 {[}Back to {]}}}\label{problem-6-back-to-top}}

\[\label{P6}\]

\emph{a) Integrate 12.4.20 for the case \(\tilde E > 1\), so that
\(1 - \tilde E^2 = 2M/R\), to get (\(\tau=0\) at r=R)}:

\[ \tau = \left( \frac{R^3}{8M} \right)^{1/2} \left[ 2 \left( \frac{r}{R} - \frac{r^2}{R^2} \right)^{1/2} + acos \left( \frac{2r}{R} - 1 \right) \right] \]

    12.4.20 states

\[ \frac{dr}{d\tau} = -\left( \tilde E^2 - 1 + \frac{2M}{r} \right)^{1/2} \]

the rule given in the problem suggests a very obvious substitution that
turns this into

\[ \frac{dr}{d\tau} = -\left( \frac{2M}{R} + \frac{2M}{r} \right)^{1/2} = -\sqrt{2M}\left( \frac{1}{R} + \frac{1}{r} \right)^{1/2} \]

    Of cousre we actually want to find the proper time so the derivative is
currently upside down. Let's pretend it's a fraction and just invert it.
Surely this won't cause any problems later.

\[ \frac{d\tau}{dr} = -\frac{1}{\sqrt{2M}}\left( \frac{1}{R} + \frac{1}{r} \right)^{-1/2} \]

And thus integrating with respect to r should get us our result!
\ldots We sure hope the computer makes this integral simpler than it
looks like it is! \ldots And yep it's a nightmare. We ended up using
\hyperref[1]{1} to evaluate it since our good friend Geogebra threw sgn
functions at us. It's still not friendly in the end but at least we have
a result.

    \[ \tau = \frac{1}{\sqrt{2M}}\frac{\sqrt{R}(R(ln(\sqrt{\frac{R+r}{r}}+1) - ln(|\sqrt{\frac{R+r}{r}}-1|))-2r\sqrt{\frac{R+r}{r}}}{2} + C   \]

Before we try to find out what C is, let's simplify a bit.

    \[ \tau = \sqrt{\frac{R^3}{8M}}\left( ln(\sqrt{\frac{R}{r}+1}+1) - ln(|\sqrt{\frac{R}{r}+1}-1|)-\frac{2r}{R}\sqrt{\frac{R}{r}+1} \right) + C   \]

    Geogebra points out that, yes, the integral is in fact correct as we
took it. However, the form doesn't match the one the answer is supposed
to take at all. So, clearly, we misunderstood \emph{something}. Leaving
this for now, next part should be doable regardless.

    \emph{b) Introduce the ``cycloid parameter''" \(\eta\) by}
\[ r = \frac{R}{2}(1+cos\eta) \] \emph{And show that}
\[ \tau = \left( \frac{R^3}{8M} \right)^{1/2} (\eta + sin\eta) \]

    Take the previous answer and start substituting:

\[ \tau = \left( \frac{R^3}{8M} \right)^{1/2} \left[ 2 \left( \frac{r}{R} - \frac{r^2}{R^2} \right)^{1/2} + acos \left( \frac{2r}{R} - 1 \right) \right] \]
\[ = \left( \frac{R^3}{8M} \right)^{1/2} \left[ 2 \left( \frac{\frac{R}{2}(1+cos\eta)}{R} - \frac{(\frac{R}{2}(1+cos\eta))^2}{R^2} \right)^{1/2} + acos \left( \frac{2\frac{R}{2}(1+cos\eta)}{R} - 1 \right) \right] \]

    \[ = \left( \frac{R^3}{8M} \right)^{1/2} \left[ \left( 2(1+cos\eta) - (1+cos\eta)^2\right)^{1/2} + acos \left( (1+cos\eta) - 1 \right) \right] \]

\[ = \left( \frac{R^3}{8M} \right)^{1/2} \left[ \left( 2+2cos\eta - 1  -2cos\eta -  cos^2\eta \right)^{1/2} + acos \left(cos\eta\right) \right] \]

\[ = \left( \frac{R^3}{8M} \right)^{1/2} \left[ \left(1 - cos^2\eta \right)^{1/2} + \eta \right] \]

\[ = \left( \frac{R^3}{8M} \right)^{1/2} \left[ (sin^2\eta)^{1/2} + \eta \right] \]

\[ = \left( \frac{R^3}{8M} \right)^{1/2} \left[ sin\eta + \eta \right] \]

Behold, the answer.

    \emph{c) Integrate 12.4.15 for t in terms of \(\eta\) to get (t=0 at
r=R):}

\[ \frac{t}{2M} = ln\left| \frac{(R/2M - 1)^{1/2}+tan(\eta/2)}{(R/2M - 1)^{1/2}-tan(\eta/2)} \right| + \left( \frac{R}{2M}-1 \right)^{1/2}\left[ \eta + \frac{R}{4M}(\eta+sin\eta)\right]\]

    12.4.15 gives us

\[ \frac{dt}{d\tau} = \frac{\tilde E}{1-2M/r} \]

Which, with our substitution, provides:

\[ \frac{dt}{d\tau} = \frac{\tilde E}{1-\frac{2M}{\frac{R}{2}(1+cos\eta)}} \]
\[ \frac{dt}{d\tau} = \frac{\tilde E}{1-\frac{4M}{R(1+cos\eta)}} \]

\ldots.You know what, we're just going to skip this. Doing long invovled
integrals by hand doesn't really illuminate much of anything.

    \hypertarget{problem-7-back-to-top}{%
\section{\texorpdfstring{Problem 7 {[}Back to
\hyperref[toc]{top}{]}}{Problem 7 {[}Back to {]}}}\label{problem-7-back-to-top}}

\[\label{P7}\]

\emph{a) Find \(\tau(r)\) and t(r) for radial infall when
\(\tilde E = 1\)}

    By 12.4.20

\[ \frac{dr}{d\tau} = -\left( \tilde E^2 - 1 + \frac{2M}{r} \right)^{1/2} \]

With our given limitation this becomes

\[ \frac{dr}{d\tau} = -\left( \frac{2M}{r} \right)^{1/2} \]

We want it in terms of proper time, so we flip it.

\[ \frac{d\tau}{d r} = -\left( \frac{r}{2M} \right)^{1/2} \]

Integrating with respect to r is actually easy in this case.

\[ \tau = -\frac{1}{\sqrt{2M}}  \frac23 r^{3/2} + C\]

    If provided with a starting radius R and an assumed proper time of 0, C
can be determined.

\[ \tau = -\frac{1}{\sqrt{2M}}  \frac23 r^{3/2} + \frac{1}{\sqrt{2M}}  \frac23 R^{3/2}\]

Note that \(\tau=0\) when r=R. The smaller r gets, the greater \(\tau\)
becomes until it hits a finite value that is C, which is what we expect.
Keep in mind that this is not the same case as was solved previously in
the book: the ``=1'' criterion means that this particle is falling as
though it were at rest at infinity. Naturally, if we actually tried to
start the particle at infinity it would take an infinite amount of time
to reach the black hole. Our choice of R thus is rather arbitrary, as
it's just a point where time ``starts,'' the particle is not at rest
there. Hence why the result is not exactly the same, though the
R-dependence is.

    Now, what about the obesrved time at infinity? (The observer is sitting
at infinity, not the particle\ldots{} necessarily.) Let's start with
12.4.15.

\[ \frac{dt}{d\tau} = \frac{\tilde E}{1-2M/r} \]

And like the crazy that we are we can pretend the differnetials are
fractions and multiply through the differential form of our previous
answer to get

\[ \frac{dt}{dr} = -\frac{\tilde E}{1-2M/r}\left( \frac{r}{2M} \right)^{1/2} \]

E goes away due to our limitation.

\[ \frac{dt}{dr} = -\frac{1}{1-2M/r}\left( \frac{r}{2M} \right)^{1/2} \]

Which can be rewritten

\[ \frac{dt}{dr} = -\frac{r^2}{\sqrt{2M}r-(2M)^{3/2}} \]

Which we then beg the comptuer to integrate for us. And hey, it actually
produces something that doesn't look insane! \hyperref[1]{1} was used
over geogebra since its results were neater.

\[ t = - \frac{ 8M^2 ln(|r-2M|) + r^2 + 4Mr }{2^{3/2} \sqrt{M}} + C \]

    Now when R=r, the time needs to equal zero, so this produces the exact
equation:

\[ t = - \frac{ 8M^2 ln(|r-2M|) + r^2 + 4Mr }{2^{3/2} \sqrt{M}} + \frac{ 8M^2 ln(|R-2M|) + R^2 + 4MR }{2^{3/2} \sqrt{M}} \]

    Now we note that there is a discontinuity where there should be,
ln(r-2M) when r=2M is not the happy times. Except it is for us becasue
it's what we were expecting. However, we still need to check the rest of
the behavior: does the time keep increasing as r decreases? Yes! in fact
we have a nice graph here:

\begin{figure}
\centering
\includegraphics{attachment:Screenshot\%20from\%202022-06-27\%2012-40-43.png}
\caption{Screenshot\%20from\%202022-06-27\%2012-40-43.png}
\end{figure}

    The bevahior \emph{inside} the event horizon is also fascinating, though
may not be realistic. Almost like time is winding backward in there.
Which usually is an absurd conclusion but black holes are weird enough
that you might not know for sure\ldots{} Regardless, falling IN, time
increases without bound. At least, as far as the observer is concerned.

    \emph{b) Find \(\tau(r), r(\eta), \tau(\eta), t(\eta)\) when
\(\tilde E > 1\). You can get these from 12.4.21-24 by defining R such
that 2M/R = \(\tilde E-1\) and changing the sign of R in these
equations. Show that 2M/R = \(v^2_\infty / (1-v^2_\infty)\)}

    Ah yes, the formulae of \textbf{Problem 6} that were basically
impossible to deal with from integreation funk. However if we take the
problem here at face value, we should be able to get our relations just
by replacing R with -R. At which point it becomes trivial.

\[ \tau = \left( \frac{-R^3}{8M} \right)^{1/2} \left[ 2 \left( -\frac{r}{R} - \frac{r^2}{R^2} \right)^{1/2} + acos \left( -\frac{2r}{R} - 1 \right) \right] \]

\[ r = -\frac{R}{2}(1+cos\eta) \]

\[ \tau = \left( -\frac{R^3}{8M} \right)^{1/2} (\eta + sin\eta) \]

\[ \frac{t}{2M} = ln\left| \frac{(-R/2M - 1)^{1/2}+tan(\eta/2)}{(-R/2M - 1)^{1/2}-tan(\eta/2)} \right| + \left( \frac{-R}{2M}-1 \right)^{1/2}\left[ \eta + \frac{-R}{4M}(\eta+sin\eta)\right]\]

    Okay so now that we have these, we want to show that the initial
velocity at infinity is related to 2M/R. This is essentially the same as
saying \(\tilde E - 1 = \frac{2M}{R}\) So if we can prove:

\[ \tilde E - 1 = \frac{v^2_\infty}{1-v^2_\infty} \]

We have it.

    We note that radial velocity is a known quantity of \$ \sqrt{2M}{r} \$
so we can rewrite our relation as:

\[\frac{2M/r}{1-2M/r} = (1-2M/r)\frac{dt}{d\tau} - 1\]

Now we can re-arrange this to say:

\[ \frac{d\tau}{dt} = \left( 1 - 2M/r \right)^2  \]

And we can potentailly get this derivative. Annoyingly the direct
solution 12.4.15 has \(\tilde E\) in it so we're going to have to go
around a backdoor to resolve this one. That backdoor is:

\[ \frac{dt}{d\eta}\frac{d\eta}{dr}\frac{dr}{dt} \]

Now the last one is just the velocity itself, which we arleady discussed
is \$ \sqrt{2M}{r} \$.

We have r(\(\eta\)) so we just reverse it, getting an equation

\[ \eta = acos\left( -\frac{2r}{R}-1 \right) \]

    And we stop ourselves here because that THIRD equation is not getting
differentiated with respect to \(\eta\). (see the t(\(\eta\)) function
above.) Just\ldots{} no, defintely no, nope. Just going to\ldots{} leave
this.

    \hypertarget{problem-8-back-to-top}{%
\section{\texorpdfstring{Problem 8 {[}Back to
\hyperref[toc]{top}{]}}{Problem 8 {[}Back to {]}}}\label{problem-8-back-to-top}}

\[\label{P8}\]

\emph{Show that 12.4.26 reduces to the familiar Newtonian expression for
particle motion in a central gravitational field when 2M/r
\textless\textless{} 1.}

First of all we had to remind ourselves of the goal with
\hyperref[2]{2}. Via Lagrangian Mechanics (and fumbling around) we
produced:

\[ \ddot r = \dot r - r \dot \phi^2 \]

Though this is the completely arbitrary case, we have a gravitaional
field. Via the convenience of not needing to assign a constraint, we can
just insert the force in like so:

\[ \ddot r = \dot r - r \dot \phi^2 - \frac{GMm}{r^2} \]

Also kind of has to be this, as when at rest the acceleration has to be
just the force.

    Which is pretty sensible: the normal acceleration term with an added
centrifugal term.

\[\left( \frac{dr}{d\tau} \right)^2 = \tilde E^2 - V(r)\]

So first of all let's examine V(r). It is given by:

\[ V(r) = \left( 1-\frac{2M}{r} \right) \left( 1+\frac{\tilde l^2}{r^2} \right) \]

Under our small limit, we end up with:

\[ V(r) =  1+\frac{\tilde l^2}{r^2} \]

From the definition of angular momentum\ldots{}

\[ V(r) =  1+\frac{(m\dot\phi r)^2}{m^2r^2} = 1 + \dot\phi^2 \]

    So now our original expression is:

\[\left( \frac{dr}{d\tau} \right)^2 = \tilde E^2 - 1 - \frac{\tilde l}{r^2}\]

Since we're going for the netwonian limit, the proper time becomes just
the time and we can replace the left side with the simple radial
velocity.

\[\dot r^2 = \tilde E^2 - 1 - \frac{\tilde l}{r^2}\]

Now E is equal to \((1-2M/r)\dot t\). However, the rate at which time
ticks is one second per second in Netwonian, so it simply reduces to 1.
\(\tilde E\) reduces to 1/m.

\[\dot r^2 = \frac{1}{m^2} - 1 - \frac{\tilde l}{r^2}\]

Substitute for l as well\ldots{}

\[\dot r^2 = \frac{1}{m^2} - 1 - \frac{r^2\dot \phi}{mr^2}\]
\[\Rightarrow \dot r^2 = \frac{1}{m^2} - 1 - \frac{\dot \phi}{m}\]

    There have been a lot more nonsense scribbled on notebook paper, but
none of it gets to what the result should be.

    \hypertarget{problem-9-back-to-top}{%
\section{\texorpdfstring{Problem 9 {[}Back to
\hyperref[toc]{top}{]}}{Problem 9 {[}Back to {]}}}\label{problem-9-back-to-top}}

\[\label{P9}\]

\emph{a) Show that \(\partial V / \partial r = 0\) when}

\[ Mr^2 - \tilde l^2 r + 3M \tilde l^2 = 0 \]

\emph{and hence that there are no maxima or minima of V for
\(\tilde l < 2\sqrt{3}M\)}

    See we did these parts out of order. So we already know the derivative.

\[ V' = \frac{2M}{r^2} - \frac{2\tilde l^2}{r^3} + \frac{6M \tilde l^2}{r^4}  \]

Similar solution method, set to zero, turn into a quadratic.

\[ 0 = \frac{2M}{r^2} - \frac{2\tilde l^2}{r^3} + \frac{6M \tilde l^2}{3r^4}  \]

\[ \Rightarrow 0 = 2Mr^2 - 2\tilde l^2r + 6M \tilde l^2  \]

\[ \Rightarrow 0 = Mr^2 - \tilde l^2r + 3M \tilde l^2  \]

Okay so maybe we see why this part came first. Ehe.

    \emph{b) Show that \(V_{max}\) = 1 for \(\tilde l = 4M\)}

    Our friend V is:

\[ V(r) = \left( 1-\frac{2M}{r} \right) \left( 1+\frac{\tilde l^2}{r^2} \right) \]

Do the suggested substitution and then have fun with it:

\[ V(r) = \left( 1-\frac{2M}{r} \right) \left( 1+\frac{16M^2}{r^2} \right) \]

    \[ = \left( 1 - \frac{2M}{r} + \frac{16M^2}{r^2} - \frac{32M^3}{r^3} \right) \]

Now take the derivative and look for zeroes.

\[ 0 = 0 + \frac{2M}{r^2} - \frac{32M^2}{r^3} + \frac{96M^3}{r^4}  \]
\[ \Rightarrow 0 + \frac{1}{r^2} - \frac{16M}{r^3} + \frac{48M^2}{r^4}  \]
\[ \Rightarrow 0 = r^2 - 16Mr + 48M^2  \]

Solve the quadratic formula.

    Answers: 4M or 12M. Plugging these back into the original expression for
V results in

V(r) = (1 - 1/2)(1 + 1) = (1/2)(2) = 1

V(r) = (1 - 1/6)(1 + 1/9) = (5/6)(10/9) = 50/54 = 25/27

We're not quite done yet, though. 1 is simply the largest MAXIMA, is it
the absolute maximum? We could conceivably grow without bound. Well, as
r goes to 0 the expression diverges to negative infinity as the cubic
term dominates. As r goes to infinity, it the expression \emph{also}
approaches 1, so it's the same maximum in both places. So yes, now we
have proven it.

Aslo as for our 12M point, that's a local minimum.

    \hypertarget{problem-10-back-to-top}{%
\section{\texorpdfstring{Problem 10 {[}Back to
\hyperref[toc]{top}{]}}{Problem 10 {[}Back to {]}}}\label{problem-10-back-to-top}}

\[\label{P10}\]

\emph{Show that the circular Schwarzchild orbits are stable if
r\textgreater6M and unstable if r\textless6M}

Okay so we could do this one algebraically but we want to do it
graphically because it'd be cool to see it. Given an angular momentum l
we can find a radius of circular orbit r which will correspond to the
larger-r solution to the quadratic formula in \textbf{Problem 9}. Using
the completely general form, this becomes:

\[ r = \frac{\frac{\tilde l^2}{M} + \sqrt{\frac{\tilde l^4}{M^2} - 12 \tilde l^2}}{2} \]

    Setting M=1 since we want units in terms of M anyway, we arrive at:

\[ r = \frac{\tilde l^2 + \sqrt{\tilde l^4 - 12 \tilde l^2}}{2} \]

And with angular momentum on x and r on y, we get:

\begin{figure}
\centering
\includegraphics{attachment:Screenshot\%20from\%202022-06-27\%2015-44-25.png}
\caption{Screenshot\%20from\%202022-06-27\%2015-44-25.png}
\end{figure}

My would you look at that, r cannot have a magnitude of less than six
(in units of M), the extrema no longer exists. This corresponds with the
moment the radical becomes negative but we're not going to calculate
that out.

    \hypertarget{problem-11-back-to-top}{%
\section{\texorpdfstring{Problem 11 {[}Back to
\hyperref[toc]{top}{]}}{Problem 11 {[}Back to {]}}}\label{problem-11-back-to-top}}

\[\label{P11}\]

\emph{a) Show that in Newtonian theory, a distant nonrelativistic test
particle can only be captured by a star of mass M and radius R if
\(\tilde l < \tilde l_{crit} \approx \sqrt{2MR}\)}

    Okay so let's think about how to do this, and what exactly the
expression means. The thing is, angular momentum is conserved, so any
object with any angular momentum will keep that angular momentum in an
isolated particle-star system. In an isolated system this means to stay
in orbit the particle would have to already \emph{be} in orbit. The
limit between being in orbit and NOT being in orbit is a parabolic
trajectory: less angular momentum is an ellipse, more is a hyperbola. So
what we need to do is find the angular momentum of a parabolic orbit,
and that will be our limit. The only R that makes sense in this equation
is the R of closest approach, as no other R except for infinity is
prefered. M is the mass of the body. The mass of the particle is
irrelevant as we will be dividing it out.

Angular momentum is mvr, so our divided out angular momentum is vr. At
closest approach it is vR, where v is just the velocity at that moment.
By conservation of energy, we note that at infinity the particle has no
speed, no energy, and no gravitaitonal potential. (This is not true for
a hypebolic orbit, where the particle can truly escape--the parabola is
the boundary condition, and thus has these extrema.) Thus the energy for
the particle out this distant is 0. The energy at the closest point is
kinetic energy, and gravitational energy--the latter of which is
negative. NOTE: as the particle itself is not rotating there is no
``rotational kinetic energy''. In fact \(I\omega^2 = mv^2\).

\[ E = \frac12 mv^2 - \frac{GMm}{r} = 0\]
\[ \Rightarrow \frac{GMm}{r} = \frac12 mv^2 \]
\[ \Rightarrow \frac{GM}{r} = \frac12 v^2 \]
\[ \Rightarrow 2GMr =  v^2r^2 \]
\[ \Rightarrow \sqrt{2GMr} = \tilde L \]

Now in general relativity G=1 and we already said r=R so\ldots{}

\[ \tilde L = \sqrt{2MR} \]

NOTE TO SELF: seriously, seriously, seriously remember the difference
betwewn GMm/r and \(GMm/r^2\). One is an energy, the other is a force.
Force. Using the wrong one can be catastrophic.

    \emph{b) Taking into account general relativity, can particles with much
larger values of angular momentum be captured by neutron stars? By whtie
dwarves?}

    Now whatever type of orbit this takes, our assumption for the energy at
infinity must remain the same--we are looking for the border, so the
energy is reduced to zero at infinity. Unmoving, but no gravity either.
However, energy is quite\ldots{} \emph{funky} in the situations here.
Let's examine the energy relation between ``local'' energy and
``observed'' energy.

\[ E = \sqrt{1-\frac{2M}{r}} E_{local}\]

Now, we know that E itself is conserved (it is the energy-at-infinity),
as it is an absolute quantity. However, the \(E_{local}\) is not. Rather
than worrying about how exactly this realtes, we note that 12.4.11 gives
us this lovely equation:

\[ l = E_{local}rv^{\hat\phi} \]

Which means we have an expression for the angular momentum in terms of
the local energy, which we then write as:

\[ l = \frac{1}{\sqrt{1-\frac{2M}{r}}}Erv^{\hat\phi} \]

    This is actually relativistic energy, though. Fortunately for us it is
just the mass of the partile as all other energy is 0, which we then
divide out to get what we want.

\[ \tilde l = \frac{1}{\sqrt{1-\frac{2M}{r}}}rv \]

Let's find this at the point of closest approach.

\[ \tilde l = \frac{1}{\sqrt{1-\frac{2M}{R}}}Rv \]

Now this is our new form of l = Rv. Now from part a) we already have rv,
it's the normal angular momentum: \(\sqrt{2MR}\) Which means our result
is:

\[ \tilde l = \frac{\sqrt{2MR}}{\sqrt{1-\frac{2M}{R}}} \]

    The controlling factor here is the M/R ratio. in the relativistic limit
M/R \textless\textless{} 1, that is, R dominates, and we end up with our
original expression. However, that ratio can grow\ldots{} at 2M/R = 1
there's a discontinuity, but at that point is the event horizon. At the
event horizon the angular momentum increases without bound, which makes
sense--at that point even an infinite angular momentum couldn't escape.
So black holes can not only absorb large angular momentum, they can
absorb infinitely large angular momentum.

However, what about neutron stars and white dwarves? Well, that depends
entirely on the M/R ratio. We take R to be the extreme: just barely
above the surface. Note that everything must be in relativistic units.
Distances remain the same in meters, but 1 kg is converted to 7.425e-28
m (General Relativity pg 187). Also helpful to know that a solar mass is
1.476e3 m.

So let's dig for some numbers. Table 1.1 provides some rough estimates,
which we take the extreme limits of.

Neutron stars have 3 solar masses at maximum and around 7000 m radius.
Except\ldots{} these numbers would create a black hole, as 5000
\textless{} 8856. making 2M/R \textgreater{} 1. But that's right,
neutron stars get \emph{smaller} as they get larger, so all right fine
let's use a lower limit neutron star mass, 1.4 solar masses. 4132 / 5000
= .826.

In other words a pretty huge difference. This increases the angular
momentum that can be captured by a factor of 2.4. Keep in mind this is a
low-end neutron star, so higher-end ones most likely could take even
more.

However, white dwarfs at the upper end have roughly the same
mass\ldots{} but a much larger radius.

4132 / 5000000 = 8.264e-4. Which is to say, not \emph{unnoticable} but
still small. This corresponds to factor of 1.0004. So you could probably
measure this.

    \hypertarget{problem-12-back-to-top}{%
\section{\texorpdfstring{Problem 12 {[}Back to
\hyperref[toc]{top}{]}}{Problem 12 {[}Back to {]}}}\label{problem-12-back-to-top}}

\[\label{P12}\]

\emph{a) Use 12.4.18 to show that the velocity of a particle in the
innermost stable circular orbit as measured by a local static observer
is \(v^{\hat\phi} = \frac12 (c=1)\)}

    12.4.18 is

\[ v^\hat\phi = \left( 1 - \frac{2M}{r} \right)^{1/2} \frac{\tilde l}{r \tilde E} \]

We know from \textbf{Problem 10} that the innermost stable circular
orbit is always at r=6M. So we make that substitution.

\[ = \left( 1 - \frac{2M}{6M} \right)^{1/2} \frac{\tilde l}{6M \tilde E} \]

\[ = \left( 1 - 1/3 \right)^{1/2} \frac{\tilde l}{6M \tilde E} \]

\[ = \left( 2/3 \right)^{1/2} \frac{\tilde l}{6M \tilde E} \]

    Now we need the angular momentum and the energy. Note that since both
quantities are divided by m, the difference between them is irrelevant.
However, how can we determine the energy and angular momentum of
something that deep in a relativistic gravity well? Keep in mind that
this is a LOCAL observer as well.

We can solve for \(\tilde l\) via \textbf{Problem 9}, getting

\[ 0 = M(6M)^2 - \tilde l^26M + 3M \tilde l^2  \]
\[ \tilde l = \sqrt{12}M \]

    So now the question is what is the energy? \ldots Turns out 12.4.28
could have given us the above, and 12.4.9 gives us the following for E.

\[ \tilde E^2 = \frac{(r-2M)^2}{r(r-3M)} = \frac{16M^2}{18M^2} \]
\[ \Rightarrow \tilde E = \frac{4}{\sqrt{18}} \]

    Spent way too long stuck in algebra here for something so simple but
here it is:

\[ v^{\hat\phi} = \frac{\sqrt{2}}{\sqrt{3}} \frac{1}{6M} \frac{6M}{\sqrt{3}} \frac{3\sqrt{2}}{4}\]
\[ = 2/4 = 1/2 \]

Which is exactly what it should be.

    *b) Suppose that the particle in part a) is emitting monochromatic light
at frequency \(\nu_{em}\) in its rest frame. Show that the frequency
received at infinity varies periodically between \$
\frac{\sqrt{2}}{3}\nu\emph{\{em\} \textless{} \nu}\infty \textless{}
\sqrt{2} \nu\_\{em\}\$. Hint: write
\(\nu_\infty/\nu_{em} = (\nu_\infty/\nu_{stat})(\nu_{stat}/\nu_{em})\)
where \(\nu_{stat}\) is the frequency measured by the local static
observer and is related to \(\nu_{em}\) by the special relativistic
Doppler formula.*

    Physically the extremes are rather obvious: the maximum happens during
the approachiung portion of the orbit, the minimum happens during the
receeding. The exact question is how do we represent this, as the
velocities most certainly do \emph{not} add. However we can track down
the special relativistic doppler formula 5.3.3

\[ \frac{\nu_{rec}}{\nu_{em}} = \frac{\sqrt{-g_{00}}_{em}}{\sqrt{-g_{00}}_{rec}} \]

    So what we need is the metric of the particle, the metric of the
stationary observer and\ldots{} we actually don't need the metric
because we have the formula for static emitters already as 12.4.10. If
we assume the local observer is static and also at r=6M, this means the
redshift \(\frac{\nu_\infty}{\nu_{stat}} = \sqrt{2/3}\)

In our solution we tried to transform a metric g by the Lorentz
transformation. This turned out to not be the proper way to do it even
though it produced the exact answer in our incorrect manner of thinking.
However, this does mean we can just use the Lorentz transformation
directly: consider the stationary observer and the moving observer at
the same point at the same ``time''. They differ in reference frame only
by the Lorentz Transformation, thus photons emitted from one will differ
from the other by the Lorentz factor. The lorentz factor is
\(\frac{2}{\sqrt{3}}\) (regardless of if the particle is moving toward
or away from the system.) This transformation makes the adjustment of
\(\gamma(1-v)\) which evalutes to either \(\frac{1}{\sqrt{3}}\) or
\(\sqrt{3}\)

Multiplying this by \(\sqrt{2/3}\) does in fact produce the desired
range of \(\sqrt{2}/3\) to \(\sqrt{2}\).

    \emph{c) Compute the orbital period for the particle as measured by the
local static observer and by the observer at infinity. Hint: since
\(d\hat\phi = rd\phi\), the proper circumference of the orbit is simply
\(2\pi r\)}

    First of all we already know quite a bit about time dilation. So the
observer at infinity and the static observer have a difference of
\(\sqrt{3/2}\) Note that the dilation is inversed from part b) since it
was dealing with a FREQUENCY, we are now dealing with a PERIOD, which
varies inversely.

Hold on, we're being silly. The orbital circumference is 2 \(\pi\) r,
which is 12 \(\pi\) M. But since we're traveling at HALF light speed,
the resulting period is 24 \(\pi\) M. \emph{obviously.}

Which means the observed period at infinity is \$ 24\sqrt{3/2} \pi M \$

And there's our answer.

    \hypertarget{problem-13-back-to-top}{%
\section{\texorpdfstring{Problem 13 {[}Back to
\hyperref[toc]{top}{]}}{Problem 13 {[}Back to {]}}}\label{problem-13-back-to-top}}

\[\label{P13}\]

\emph{a) Show that the angular velocity as measured from infinity,
\(\Omega = d\phi / dt\), has the same form in the Schwarzchild geometry
as for circular orbits in Newtonian gravity, namely
\(\Omega = \sqrt{\frac{M}{r^3}}\)}

    Okay so it took quite a while to see this, BUT\ldots{}

\[ v^\hat\phi = \frac{d\hat\phi}{d\hat t} = \frac{rd\phi}{\sqrt{1-\frac{2M}{r}} dt} = \frac{r}{\sqrt{1-\frac{2M}{r}}} v^\phi \]

    Thus we have found the conversion factor from distant to local and vice
versa. Using the relation for local, we can get:

\[ v^\phi = \frac{\sqrt{1-\frac{2M}{r}}}{r} \sqrt{1-\frac{2M}{r}} \frac{\tilde l}{r \tilde E} \]

Which becomes

\[ = \frac{1}{r^2} (1-\frac{2M}{r}) \sqrt{\frac{Mr^2}{r-3M}} \sqrt{\frac{r(r-3M)}{(r-2M)^2}} \]
\[ = \frac{1}{r^3}  \sqrt{Mr^3} \] \[ = \sqrt{\frac{M}{r^3}} \]

Which is what we sought.

    \emph{b) Use this result to confirm the value of \(T_\infty\) found in
\textbf{Problem 12}.}

    Replace r with 6M, get \(\sqrt{\frac{1}{216M^2}}\) for the angular
frequency. This is presumably in radians per second, and we want
seconds, so flip it and multiply by \(2\pi\) to get

\[ T = 2\pi\sqrt{216M^2} = 2\pi M \sqrt{216} = 12 \pi M \sqrt{6} = 24 \pi M \sqrt{3/2} \]

Yes this is the relation we got before.

    \hypertarget{problem-14-back-to-top}{%
\section{\texorpdfstring{Problem 14 {[}Back to
\hyperref[toc]{top}{]}}{Problem 14 {[}Back to {]}}}\label{problem-14-back-to-top}}

\[\label{P14}\]

\emph{Show that an outwart-direacted photon emitted between r=2M and
r=3M escapes if}

\[ \sin \psi < \frac{3\sqrt{3} M}{r} \sqrt{1-\frac{2M}{r}} \]

    All this does is flip the inequality sign, very curious that the
relation is equally mirrored especially when the outer one technically
goes to infinite distance and this one just goes from 2M to 3M.

Let's think about it in reverse: the impact parameter of an outgoing
photon at the ``edge'' would be the minimum impact parameter for that.
No matter what b \textgreater{} \(3\sqrt{3}M\) is required. In this
case, we make the requirement of escape \(v^{\hat r} > 0\) AND b
\textgreater{} \(3\sqrt{3}M\). The difference in condition between this
and the ``inward'' version is the inequality flip on the sign of the
radial velocity. In practice this jsut means that we just have to
examine 12.5.14 more closely:

\[ sin\phi = v^\hat\phi = \frac{b}{r} \left( 1-\frac{2M}{r} \right)^{1/2} \]

Now the limit is where b \textgreater{} \(3\sqrt{3}M\)\$. The question
is, how can we show the sign's direction?

    The answer is we note that the angle \(\psi\) is beteen 0 and 90
degrees, not 90 and 180 degrees like for the inward version. This means
that the shape of the sine curve is \emph{reversed}, and thus the sign
goes the other way. We can be extra sure we're calibraitng it correctly
due to the 0-degree direction \emph{needing} to be the one that escapes,
and what do you know as r approaches 2M we do in fact approach zero
overall. So\ldots{}

\[ sin\phi < \frac{b}{r} \left( 1-\frac{2M}{r} \right)^{1/2} \]

In a sense we can consider the line r=3M to be a ``mirror'' of interior
and exterior solutions, with the ends of the various ``Sides'' being the
event horizon and the distance at infinity. What fun.

    \hypertarget{problem-15-back-to-top}{%
\section{\texorpdfstring{Problem 15 {[}Back to
\hyperref[toc]{top}{]}}{Problem 15 {[}Back to {]}}}\label{problem-15-back-to-top}}

\[\label{P15}\]

\emph{The angular momentum of the sun (assuming uniform rotation) is
1.63e48 \(g cm^2 / s\). What is a/M for the sun?}

This is just an exersise in unit conversion. Solar mass M is 1.476e3 in
c=G=1 units.

First, convert our angular momentum to metric, which means converting
square cm to m, which is an order of magnitude 4 adjustment to 1.63e44.
Then g to kg which is also down, order of 3, to 1.63e41. Now we're
cooking in \(kg m^2 / s\). Curiously this is the same exact units as the
constant \(\hbar\), presumably because it has something to do with spin.
Either way, the conversion factor is known: 2.476e-36 converts into
units of \(m^2.\)

So J is 4.036e5 \(m^2\).

Thus J/M = 273.4 = a.

And a/M is 0.185. Which is exactly the answer we're supposed to get.

Now we don't exactly know the sigificance of a/M which is just \(J/M^2\)
but perhaps it will become clear later.

Next page points out that if the sun were a black hole it would be a
naked singularity without an event horizon as its a \textless{} M.

    \hypertarget{problem-16-back-to-top}{%
\section{\texorpdfstring{Problem 16 {[}Back to
\hyperref[toc]{top}{]}}{Problem 16 {[}Back to {]}}}\label{problem-16-back-to-top}}

\[\label{P16}\]

\emph{Discuss the restriction 12.7.6 in the weak-field limit.}

    12.7.6 is \(\Omega_{min} < \Omega < \Omega_{max}\)

Now we can actually calculate these terms via 12.7.7

\[ \Omega_{ ^{min}_{max}} = \frac{-g_{t\phi} \pm \sqrt{g^2_{t\phi} - g_{tt}g_{\phi\phi}}}{g_{\phi\phi}}\]

    Now, yes, we are in the weak field limit, but we are also in the Kerr
metric. Assuming that ``weak field'' means that the \(\Sigma\) gets
arbitrarily large so the tt term goes to -1, this results in:

\[ \Omega_{ ^{min}_{max}} = \frac{ \pm \sqrt{ - (-1)(r^2+a^2)sin^2\theta}}{(r^2+a^2)sin^2\theta}\]

\[ = \frac{ \pm \sqrt{(r^2+a^2)}}{(r^2+a^2)sin\theta}\]

\[ = \frac{ \pm 1}{\sqrt{r^2+a^2}sin\theta}\]

    Which reduces to what we want if we let r \textgreater\textgreater{} a
which seems reasonable, as the angular momentum has an upper limit to
realism, r does not.

\[ \Omega_{ ^{min}_{max}} = \frac{ \pm 1}{r sin\theta}\]

Which would be multiplied by c to get the physical velocities, but let's
stay in c=1 units for now.

    Now this is not the ergosphere condition, we can in fact make a
situation where the angular velocity is zero. What this refers to is how
fast something must be rotating based on its azimuthal angle,
\(\theta\), for the observer ``stationary'' to the rotating black hole.
Notably we can't get a value any greater than c out of this which, well,
\emph{better} be true because otherwise everything would break, and
rahter spectacularly at that.

While this is not the ergosphere we can start to see the shape of the
ergosphere from this, and from the equation we used we can see that the
off-diagonal terms are actually rather important in defining it.

    \hypertarget{problem-17-back-to-top}{%
\section{\texorpdfstring{Problem 17 {[}Back to
\hyperref[toc]{top}{]}}{Problem 17 {[}Back to {]}}}\label{problem-17-back-to-top}}

\[\label{P17}\]

\emph{Show that Kepler's Third Law takes the form}

\[ \Omega = \pm \frac{\sqrt{M}}{r^{3/2} \pm a\sqrt{M}} \]

\emph{for circular equatorial orbits in the Kerr metric. Here
\(\Omega = d\phi / dt = \dot\phi / \dot{t}\)}

    So all we need to do is divide the given derivative forms from
12.7.12-13. Let's go ahead and arrange it all into:

\[ \Omega  = \frac{(r-2M)l+2aME}{r\Delta}\frac{r\Delta}{(r^3+a^2r+2Ma^2)E-2aMl}\]

\[ = \frac{(r-2M)l+2aME}{(r^3+a^2r+2Ma^2)E-2aMl}\]

    This is the general form. Now we need to specify for circular orbits.
This condition is given to us in 12.7.16 which, very luckily, we don't
have to solve and isntead can rely on the l and E values in 12.7.17-18.
Notably these are the ``1/m'' versions, but since we have an equal
number on top and bottom, that will cancel out. This substitution
provides:

\[ = \frac{(r-2M)\left( \pm \frac{\sqrt{Mr}(r^2 \mp 2a\sqrt{Mr}+a^2)}{r(\sqrt{r^2-3Mr \pm 2a\sqrt{Mr}})} \right)+2aM\left( \frac{r^2 - 2Mr \pm a\sqrt{Mr}}{r\sqrt{r^2-3Mr \pm 2a\sqrt{Mr}}} \right)}{(r^3+a^2r+2Ma^2)\left( \frac{r^2 - 2Mr \pm a\sqrt{Mr}}{r\sqrt{r^2-3Mr \pm 2a\sqrt{Mr}}} \right)-2aM\left( \pm \frac{\sqrt{Mr}(r^2 \mp 2a\sqrt{Mr}+a^2)}{r(\sqrt{r^2-3Mr \pm 2a\sqrt{Mr}})} \right)}\]

    Now that's a beast right there. However, \emph{all} the denominators are
the same, so they can just be removed.

\[ = \frac{(r-2M)\left( \pm \sqrt{Mr}(r^2 \mp 2a\sqrt{Mr}+a^2) \right)+2aM\left( r^2 - 2Mr \pm a\sqrt{Mr} \right)}{(r^3+a^2r+2Ma^2)\left( r^2 - 2Mr \pm a\sqrt{Mr} \right)-2aM\left( \pm \sqrt{Mr}(r^2 \mp 2a\sqrt{Mr}+a^2) \right)}\]

    Multiply \emph{everything} out, look for cancelations or common terms.

\[ = \frac{ \pm r\sqrt{Mr}(r^2 \mp 2a\sqrt{Mr}+a^2) \mp 2M\sqrt{Mr}(r^2 \mp 2a\sqrt{Mr}+a^2) + 2aMr^2 - 4aM^2r \pm 2a^2M\sqrt{Mr} }{r^5 + a^2r^3 + 2Ma^2r^2 - 2Mr^4 - 2Ma^2r^2 - 4M^2a^2r \pm r^3a\sqrt{Mr} \pm a^3r\sqrt{Mr} \pm 2Ma^3\sqrt{Mr} \mp 2aM\sqrt{Mr}(r^2 \mp 2a\sqrt{Mr}+a^2)}\]

Getting ugly but we still aren't done yet.

    \[ = \frac{ \pm r^3\sqrt{Mr} - 2aMr^2 \pm a^2r\sqrt{Mr} \mp 2Mr^2\sqrt{Mr} + 4aM^2r \mp 2Ma^2\sqrt{Mr} + 2aMr^2 - 4aM^2r \pm 2a^2M\sqrt{Mr} }{r^5 + a^2r^3 + 2Ma^2r^2 - 2Mr^4 - 2Ma^2r^2 - 4M^2a^2r \pm r^3a\sqrt{Mr} \pm a^3r\sqrt{Mr} \pm 2Ma^3\sqrt{Mr} \mp 2aMr^2\sqrt{Mr} + 4a^2M^2r + \mp 2a^3M\sqrt{Mr}}\]

And NOW we can look for hopefully obvious cancelations or common terms.

    \[ = \frac{ \pm r^3\sqrt{Mr} \pm a^2r\sqrt{Mr} \mp 2Mr^2\sqrt{Mr} }{r^5 + a^2r^3 - 2Mr^4 \pm r^3a\sqrt{Mr} \pm a^3r\sqrt{Mr} \mp 2aMr^2\sqrt{Mr}}\]

Good\ldots{} good. Now start pulling out terms of the goal. The
numerator should reveal what needs to cancel.

    \[ = \frac{ (\pm \sqrt{M})(r^3\sqrt{r} + a^2r\sqrt{r} - 2Mr^2\sqrt{r}) }{r^{3/2}(r^{7/2} + a^2r^{3/2} - 2Mr^{5/2}) \pm ( a\sqrt{M})(r^3\sqrt{r} + a^2r\sqrt{r} - 2Mr^2\sqrt{r})}\]

And would you look at that.

\[ = \frac{ \pm \sqrt{M} }{r^{3/2} \pm a\sqrt{M}}\]

Almost like magic.

    \hypertarget{problem-18-back-to-top}{%
\section{\texorpdfstring{Problem 18 {[}Back to
\hyperref[toc]{top}{]}}{Problem 18 {[}Back to {]}}}\label{problem-18-back-to-top}}

\[\label{P18}\]

\emph{Consider a particle with \(\tilde l=0\) released form rest far
from a Kerr black hole. Show that the particle ``Corotates with the
geometry'' as it spirals toward the hole along a conical surface of
constant \(\theta\). In other words, show that the particle aquires an
angular velocity \(d \phi / d t = \omega(r,\theta)\) as viewed from
infinity, where:}

\[ \omega(r,\theta) = \frac{2aMr}{(r^2+a^2)^2 - \Delta a^2sin^2\theta} \]

\emph{Note: observers at fixed r and \(\theta\) with zero angular
momentum also ``corotate with the geometry'' with angular velocity
\(\omega\). Such observers define the so-called ``locally nonrotating
frame'' (LNRF); according to such observers, the released particle
described above appears to move \textbf{radially} locally.}

    So in \textbf{Problem 17} we did this for circular orbits. However,
this, quite frankly, is NOT a circular orbit. It's a spiraling orbit.
However, what we know is that we're dropping something with angular
momentum zero at essentially infinity. Angular momentum is zero in this
case, and the Energy is just the rest mass m. Of course, since l=0, the
energy is just going to divide out.

\[ \omega = \frac{2aME}{r\Delta}\frac{r\Delta}{(r^3 + a^2r + 2Ma^2)E}\]
\[ = \frac{2aME}{(r^3 + a^2r + 2Ma^2)E}\]
\[ = \frac{2aM}{r^3 + a^2r + 2Ma^2}\]
\[ = \frac{2aMr}{r^4 + a^2r^2 + 2Ma^2r}\]
\[ = \frac{2aMr}{r^4 + 2a^2r^2 + a^4 - a^2r^2 - a^4 + 2Ma^2r}\]
\[ = \frac{2aMr}{(r^2+a^2)^2 - (a^2r^2 + a^4 - 2Ma^2r)}\]

    Okay so this is in the right \emph{FORM}. However, we don't have
\(\Delta a^2\) and our relation is only valid at \(\theta = \pi/2.\)
Clearly we need to find a way to show that in our particular case
\(\Delta a^2 sin^2\theta\) recuses to the expression we have in the
parentheses.

First of all, we note that \(\Delta\) is part of the Kerr metric:
\(r^2 - 2Mr + a^2\). So this immediately nets us:

\[ = \frac{2aMr}{(r^2+a^2)^2 - a^2\Delta}\]

Which is rather clearly the reduced form of what we're going for with
\(\theta = \pi/2\). The question is, how can we prove this is the case?
Where does the \(\theta\) dependence come from?

    We note that the denominator is entirely part of the \emph{time}
derivative. Which means it looks like we'll have to re-work 12.7.12 to
be nonambiguous. Which is going to be a little funky, but we do have the
relations 12.7.10-11 and the knowledge that l=0, which simplifies this
\emph{considerably}. To get new criteria, we craft the usual 2L
Lagrangian (a new version of 12.7.9) from the Kerr metric (12.7.1) and
apply them to the differential relations.

With these criteria, we get the equations:

\[ -\left( 1-\frac{2Mr}{\Sigma} \right) \dot t - \frac12 \frac{4aMrsin^2\theta}{\Sigma} \dot \phi = -E \]
\[ - \frac12 \frac{4aMrsin^2\theta}{\Sigma} \dot t + \left( r^2 + a^2 + \frac{2Mra^2sin^2\theta}{\Sigma} \right) sin^2\theta \dot \phi = 0 \]

That \emph{second} relation is actually even more useful than we could
have ever realized, for it allows us to calculate the ratio directly!

\[ \Rightarrow \frac12 \frac{4aMrsin^2\theta}{\Sigma} \dot t = \left( r^2 + a^2 + \frac{2Mra^2sin^2\theta}{\Sigma} \right) sin^2\theta \dot \phi \]

    \[ \Rightarrow \frac{\frac12 \frac{4aMrsin^2\theta}{\Sigma}}{\left( r^2 + a^2 + \frac{2Mra^2sin^2\theta}{\Sigma} \right) sin^2\theta} =  \frac{\dot \phi}{\dot t} \]

\[ \Rightarrow \frac{\frac12 \frac{4aMr}{\Sigma}}{\left( r^2 + a^2 + \frac{2Mra^2sin^2\theta}{\Sigma} \right) } =  \frac{\dot \phi}{\dot t} \]

\[ \Rightarrow \frac{2aMr}{r^2\Sigma + a^2\Sigma + 2Mra^2sin^2\theta } =  \frac{\dot \phi}{\dot t} \]

\(\Sigma\) is the Kerr value of \(r^2+a^2cos^2\theta\). Which is to say,
exactly what we need it to be\ldots{}

    \[ \Rightarrow \frac{2aMr}{r^2(r^2+a^2cos^2\theta)+ a^2(r^2+a^2cos^2\theta) + 2Mra^2sin^2\theta } =  \frac{\dot \phi}{\dot t} \]

\[ \Rightarrow \frac{2aMr}{r^4+a^2r^2cos^2\theta+ a^2r^2+a^4cos^2\theta + 2Mra^2sin^2\theta } =  \frac{\dot \phi}{\dot t} \]

\[ \Rightarrow \frac{2aMr}{r^4+a^2r^2+a^2r^2sin^2\theta+ a^2r^2+a^4+a^4sin^2\theta + 2Mra^2sin^2\theta } =  \frac{\dot \phi}{\dot t} \]

\[ \Rightarrow \frac{2aMr}{(r^2+a^2)^2 + a^2r^2sin^2\theta +a^4sin^2\theta + 2Mra^2sin^2\theta } =  \frac{\dot \phi}{\dot t} \]

\[ \Rightarrow \frac{2aMr}{(r^2+a^2)^2 + a^2\Delta sin^2\theta } =  \frac{\dot \phi}{\dot t} \]

Aha! BINGO!

    \hypertarget{problem-19-back-to-top}{%
\section{\texorpdfstring{Problem 19 {[}Back to
\hyperref[toc]{top}{]}}{Problem 19 {[}Back to {]}}}\label{problem-19-back-to-top}}

\[\label{P19}\]

\emph{Using Hawking's area theorem to find the minimum mass M\_2 of a
Schwarzchild black hole that results from the collision of two Kerr
black holes of equal mass M and opposite angular momentum parameter a.
Show that if \textbar a\textbar{} -\textgreater{} M, 50\% of the rest
mass is allowed to be radiated away. Show that no other combinations of
masses and angular momenta lead to higher possible efficienceis. Show
that if a=0, the maximum efficiency is 29\%.}

\emph{Note: the \textbf{actual} amount of radiation generated by such a
collision is amenable to numerical computation. The result is not yet
known for the general case, but for a=0 it is \textasciitilde0.1\%.}

    Hawking'sradiation theorem is

\[ A = 8\pi M \left[ M + \sqrt{M^2-a^2} \right] \]

And we also know that surface area is CONSERVED, that is, it can never
decrease, even in black hole interactions. So, while it is possible to
get \emph{more} surface area (entropy-like behavior), it's not possible
to get less.

\[ A_0+A_1 = 8\pi M_2 \left[ M_2 + \sqrt{M_2^2-a_2^2} \right] \]
\[ \Rightarrow 8\pi M_0 \left[ M_0 + \sqrt{M_0^2-a_0^2} \right]+8\pi M_1 \left[ M_1 + \sqrt{M_1^2-a_1^2} \right] = 8\pi M_2 \left[ M_2 + \sqrt{M_2^2-a_2^2} \right] \]

Thus let's just note that the original surface area is 2A, and that the
magnitude of the original black hole a's match, as do their Ms.

\[ \Rightarrow 16\pi M \left[ M + \sqrt{M^2-a^2} \right] = 8\pi M_2 \left[ M_2 + \sqrt{M_2^2-a_2^2} \right] \]
\[ \Rightarrow 2 M \left[ M + \sqrt{M^2-a^2} \right] = M_2 \left[ M_2 + \sqrt{M_2^2-a_2^2} \right] \]

Now we note that the angular momentum of the resulting black hole has to
be zero since they are opposite angular momentum! Conservation of
angular momentum and all.

\[ \Rightarrow 2 M \left[ M + \sqrt{M^2-a^2} \right] = M_2 \left[ M_2 + \sqrt{M_2^2} \right] \]
\[ \Rightarrow 2 M \left[ M + \sqrt{M^2-a^2} \right] = 2M_2^2 \]

    The first case we are asked to solve for is if \textbar a\textbar=M,
which results in the cancelation of the square root and\ldots{}

\[ 2 M \left[ M + \sqrt{M^2-M^2} \right] = 2M_2^2 \]
\[ \Rightarrow 2 M^2 = 2M_2^2 \] \[ \Rightarrow M = M_2 \]

Think about this. \emph{This means that the resulting black hole only
has the mass of one of the original ones.} Shocking. Granted this
celarly only applies for really ridiculously high a, but wow, that's
sure something.

It is rather obvious to see that adjusting a (or M) will not increase
this efficiency any more: any adjustment makes \(M^2-a^2 > 0\) and thus
adds minimum mass to the final relation, meaning less than 50\% mass is
radiated by the area theorem. (Keep in mind, at a minimum.)

    Now the other case is a=0.

\[ 2 M \left[ M + \sqrt{M^2} \right] = 2M_2^2 \]
\[ 2 M ( 2M) = 2M_2^2 \] \[ 4 M^2 = 2M_2^2 \] \[ \sqrt{2}M = M_2 \]

Now the exact percentage of this relation is uncertain, but we start
with \(2M\) and end with \(\sqrt{2}M\) which is a radiation of about
29.3\%. Which is what we were supposed to get.

    \hypertarget{problem-20-back-to-top}{%
\section{\texorpdfstring{Problem 20 {[}Back to
\hyperref[toc]{top}{]}}{Problem 20 {[}Back to {]}}}\label{problem-20-back-to-top}}

\[\label{P20}\]

\emph{verify the numerical relations in 12.8.6 and 12.8.7}

12.8.6: \(T = \frac{\hbar}{8\pi k M} \approx 1e-7 K \frac{M_\odot}{M}\)

12.8.7: \$ \sqrt{\hbar} = 2.2e-5 g = 1.6e-33 cm \$

    The only variable in 12.8.6 is M, and we can just pull that out. Yoink.
Also K is just Kelvin, so let's ignore it.

\[ \frac{\hbar}{8\pi k} \approx 1e-7 M_\odot\]

Now what, exactly is the value of k? Botlzmann's constant? (definitely
not Coulomb's constant, that has charge). B's constant has 1.381e-23
J/K, or \(kgm^2/Ks^2\). This is defintely not in nice units, so let's
change it. The Joule conversion factor works just fine here. \ldots So
let's just look up help in \hyperref[3]{3}. The multiplication factor is
\(G/c^4\) which has a value of 8.261e-45. This alters k to 1.141e-67
m/K.

This gives us\ldots{}

\[ 9.110e-5 \approx 1e-7 M_\odot\]

Divide out the sun\ldots{}

\[ 6.172e-8 \approx 1e-7 \]

Yeah, this checks out.

    So, what is the actual temperature of a black hole in units we can
understand? K is not adjusted by our c=G=1 formulation that we have been
working with, so it's still the same. So let's just consider a stellar
mass black hole which seems to be around 5 suns in general, which gives
us\ldots{} 2e-8 K.

This is ridiculously small.

For the record the temperature of actual empty space in general is 2K.
yes. 2. Without any exponential.

Laboratory conditions have achieved 2.8e-10 though. However,
supermassive black holes can clearly get colder than this, with a
one-million solar mass black hole having a temperature of 1e-13 K.

    12.8.7 is trivial. Grab the value of \(\hbar\) where G=c=1 and take the
root. 2.612e-70 \(m^2\) = 1.616e-35 m and thus about 1.6ee-33 cm.

The conversion factor between m and kg is 7.425e-28 m/kg, so dividing by
this should net us the other result. And whaddoyaknow, 2.18e-5 g or
2.18e-8 kg. Please note that we prefer to use units of m in c=G=1, but
both are valid.

    \hypertarget{problem-21-back-to-top}{%
\section{\texorpdfstring{Problem 21 {[}Back to
\hyperref[toc]{top}{]}}{Problem 21 {[}Back to {]}}}\label{problem-21-back-to-top}}

\[\label{P21}\]

\emph{a) Compute the entropy of a 1 \(M_\odot\) black hole in units of
k, Boltzmann's constant.}

Oh hey how convenient NOW they define k.

AHEM. Note. Entropy. Not. Temperature.

anyway \(S = k\frac{c^3A}{4G\hbar}\) from 12.8.10. The only value we
don't automatically know is A. However, it is determinable by 12.8.8,
\(A = 4\pi(\frac{2GM}{c^2})^2\). This alters our expression to:

\[ S = k\frac{4 \pi G M^2}{\hbar c} \]

    Let's not worry about the math and just do the multiplications,
everything is in SI units.

The reasult is 1.0481e77 in units of J/K. Which is what k is, so really
1e77 is dimensionless.

S=1e77k.

    \emph{b) Estimate the entropy of the sun. Assume it consists of
completely ionized hyudrogen with a mean density of 1 g/\(cm^3\) and a
mean temperature of 1e6 K}

    So, we didn't do all the equation of state and thermodynamics chapters,
so we do not have the skills to actually perform this estimation within
a reasonable amount of time (and would likely want to refer to our
Statistical Mechanics text to boot.) Thus, we skip both this and the
following segment.

However, we can compare entropies. The answer for this part is given at
2e58 k for the sun. Trying to find entropies for neutron stars turned up
flat but we can assume it's between the order 58 and 77. Which,
admittedly is a \emph{very} large range.

Naturally the sun has a lot of entropy, but black holes have OODLES
more.

    \emph{c) Estimate the entropy of a 1\(M_\odot\) iron white dwarf with a
1\(M_\odot\) neutron star. Take the mean tempereature to be 1e8 K, and
the mean densities to be 1e6 \(g/cm^3\) and 1e14 \(g/cm^3\),
respectively. Note that the expression 11.8.1 for \(C_\nu\) for a
degenerate ideal gas is also equal to S, since \(C_\nu = T dS / dT\).}

    Skipped.

    \hypertarget{addendum-output-this-notebook-to-latex-formatted-pdf-file-back-to-top}{%
\section{\texorpdfstring{Addendum: Output this notebook to
\(\LaTeX\)-formatted PDF file {[}Back to
\hyperref[toc]{top}{]}}{Addendum: Output this notebook to \textbackslash LaTeX-formatted PDF file {[}Back to {]}}}\label{addendum-output-this-notebook-to-latex-formatted-pdf-file-back-to-top}}

\[\label{latex_pdf_output}\]

The following code cell converts this Jupyter notebook into a proper,
clickable \(\LaTeX\)-formatted PDF file. After the cell is successfully
run, the generated PDF may be found in the root NRPy+ tutorial
directory, with filename \url{CO-12.pdf} (Note that clicking on this
link may not work; you may need to open the PDF file through another
means.)

\textbf{Important Note}: Make sure that the file name is right in all
six locations, two here in the Markdown, four in the code below.

\begin{itemize}
\tightlist
\item
  CO-12.pdf
\item
  CO-12.ipynb
\item
  CO-12.tex
\end{itemize}

    \begin{tcolorbox}[breakable, size=fbox, boxrule=1pt, pad at break*=1mm,colback=cellbackground, colframe=cellborder]
\prompt{In}{incolor}{1}{\boxspacing}
\begin{Verbatim}[commandchars=\\\{\}]
\PY{k+kn}{import} \PY{n+nn}{cmdline\PYZus{}helper} \PY{k}{as} \PY{n+nn}{cmd}    \PY{c+c1}{\PYZsh{} NRPy+: Multi\PYZhy{}platform Python command\PYZhy{}line interface}
\PY{n}{cmd}\PY{o}{.}\PY{n}{output\PYZus{}Jupyter\PYZus{}notebook\PYZus{}to\PYZus{}LaTeXed\PYZus{}PDF}\PY{p}{(}\PY{l+s+s2}{\PYZdq{}}\PY{l+s+s2}{CO\PYZhy{}12}\PY{l+s+s2}{\PYZdq{}}\PY{p}{)}
\end{Verbatim}
\end{tcolorbox}

    \begin{Verbatim}[commandchars=\\\{\}]
Created GR-08.tex, and compiled LaTeX file to PDF file GR-08.pdf
    \end{Verbatim}

    \begin{tcolorbox}[breakable, size=fbox, boxrule=1pt, pad at break*=1mm,colback=cellbackground, colframe=cellborder]
\prompt{In}{incolor}{ }{\boxspacing}
\begin{Verbatim}[commandchars=\\\{\}]

\end{Verbatim}
\end{tcolorbox}

    \begin{tcolorbox}[breakable, size=fbox, boxrule=1pt, pad at break*=1mm,colback=cellbackground, colframe=cellborder]
\prompt{In}{incolor}{ }{\boxspacing}
\begin{Verbatim}[commandchars=\\\{\}]

\end{Verbatim}
\end{tcolorbox}


    % Add a bibliography block to the postdoc
    
    
    
\end{document}
