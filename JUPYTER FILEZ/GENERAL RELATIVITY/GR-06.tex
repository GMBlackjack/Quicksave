% Based on http://nbviewer.jupyter.org/github/ipython/nbconvert-examples/blob/master/citations/Tutorial.ipynb , authored by Brian E. Granger
    % Declare the document class
    \documentclass[landscape,letterpaper,10pt,english]{article}


    \usepackage[breakable]{tcolorbox}
    \usepackage{parskip} % Stop auto-indenting (to mimic markdown behaviour)
    

    % Basic figure setup, for now with no caption control since it's done
    % automatically by Pandoc (which extracts ![](path) syntax from Markdown).
    \usepackage{graphicx}
    % Maintain compatibility with old templates. Remove in nbconvert 6.0
    \let\Oldincludegraphics\includegraphics
    % Ensure that by default, figures have no caption (until we provide a
    % proper Figure object with a Caption API and a way to capture that
    % in the conversion process - todo).
    \usepackage{caption}
    \DeclareCaptionFormat{nocaption}{}
    \captionsetup{format=nocaption,aboveskip=0pt,belowskip=0pt}

    \usepackage{float}
    \floatplacement{figure}{H} % forces figures to be placed at the correct location
    \usepackage{xcolor} % Allow colors to be defined
    \usepackage{enumerate} % Needed for markdown enumerations to work
    \usepackage{geometry} % Used to adjust the document margins
    \usepackage{amsmath} % Equations
    \usepackage{amssymb} % Equations
    \usepackage{textcomp} % defines textquotesingle
    % Hack from http://tex.stackexchange.com/a/47451/13684:
    \AtBeginDocument{%
        \def\PYZsq{\textquotesingle}% Upright quotes in Pygmentized code
    }
    \usepackage{upquote} % Upright quotes for verbatim code
    \usepackage{eurosym} % defines \euro

    \usepackage{iftex}
    \ifPDFTeX
        \usepackage[T1]{fontenc}
        \IfFileExists{alphabeta.sty}{
              \usepackage{alphabeta}
          }{
              \usepackage[mathletters]{ucs}
              \usepackage[utf8x]{inputenc}
          }
    \else
        \usepackage{fontspec}
        \usepackage{unicode-math}
    \fi

    \usepackage{fancyvrb} % verbatim replacement that allows latex
    \usepackage{grffile} % extends the file name processing of package graphics
                         % to support a larger range
    \makeatletter % fix for old versions of grffile with XeLaTeX
    \@ifpackagelater{grffile}{2019/11/01}
    {
      % Do nothing on new versions
    }
    {
      \def\Gread@@xetex#1{%
        \IfFileExists{"\Gin@base".bb}%
        {\Gread@eps{\Gin@base.bb}}%
        {\Gread@@xetex@aux#1}%
      }
    }
    \makeatother
    \usepackage[Export]{adjustbox} % Used to constrain images to a maximum size
    \adjustboxset{max size={0.9\linewidth}{0.9\paperheight}}

    % The hyperref package gives us a pdf with properly built
    % internal navigation ('pdf bookmarks' for the table of contents,
    % internal cross-reference links, web links for URLs, etc.)
    \usepackage{hyperref}
    % The default LaTeX title has an obnoxious amount of whitespace. By default,
    % titling removes some of it. It also provides customization options.
    \usepackage{titling}
    \usepackage{longtable} % longtable support required by pandoc >1.10
    \usepackage{booktabs}  % table support for pandoc > 1.12.2
    \usepackage{array}     % table support for pandoc >= 2.11.3
    \usepackage{calc}      % table minipage width calculation for pandoc >= 2.11.1
    \usepackage[inline]{enumitem} % IRkernel/repr support (it uses the enumerate* environment)
    \usepackage[normalem]{ulem} % ulem is needed to support strikethroughs (\sout)
                                % normalem makes italics be italics, not underlines
    \usepackage{mathrsfs}
    

    
    % Colors for the hyperref package
    \definecolor{urlcolor}{rgb}{0,.145,.698}
    \definecolor{linkcolor}{rgb}{.71,0.21,0.01}
    \definecolor{citecolor}{rgb}{.12,.54,.11}

    % ANSI colors
    \definecolor{ansi-black}{HTML}{3E424D}
    \definecolor{ansi-black-intense}{HTML}{282C36}
    \definecolor{ansi-red}{HTML}{E75C58}
    \definecolor{ansi-red-intense}{HTML}{B22B31}
    \definecolor{ansi-green}{HTML}{00A250}
    \definecolor{ansi-green-intense}{HTML}{007427}
    \definecolor{ansi-yellow}{HTML}{DDB62B}
    \definecolor{ansi-yellow-intense}{HTML}{B27D12}
    \definecolor{ansi-blue}{HTML}{208FFB}
    \definecolor{ansi-blue-intense}{HTML}{0065CA}
    \definecolor{ansi-magenta}{HTML}{D160C4}
    \definecolor{ansi-magenta-intense}{HTML}{A03196}
    \definecolor{ansi-cyan}{HTML}{60C6C8}
    \definecolor{ansi-cyan-intense}{HTML}{258F8F}
    \definecolor{ansi-white}{HTML}{C5C1B4}
    \definecolor{ansi-white-intense}{HTML}{A1A6B2}
    \definecolor{ansi-default-inverse-fg}{HTML}{FFFFFF}
    \definecolor{ansi-default-inverse-bg}{HTML}{000000}

    % common color for the border for error outputs.
    \definecolor{outerrorbackground}{HTML}{FFDFDF}

    % commands and environments needed by pandoc snippets
    % extracted from the output of `pandoc -s`
    \providecommand{\tightlist}{%
      \setlength{\itemsep}{0pt}\setlength{\parskip}{0pt}}
    \DefineVerbatimEnvironment{Highlighting}{Verbatim}{commandchars=\\\{\}}
    % Add ',fontsize=\small' for more characters per line
    \newenvironment{Shaded}{}{}
    \newcommand{\KeywordTok}[1]{\textcolor[rgb]{0.00,0.44,0.13}{\textbf{{#1}}}}
    \newcommand{\DataTypeTok}[1]{\textcolor[rgb]{0.56,0.13,0.00}{{#1}}}
    \newcommand{\DecValTok}[1]{\textcolor[rgb]{0.25,0.63,0.44}{{#1}}}
    \newcommand{\BaseNTok}[1]{\textcolor[rgb]{0.25,0.63,0.44}{{#1}}}
    \newcommand{\FloatTok}[1]{\textcolor[rgb]{0.25,0.63,0.44}{{#1}}}
    \newcommand{\CharTok}[1]{\textcolor[rgb]{0.25,0.44,0.63}{{#1}}}
    \newcommand{\StringTok}[1]{\textcolor[rgb]{0.25,0.44,0.63}{{#1}}}
    \newcommand{\CommentTok}[1]{\textcolor[rgb]{0.38,0.63,0.69}{\textit{{#1}}}}
    \newcommand{\OtherTok}[1]{\textcolor[rgb]{0.00,0.44,0.13}{{#1}}}
    \newcommand{\AlertTok}[1]{\textcolor[rgb]{1.00,0.00,0.00}{\textbf{{#1}}}}
    \newcommand{\FunctionTok}[1]{\textcolor[rgb]{0.02,0.16,0.49}{{#1}}}
    \newcommand{\RegionMarkerTok}[1]{{#1}}
    \newcommand{\ErrorTok}[1]{\textcolor[rgb]{1.00,0.00,0.00}{\textbf{{#1}}}}
    \newcommand{\NormalTok}[1]{{#1}}

    % Additional commands for more recent versions of Pandoc
    \newcommand{\ConstantTok}[1]{\textcolor[rgb]{0.53,0.00,0.00}{{#1}}}
    \newcommand{\SpecialCharTok}[1]{\textcolor[rgb]{0.25,0.44,0.63}{{#1}}}
    \newcommand{\VerbatimStringTok}[1]{\textcolor[rgb]{0.25,0.44,0.63}{{#1}}}
    \newcommand{\SpecialStringTok}[1]{\textcolor[rgb]{0.73,0.40,0.53}{{#1}}}
    \newcommand{\ImportTok}[1]{{#1}}
    \newcommand{\DocumentationTok}[1]{\textcolor[rgb]{0.73,0.13,0.13}{\textit{{#1}}}}
    \newcommand{\AnnotationTok}[1]{\textcolor[rgb]{0.38,0.63,0.69}{\textbf{\textit{{#1}}}}}
    \newcommand{\CommentVarTok}[1]{\textcolor[rgb]{0.38,0.63,0.69}{\textbf{\textit{{#1}}}}}
    \newcommand{\VariableTok}[1]{\textcolor[rgb]{0.10,0.09,0.49}{{#1}}}
    \newcommand{\ControlFlowTok}[1]{\textcolor[rgb]{0.00,0.44,0.13}{\textbf{{#1}}}}
    \newcommand{\OperatorTok}[1]{\textcolor[rgb]{0.40,0.40,0.40}{{#1}}}
    \newcommand{\BuiltInTok}[1]{{#1}}
    \newcommand{\ExtensionTok}[1]{{#1}}
    \newcommand{\PreprocessorTok}[1]{\textcolor[rgb]{0.74,0.48,0.00}{{#1}}}
    \newcommand{\AttributeTok}[1]{\textcolor[rgb]{0.49,0.56,0.16}{{#1}}}
    \newcommand{\InformationTok}[1]{\textcolor[rgb]{0.38,0.63,0.69}{\textbf{\textit{{#1}}}}}
    \newcommand{\WarningTok}[1]{\textcolor[rgb]{0.38,0.63,0.69}{\textbf{\textit{{#1}}}}}


    % Define a nice break command that doesn't care if a line doesn't already
    % exist.
    \def\br{\hspace*{\fill} \\* }
    % Math Jax compatibility definitions
    \def\gt{>}
    \def\lt{<}
    \let\Oldtex\TeX
    \let\Oldlatex\LaTeX
    \renewcommand{\TeX}{\textrm{\Oldtex}}
    \renewcommand{\LaTeX}{\textrm{\Oldlatex}}
    % Document parameters
    % Document title
    \title{GR-06}
    
    
    
    
    
% Pygments definitions
\makeatletter
\def\PY@reset{\let\PY@it=\relax \let\PY@bf=\relax%
    \let\PY@ul=\relax \let\PY@tc=\relax%
    \let\PY@bc=\relax \let\PY@ff=\relax}
\def\PY@tok#1{\csname PY@tok@#1\endcsname}
\def\PY@toks#1+{\ifx\relax#1\empty\else%
    \PY@tok{#1}\expandafter\PY@toks\fi}
\def\PY@do#1{\PY@bc{\PY@tc{\PY@ul{%
    \PY@it{\PY@bf{\PY@ff{#1}}}}}}}
\def\PY#1#2{\PY@reset\PY@toks#1+\relax+\PY@do{#2}}

\@namedef{PY@tok@w}{\def\PY@tc##1{\textcolor[rgb]{0.73,0.73,0.73}{##1}}}
\@namedef{PY@tok@c}{\let\PY@it=\textit\def\PY@tc##1{\textcolor[rgb]{0.24,0.48,0.48}{##1}}}
\@namedef{PY@tok@cp}{\def\PY@tc##1{\textcolor[rgb]{0.61,0.40,0.00}{##1}}}
\@namedef{PY@tok@k}{\let\PY@bf=\textbf\def\PY@tc##1{\textcolor[rgb]{0.00,0.50,0.00}{##1}}}
\@namedef{PY@tok@kp}{\def\PY@tc##1{\textcolor[rgb]{0.00,0.50,0.00}{##1}}}
\@namedef{PY@tok@kt}{\def\PY@tc##1{\textcolor[rgb]{0.69,0.00,0.25}{##1}}}
\@namedef{PY@tok@o}{\def\PY@tc##1{\textcolor[rgb]{0.40,0.40,0.40}{##1}}}
\@namedef{PY@tok@ow}{\let\PY@bf=\textbf\def\PY@tc##1{\textcolor[rgb]{0.67,0.13,1.00}{##1}}}
\@namedef{PY@tok@nb}{\def\PY@tc##1{\textcolor[rgb]{0.00,0.50,0.00}{##1}}}
\@namedef{PY@tok@nf}{\def\PY@tc##1{\textcolor[rgb]{0.00,0.00,1.00}{##1}}}
\@namedef{PY@tok@nc}{\let\PY@bf=\textbf\def\PY@tc##1{\textcolor[rgb]{0.00,0.00,1.00}{##1}}}
\@namedef{PY@tok@nn}{\let\PY@bf=\textbf\def\PY@tc##1{\textcolor[rgb]{0.00,0.00,1.00}{##1}}}
\@namedef{PY@tok@ne}{\let\PY@bf=\textbf\def\PY@tc##1{\textcolor[rgb]{0.80,0.25,0.22}{##1}}}
\@namedef{PY@tok@nv}{\def\PY@tc##1{\textcolor[rgb]{0.10,0.09,0.49}{##1}}}
\@namedef{PY@tok@no}{\def\PY@tc##1{\textcolor[rgb]{0.53,0.00,0.00}{##1}}}
\@namedef{PY@tok@nl}{\def\PY@tc##1{\textcolor[rgb]{0.46,0.46,0.00}{##1}}}
\@namedef{PY@tok@ni}{\let\PY@bf=\textbf\def\PY@tc##1{\textcolor[rgb]{0.44,0.44,0.44}{##1}}}
\@namedef{PY@tok@na}{\def\PY@tc##1{\textcolor[rgb]{0.41,0.47,0.13}{##1}}}
\@namedef{PY@tok@nt}{\let\PY@bf=\textbf\def\PY@tc##1{\textcolor[rgb]{0.00,0.50,0.00}{##1}}}
\@namedef{PY@tok@nd}{\def\PY@tc##1{\textcolor[rgb]{0.67,0.13,1.00}{##1}}}
\@namedef{PY@tok@s}{\def\PY@tc##1{\textcolor[rgb]{0.73,0.13,0.13}{##1}}}
\@namedef{PY@tok@sd}{\let\PY@it=\textit\def\PY@tc##1{\textcolor[rgb]{0.73,0.13,0.13}{##1}}}
\@namedef{PY@tok@si}{\let\PY@bf=\textbf\def\PY@tc##1{\textcolor[rgb]{0.64,0.35,0.47}{##1}}}
\@namedef{PY@tok@se}{\let\PY@bf=\textbf\def\PY@tc##1{\textcolor[rgb]{0.67,0.36,0.12}{##1}}}
\@namedef{PY@tok@sr}{\def\PY@tc##1{\textcolor[rgb]{0.64,0.35,0.47}{##1}}}
\@namedef{PY@tok@ss}{\def\PY@tc##1{\textcolor[rgb]{0.10,0.09,0.49}{##1}}}
\@namedef{PY@tok@sx}{\def\PY@tc##1{\textcolor[rgb]{0.00,0.50,0.00}{##1}}}
\@namedef{PY@tok@m}{\def\PY@tc##1{\textcolor[rgb]{0.40,0.40,0.40}{##1}}}
\@namedef{PY@tok@gh}{\let\PY@bf=\textbf\def\PY@tc##1{\textcolor[rgb]{0.00,0.00,0.50}{##1}}}
\@namedef{PY@tok@gu}{\let\PY@bf=\textbf\def\PY@tc##1{\textcolor[rgb]{0.50,0.00,0.50}{##1}}}
\@namedef{PY@tok@gd}{\def\PY@tc##1{\textcolor[rgb]{0.63,0.00,0.00}{##1}}}
\@namedef{PY@tok@gi}{\def\PY@tc##1{\textcolor[rgb]{0.00,0.52,0.00}{##1}}}
\@namedef{PY@tok@gr}{\def\PY@tc##1{\textcolor[rgb]{0.89,0.00,0.00}{##1}}}
\@namedef{PY@tok@ge}{\let\PY@it=\textit}
\@namedef{PY@tok@gs}{\let\PY@bf=\textbf}
\@namedef{PY@tok@gp}{\let\PY@bf=\textbf\def\PY@tc##1{\textcolor[rgb]{0.00,0.00,0.50}{##1}}}
\@namedef{PY@tok@go}{\def\PY@tc##1{\textcolor[rgb]{0.44,0.44,0.44}{##1}}}
\@namedef{PY@tok@gt}{\def\PY@tc##1{\textcolor[rgb]{0.00,0.27,0.87}{##1}}}
\@namedef{PY@tok@err}{\def\PY@bc##1{{\setlength{\fboxsep}{\string -\fboxrule}\fcolorbox[rgb]{1.00,0.00,0.00}{1,1,1}{\strut ##1}}}}
\@namedef{PY@tok@kc}{\let\PY@bf=\textbf\def\PY@tc##1{\textcolor[rgb]{0.00,0.50,0.00}{##1}}}
\@namedef{PY@tok@kd}{\let\PY@bf=\textbf\def\PY@tc##1{\textcolor[rgb]{0.00,0.50,0.00}{##1}}}
\@namedef{PY@tok@kn}{\let\PY@bf=\textbf\def\PY@tc##1{\textcolor[rgb]{0.00,0.50,0.00}{##1}}}
\@namedef{PY@tok@kr}{\let\PY@bf=\textbf\def\PY@tc##1{\textcolor[rgb]{0.00,0.50,0.00}{##1}}}
\@namedef{PY@tok@bp}{\def\PY@tc##1{\textcolor[rgb]{0.00,0.50,0.00}{##1}}}
\@namedef{PY@tok@fm}{\def\PY@tc##1{\textcolor[rgb]{0.00,0.00,1.00}{##1}}}
\@namedef{PY@tok@vc}{\def\PY@tc##1{\textcolor[rgb]{0.10,0.09,0.49}{##1}}}
\@namedef{PY@tok@vg}{\def\PY@tc##1{\textcolor[rgb]{0.10,0.09,0.49}{##1}}}
\@namedef{PY@tok@vi}{\def\PY@tc##1{\textcolor[rgb]{0.10,0.09,0.49}{##1}}}
\@namedef{PY@tok@vm}{\def\PY@tc##1{\textcolor[rgb]{0.10,0.09,0.49}{##1}}}
\@namedef{PY@tok@sa}{\def\PY@tc##1{\textcolor[rgb]{0.73,0.13,0.13}{##1}}}
\@namedef{PY@tok@sb}{\def\PY@tc##1{\textcolor[rgb]{0.73,0.13,0.13}{##1}}}
\@namedef{PY@tok@sc}{\def\PY@tc##1{\textcolor[rgb]{0.73,0.13,0.13}{##1}}}
\@namedef{PY@tok@dl}{\def\PY@tc##1{\textcolor[rgb]{0.73,0.13,0.13}{##1}}}
\@namedef{PY@tok@s2}{\def\PY@tc##1{\textcolor[rgb]{0.73,0.13,0.13}{##1}}}
\@namedef{PY@tok@sh}{\def\PY@tc##1{\textcolor[rgb]{0.73,0.13,0.13}{##1}}}
\@namedef{PY@tok@s1}{\def\PY@tc##1{\textcolor[rgb]{0.73,0.13,0.13}{##1}}}
\@namedef{PY@tok@mb}{\def\PY@tc##1{\textcolor[rgb]{0.40,0.40,0.40}{##1}}}
\@namedef{PY@tok@mf}{\def\PY@tc##1{\textcolor[rgb]{0.40,0.40,0.40}{##1}}}
\@namedef{PY@tok@mh}{\def\PY@tc##1{\textcolor[rgb]{0.40,0.40,0.40}{##1}}}
\@namedef{PY@tok@mi}{\def\PY@tc##1{\textcolor[rgb]{0.40,0.40,0.40}{##1}}}
\@namedef{PY@tok@il}{\def\PY@tc##1{\textcolor[rgb]{0.40,0.40,0.40}{##1}}}
\@namedef{PY@tok@mo}{\def\PY@tc##1{\textcolor[rgb]{0.40,0.40,0.40}{##1}}}
\@namedef{PY@tok@ch}{\let\PY@it=\textit\def\PY@tc##1{\textcolor[rgb]{0.24,0.48,0.48}{##1}}}
\@namedef{PY@tok@cm}{\let\PY@it=\textit\def\PY@tc##1{\textcolor[rgb]{0.24,0.48,0.48}{##1}}}
\@namedef{PY@tok@cpf}{\let\PY@it=\textit\def\PY@tc##1{\textcolor[rgb]{0.24,0.48,0.48}{##1}}}
\@namedef{PY@tok@c1}{\let\PY@it=\textit\def\PY@tc##1{\textcolor[rgb]{0.24,0.48,0.48}{##1}}}
\@namedef{PY@tok@cs}{\let\PY@it=\textit\def\PY@tc##1{\textcolor[rgb]{0.24,0.48,0.48}{##1}}}

\def\PYZbs{\char`\\}
\def\PYZus{\char`\_}
\def\PYZob{\char`\{}
\def\PYZcb{\char`\}}
\def\PYZca{\char`\^}
\def\PYZam{\char`\&}
\def\PYZlt{\char`\<}
\def\PYZgt{\char`\>}
\def\PYZsh{\char`\#}
\def\PYZpc{\char`\%}
\def\PYZdl{\char`\$}
\def\PYZhy{\char`\-}
\def\PYZsq{\char`\'}
\def\PYZdq{\char`\"}
\def\PYZti{\char`\~}
% for compatibility with earlier versions
\def\PYZat{@}
\def\PYZlb{[}
\def\PYZrb{]}
\makeatother


    % For linebreaks inside Verbatim environment from package fancyvrb.
    \makeatletter
        \newbox\Wrappedcontinuationbox
        \newbox\Wrappedvisiblespacebox
        \newcommand*\Wrappedvisiblespace {\textcolor{red}{\textvisiblespace}}
        \newcommand*\Wrappedcontinuationsymbol {\textcolor{red}{\llap{\tiny$\m@th\hookrightarrow$}}}
        \newcommand*\Wrappedcontinuationindent {3ex }
        \newcommand*\Wrappedafterbreak {\kern\Wrappedcontinuationindent\copy\Wrappedcontinuationbox}
        % Take advantage of the already applied Pygments mark-up to insert
        % potential linebreaks for TeX processing.
        %        {, <, #, %, $, ' and ": go to next line.
        %        _, }, ^, &, >, - and ~: stay at end of broken line.
        % Use of \textquotesingle for straight quote.
        \newcommand*\Wrappedbreaksatspecials {%
            \def\PYGZus{\discretionary{\char`\_}{\Wrappedafterbreak}{\char`\_}}%
            \def\PYGZob{\discretionary{}{\Wrappedafterbreak\char`\{}{\char`\{}}%
            \def\PYGZcb{\discretionary{\char`\}}{\Wrappedafterbreak}{\char`\}}}%
            \def\PYGZca{\discretionary{\char`\^}{\Wrappedafterbreak}{\char`\^}}%
            \def\PYGZam{\discretionary{\char`\&}{\Wrappedafterbreak}{\char`\&}}%
            \def\PYGZlt{\discretionary{}{\Wrappedafterbreak\char`\<}{\char`\<}}%
            \def\PYGZgt{\discretionary{\char`\>}{\Wrappedafterbreak}{\char`\>}}%
            \def\PYGZsh{\discretionary{}{\Wrappedafterbreak\char`\#}{\char`\#}}%
            \def\PYGZpc{\discretionary{}{\Wrappedafterbreak\char`\%}{\char`\%}}%
            \def\PYGZdl{\discretionary{}{\Wrappedafterbreak\char`\$}{\char`\$}}%
            \def\PYGZhy{\discretionary{\char`\-}{\Wrappedafterbreak}{\char`\-}}%
            \def\PYGZsq{\discretionary{}{\Wrappedafterbreak\textquotesingle}{\textquotesingle}}%
            \def\PYGZdq{\discretionary{}{\Wrappedafterbreak\char`\"}{\char`\"}}%
            \def\PYGZti{\discretionary{\char`\~}{\Wrappedafterbreak}{\char`\~}}%
        }
        % Some characters . , ; ? ! / are not pygmentized.
        % This macro makes them "active" and they will insert potential linebreaks
        \newcommand*\Wrappedbreaksatpunct {%
            \lccode`\~`\.\lowercase{\def~}{\discretionary{\hbox{\char`\.}}{\Wrappedafterbreak}{\hbox{\char`\.}}}%
            \lccode`\~`\,\lowercase{\def~}{\discretionary{\hbox{\char`\,}}{\Wrappedafterbreak}{\hbox{\char`\,}}}%
            \lccode`\~`\;\lowercase{\def~}{\discretionary{\hbox{\char`\;}}{\Wrappedafterbreak}{\hbox{\char`\;}}}%
            \lccode`\~`\:\lowercase{\def~}{\discretionary{\hbox{\char`\:}}{\Wrappedafterbreak}{\hbox{\char`\:}}}%
            \lccode`\~`\?\lowercase{\def~}{\discretionary{\hbox{\char`\?}}{\Wrappedafterbreak}{\hbox{\char`\?}}}%
            \lccode`\~`\!\lowercase{\def~}{\discretionary{\hbox{\char`\!}}{\Wrappedafterbreak}{\hbox{\char`\!}}}%
            \lccode`\~`\/\lowercase{\def~}{\discretionary{\hbox{\char`\/}}{\Wrappedafterbreak}{\hbox{\char`\/}}}%
            \catcode`\.\active
            \catcode`\,\active
            \catcode`\;\active
            \catcode`\:\active
            \catcode`\?\active
            \catcode`\!\active
            \catcode`\/\active
            \lccode`\~`\~
        }
    \makeatother

    \let\OriginalVerbatim=\Verbatim
    \makeatletter
    \renewcommand{\Verbatim}[1][1]{%
        %\parskip\z@skip
        \sbox\Wrappedcontinuationbox {\Wrappedcontinuationsymbol}%
        \sbox\Wrappedvisiblespacebox {\FV@SetupFont\Wrappedvisiblespace}%
        \def\FancyVerbFormatLine ##1{\hsize\linewidth
            \vtop{\raggedright\hyphenpenalty\z@\exhyphenpenalty\z@
                \doublehyphendemerits\z@\finalhyphendemerits\z@
                \strut ##1\strut}%
        }%
        % If the linebreak is at a space, the latter will be displayed as visible
        % space at end of first line, and a continuation symbol starts next line.
        % Stretch/shrink are however usually zero for typewriter font.
        \def\FV@Space {%
            \nobreak\hskip\z@ plus\fontdimen3\font minus\fontdimen4\font
            \discretionary{\copy\Wrappedvisiblespacebox}{\Wrappedafterbreak}
            {\kern\fontdimen2\font}%
        }%

        % Allow breaks at special characters using \PYG... macros.
        \Wrappedbreaksatspecials
        % Breaks at punctuation characters . , ; ? ! and / need catcode=\active
        \OriginalVerbatim[#1,codes*=\Wrappedbreaksatpunct]%
    }
    \makeatother

    % Exact colors from NB
    \definecolor{incolor}{HTML}{303F9F}
    \definecolor{outcolor}{HTML}{D84315}
    \definecolor{cellborder}{HTML}{CFCFCF}
    \definecolor{cellbackground}{HTML}{F7F7F7}

    % prompt
    \makeatletter
    \newcommand{\boxspacing}{\kern\kvtcb@left@rule\kern\kvtcb@boxsep}
    \makeatother
    \newcommand{\prompt}[4]{
        {\ttfamily\llap{{\color{#2}[#3]:\hspace{3pt}#4}}\vspace{-\baselineskip}}
    }
    

    
% Start the section counter at -1, so the Table of Contents is Section 0
   \setcounter{section}{-2}
% Prevent overflowing lines due to hard-to-break entities
    \sloppy
    % Setup hyperref package
    \hypersetup{
      breaklinks=true,  % so long urls are correctly broken across lines
      colorlinks=true,
      urlcolor=urlcolor,
      linkcolor=linkcolor,
      citecolor=citecolor,
      }

    % Slightly bigger margins than the latex defaults
    \geometry{verbose,tmargin=0.5in,bmargin=0.5in,lmargin=0.5in,rmargin=0.5in}


\begin{document}
    
    \maketitle
    
    

    
    \hypertarget{general-relativity-problems-chapter-6-curved-manifolds}{%
\section{General Relativity Problems Chapter 6: Curved
Manifolds}\label{general-relativity-problems-chapter-6-curved-manifolds}}

\hypertarget{authors-gabriel-m-steward}{%
\subsection{Authors: Gabriel M
Steward}\label{authors-gabriel-m-steward}}

    https://github.com/zachetienne/nrpytutorial/blob/master/Tutorial-Template\_Style\_Guide.ipynb

Link to the Style Guide. Not internal in case something breaks.

    \hypertarget{nrpy-source-code-for-this-module}{%
\subsubsection{\texorpdfstring{ NRPy+ Source Code for this
module:}{ NRPy+ Source Code for this module:}}\label{nrpy-source-code-for-this-module}}

None!

\hypertarget{introduction}{%
\subsection{Introduction:}\label{introduction}}

Are we finally giong to get to actual curved space? Will the land of
special relativity be left behind at long last? We've certainly spent a
lot of work building up to this point\ldots{}

\hypertarget{other-optional}{%
\subsection{\texorpdfstring{ Other
(Optional):}{ Other (Optional):}}\label{other-optional}}

Placeholder.

\hypertarget{note-on-notation}{%
\subsubsection{Note on Notation:}\label{note-on-notation}}

Any new notation will be brought up in the notebook when it becomes
relevant.

\hypertarget{citations}{%
\subsubsection{Citations:}\label{citations}}

All citations will be collected here.

    \hypertarget{table-of-contents}{%
\section{Table of Contents}\label{table-of-contents}}

\[\label{toc}\]

\hyperref[p1]{Problem 1} (What are Manifolds?)

\hyperref[p2]{Problem 2} (Manifold Metrics)

\hyperref[p3]{Problem 3} (Proof of Metric Transform's Existence)

\hyperref[p4]{Problem 4} (Local Flatness Theorem)

\hyperref[p5]{Problem 5} (Curved and Flat Christoffel Coefficients)

\hyperref[p6]{Problem 6} (A Vanishing Proof)

\hyperref[p6]{Problem 6} (A Vanishing Proof)

\hyperref[latex_pdf_output]{PDF} (turn this into a PDF)

    \hypertarget{problem-1-back-to-top}{%
\section{\texorpdfstring{Problem 1 {[}Back to
\hyperref[toc]{top}{]}}{Problem 1 {[}Back to {]}}}\label{problem-1-back-to-top}}

\[\label{P1}\]

\emph{Decide if the following sets are manifolds and say why. If there
are any exceptional points at which the sets are not manifolds, give
them:}

\emph{a) Phase space of Hamiltonian mechanics, the space of the
canonical coordinates and momenta \(p_i\) and \(q^i\)}

    A manifold is best thought of as just a parameterizable space. The space
of coordinates and momentum are perfectly fine in this way, and are in
fact often mapped to cartesian grids in the first place. So yes, this is
a manifold.

    \emph{b) The interior of a circle of unit radius in two-dimensional
Euclidian space}

Is a manifold everywhere except the center, much like polar coordinates.
(Note: we are definitely assuming that we can parameterize however we
want.)

    \emph{c) The set of permutations of n objects.}

This would be a no. Because we need CONTINUOUS parameterization. No
matter how fancy the set of permutations is, it's still discrete.

    \emph{d) The subset of Euclidean space of two dimensions (coordinates x
and y) which is a solution to xy\((x^2+y^2-1)\)=0}

Well this is interesting\ldots{}

\begin{figure}
\centering
\includegraphics{attachment:Screenshot\%20from\%202022-05-29\%2014-18-43.png}
\caption{Screenshot\%20from\%202022-05-29\%2014-18-43.png}
\end{figure}

    The circle by itself can clearly be a manifold (of one dimension too!)
but the rest? Discontinuities all over the place\ldots{} no, I don't
think this is a manifold, as trying to parameterize it doesn't seem to
pan out.

As for a more strict reason why not\ldots{} er\ldots{} So define the
angle around the circle as one dimension, right? Well, suddenly that
dimension can't be used if the distance from the origin is anything but
1--that is, it becomes FOUR DISCRETE VALUES, not continuous.

If we could trace a single line through the shape that covers every
portion, we would have a way out. However we have four ``exit'' points
which makes that impossible, meaning we do need two parameters for this
space, of which no continouous one exists.

    \hypertarget{problem-2-back-to-top}{%
\section{\texorpdfstring{Problem 2 {[}Back to
\hyperref[toc]{top}{]}}{Problem 2 {[}Back to {]}}}\label{problem-2-back-to-top}}

\[\label{P2}\]

\emph{Of the manifolds in \textbf{Problem 1}, on which is it customary
to use a metric, and what is that metric? On which would a metric not
normally be defined and why?}

First of all the permutaiions and ``sniper shot'' set can be dismissed,
as they are not manifolds, which just leaves Phase Space and The Circle.

    For Phase Space, it's generally just a cartesian metric, so it's the
cartesian metric.

For the circle, one would use the radial metric. However, it's not
actually customary to do this, as most use unit vectors which are
decidedly not coordinate bases. usually this isn't a problem as
cartesian circles have identical one forms and vectors so there's no
need to convert.

((Perhaps later understanding will be gained that makes all of this look
silly. Maybe the ``sniper shot'' really is a manifold, though clearly a
metric wouldn't be very helpful as it's not smooth at all anywhere. Look
at those holes\ldots))

    \hypertarget{problem-3-back-to-top}{%
\section{\texorpdfstring{Problem 3 {[}Back to
\hyperref[toc]{top}{]}}{Problem 3 {[}Back to {]}}}\label{problem-3-back-to-top}}

\[\label{P3}\]

\emph{It is well known that for any symmetric matrix A (with real
entries) ther exists a matrix H for which the matrix \(H^TAH\) is a
diagonal matrix whos entries are the eigenvalues of A.}

\emph{a) Show that there is a matrix R such that \(R^TH^TAHR\) is the
same matrix as \(H^TAH\) except with the eigenvalues rearranged in
ascending order along the main diagonal from top to bottom.}

    So we start by knowing that there is a diagonal matrix. What we need to
show is that there exists a ``shuffling'' matrix that can re-arrange the
eigenvalues. Note: we just need to show that one exists. It could be
anything, and it doesn't have to be ``smart'' about it, that is, it
doesn't have to \emph{choose} the highest value to go to the highest
place.

R will be a matrix with 1 and 0 values off the diagonal (or on the
diagonal if a value is already in the right spot). Note that R itself
will be a symmetric matrix, using ``1'' values off the edge to swap the
position of two eigenvalues. If more than one swap needs to be done, R
can just be a combination of all the matrices required to do the swap.

Good stuff, proven.

    \emph{b) Show that there exists a third matrix such that
\(N^TR^TH^TAHRN\) is a diagonal matrix whose entries on the diagonal are
-1,0, or +1.}

    This one's even easier. Let N itself be a diagonal matrix, so that all
the values are simply diagnonal values multiplied by diagonal values,
there is no swapping or moving. In this case, transpose matrix
multiplicaiton is commutative, so we can just make \(N^TN\) = NN. We do
this so we can define N to be its own transpose and say that the NN
matrix is just 1/\textbar eigenvalue\textbar{} of every eigenvalue in
the right position. This will reduce everything into units.

    \emph{c) Show that if A has an inverse, none of the diagonal elements in
b) is zero.}

If you have a zero eigenvalue, then the determinant is zero, which means
the matrix has no inverse.

So you have to have all your eigenvalues to be invertible.

    \emph{d) Show from a)-c) that there exists a transformation matrix
\(\Lambda\) which produces 6.2}

6.2 is just the general metric tensor \(\eta\) with diagonal.
(-1,1,1,1).

Assuming A is invertible (it better be), then we automatically have a
way to convert to \((\pm1,\pm1,\pm1,\pm1)\). When building your nice
transformation matrix, just be sure to add a final matrix to the end:
one that changes the signs along the diagonal to what we need.

What about the fact that some of our matrices are past the right side?
We can easily move N across, but what about R and H? All we can really
say for sure is that there exists a SANDWICH around A that produces the
correct metric, that is \(L^TAL = \textbf{g}\).

But let's think about what this means, expanding it into tensor
notation.

\[ L^{\alpha'}_{\beta} A^\beta_{\delta} L^{\delta}_{\gamma'} = L^{\alpha'}_{\beta} L^{\delta}_{\gamma'}  A^\beta_{\delta} \]

    So the quesiton now is, can the actions of the two matrices be done by
one? The only fact we have is that one is the transpose of the other.
Playing the game of transposes does nothing.

How on earth does \$ L\^{}T A L = \eta \rightarrow \Lambda A = \eta \$.

All we've proven is that there exists a ``sandwich'' that will transform
A into the standard metric, not a \textbf{single} matrix.

    Wait, hold on, we're stupid.

The transformation matrix IS NOT USED ONLY ONCE TO TRANSFORM A MATRIX.

The matrix transforms one of the basis vectors.

TO GET A FULL TRANSFORM ANY MATRIX MUST BE USED TWICE. Once for the
rows, once for the columns.

That's what the transpose notaiton MEANS.

    \hypertarget{problem-4-back-to-top}{%
\section{\texorpdfstring{Problem 4 {[}Back to
\hyperref[toc]{top}{]}}{Problem 4 {[}Back to {]}}}\label{problem-4-back-to-top}}

\[\label{P4}\]

\emph{Prove the following results used in the proof of the local
flatness theorem in Section 6.2}

\emph{a) the number of independent values of
\(\partial^2x^\alpha/\partial x^{\gamma'} \partial x^{\mu'}|_0\) is 40}

    We're looking for independent values of a matrix, here. We know we are
symmetric in \(\gamma\) and \(\mu\) since the order you take derivatives
in does not matter. For a symmetric 2x2 matrix, there are 10 independnet
values (4 along the diagonal, 6 off-diagonals).

Simply multiply this by the four indeces on top to get 40.

    \emph{b) the number of independent values of
\(\partial^3x^\alpha/\partial x^{\lambda'} \partial x^{\mu'} \partial x^{\nu'}|_0\)
is 80}

    What is the symmetry along a 4x4x4 3D symmetry matrix? We have to look
at combinations here, every simple two-fold shuffling will be the same.
So let's just list unique numbers.

000

111

222

333

These are the diagonal, they are independent, and there are four of
them. Now let's classify all the others:

001 = 010 = 100

110 = 101 = 011

So each combination of two numbers has 3 permutations. We have 10, 12,
13, 20, 23, 30. That Gives us 12 independent places.

123 = 231 = 312 = 321 = 213 = 132

Every combination of three numbers gives 6 permutation. as there are
four different combinations of three, we have 4 independent places.

4 + 12 + 4 = 20.

20 differentiated four ways produces 80. As we were supposed to have.

    \emph{c) The corresponding number for
\(g_{\alpha\beta,\gamma',\mu'}|_0\) is 100}

First of all, on page 150, we see that the metric itself has 10
independent terms. Why? While it is symmetric, aren't all numbers aside
from the diagonal zero? Well, no actually, the metric can be transformed
into many things, not all of which are along the diagonal, but it must
be symmetric, so we have 10 independent for hte base matric.

The base metric differentiated once is obviously just 10 times 4 = 40.

However, what we want is the metric differentiated \emph{twice}.

Fortunately we know that taking derivatives in eitehr order don't change
anything, so the ``derivative'' matrix itself is symmetric, arriving at
10 independent terms.

10 by 10 is 100, and we are done!

    \hypertarget{problem-5-back-to-top}{%
\section{\texorpdfstring{Problem 5 {[}Back to
\hyperref[toc]{top}{]}}{Problem 5 {[}Back to {]}}}\label{problem-5-back-to-top}}

\[\label{P5}\]

\emph{a) Prove that
\(\Gamma^\mu_{\alpha\beta} = \Gamma^\mu_{\beta\alpha}\) in any
coordinate system in a curved Reimannian space.}

    Actually 5.74 points out that it is already true for ANY coordinate
system, but lets go through the proof as to why.

To start, we consider an arbitrary scalar field \(\phi\), that is, a
rule for getting numbers out of every point on some space. Plug in
various numbers, get a single number out. It's first derivative is a
one-form: \(\nabla \phi\) with components \(\phi_{,\beta}\), that is to
day derivatives taken with respect to every single coordinate. We don't
care how many there are, there could be thousands.

the second covariant derivative \(\nabla\nabla\phi\) has components
\(\phi_{,\beta ;\alpha}\) and it should be rather obvious that this is a
\(0\choose2\) tensor. (one form it twice). We do find ourselves asking
why one of them has a comma and the other a semicolon\ldots{} the
reason, it turns out, is because first we were differentiating a scalar,
but the second one had us differentiating a one-form which will always
add the Christoffel symbol.

Regardless, since this second derivative is a tensor, we note that it is
the same no matter what basis it is in. Since we can alter the order of
differentiation and change absolutely nothing in cartesian coordiantes,
we can alter the order here. So\ldots{}

\(\phi_{,\alpha;\beta} = \phi_{,\beta;\alpha}\)

Now if we expand these we end up with things of the form
\$\phi\emph{\{,\alpha,\beta\} - \phi}\{,\mu\}
\Gamma\^{}\mu\emph{\{\beta\alpha\} = \phi}\{,\beta,\alpha\} -
\phi\emph{\{,\mu\} \Gamma\^{}\mu}\{\alpha\beta\} \$

Of course we ALSO know that the individual derivative terms are also
always interchangeable in any coordinate system. So that jsut
leaves\ldots{}

\[ \phi_{,\mu} \Gamma^\mu_{\beta\alpha} = \phi_{,\mu} \Gamma^\mu_{\alpha\beta} \]

Which rather trivially shows that the Christoffel coefficients are the
same no matter WHAT coordinate system we are in. This hinges on two main
facts: 1) that derivative order does not matter and 2) tensors are the
same in every coordinate system.

    \emph{b) Use this to prove that 6.32 can be derived in the same manner
as in flat space.}

    The derivation, so far as we can tell, is exactly the same since any
curved space is locally flat. By that rule alone, anything that can have
a metric assigned to it can have calculated Christoffel coefficietns.

    \hypertarget{problem-6-back-to-top}{%
\section{\texorpdfstring{Problem 6 {[}Back to
\hyperref[toc]{top}{]}}{Problem 6 {[}Back to {]}}}\label{problem-6-back-to-top}}

\[\label{P6}\]

\emph{Prove that the first term in 6.37 vanishes.}

6.37: \$ \Gamma\^{}\alpha\emph{\{\mu\alpha\}=\frac12
g\^{}\{\alpha\beta\} (g}\{\beta\mu,\alpha\} - g\_\{\mu\alpha,\beta\}) +
\frac12 g\^{}\{\alpha\beta\} g\_\{\alpha\beta,\mu\} \$

    First of all, note that \(g^{\alpha\beta}\) is symmetric by definition.

The term in the parenthesis is antisymmetric in \(\alpha\beta\) since
when they flip, the two terms essentially swap places. (note: the first
two indeces on g make no change as g itself is symmetric.) Thus, every
difference is also going to have it's inverse.

We note that we are summing over the exact antisymmetric indeces, so
every single value is going to be added to every other one. If the
magnitudes of both are the same, they cancel. and whatddoyaknow, we
already established that our leading term was symmetric! So it alllll
vanishes!

Proof complete.

    \hypertarget{problem-7-back-to-top}{%
\section{\texorpdfstring{Problem 7 {[}Back to
\hyperref[toc]{top}{]}}{Problem 7 {[}Back to {]}}}\label{problem-7-back-to-top}}

\[\label{P7}\]

\emph{a) Give the definition of the determinant of a matrix A in terms
of cofactors of elements.}

    \emph{b) Differentiate the determinant of an arbitrary 2x2 matrix and
show that it satisfies 6.39}

    \emph{c) Generalize 6.39 (by induction or otherwise) to arbitrary nxn
matrices.}

    \hypertarget{addendum-output-this-notebook-to-latex-formatted-pdf-file-back-to-top}{%
\section{\texorpdfstring{Addendum: Output this notebook to
\(\LaTeX\)-formatted PDF file {[}Back to
\hyperref[toc]{top}{]}}{Addendum: Output this notebook to \textbackslash LaTeX-formatted PDF file {[}Back to {]}}}\label{addendum-output-this-notebook-to-latex-formatted-pdf-file-back-to-top}}

\[\label{latex_pdf_output}\]

The following code cell converts this Jupyter notebook into a proper,
clickable \(\LaTeX\)-formatted PDF file. After the cell is successfully
run, the generated PDF may be found in the root NRPy+ tutorial
directory, with filename \url{GR-06.pdf} (Note that clicking on this
link may not work; you may need to open the PDF file through another
means.)

\textbf{Important Note}: Make sure that the file name is right in all
six locations, two here in the Markdown, four in the code below.

\begin{itemize}
\tightlist
\item
  GR-06.pdf
\item
  GR-06.ipynb
\item
  GR-06.tex
\end{itemize}

    \begin{tcolorbox}[breakable, size=fbox, boxrule=1pt, pad at break*=1mm,colback=cellbackground, colframe=cellborder]
\prompt{In}{incolor}{1}{\boxspacing}
\begin{Verbatim}[commandchars=\\\{\}]
\PY{k+kn}{import} \PY{n+nn}{cmdline\PYZus{}helper} \PY{k}{as} \PY{n+nn}{cmd}    \PY{c+c1}{\PYZsh{} NRPy+: Multi\PYZhy{}platform Python command\PYZhy{}line interface}
\PY{n}{cmd}\PY{o}{.}\PY{n}{output\PYZus{}Jupyter\PYZus{}notebook\PYZus{}to\PYZus{}LaTeXed\PYZus{}PDF}\PY{p}{(}\PY{l+s+s2}{\PYZdq{}}\PY{l+s+s2}{GR\PYZhy{}06}\PY{l+s+s2}{\PYZdq{}}\PY{p}{)}
\end{Verbatim}
\end{tcolorbox}

    \begin{Verbatim}[commandchars=\\\{\}]
Created GR-06.tex, and compiled LaTeX file to PDF file GR-06.pdf
    \end{Verbatim}

    \begin{tcolorbox}[breakable, size=fbox, boxrule=1pt, pad at break*=1mm,colback=cellbackground, colframe=cellborder]
\prompt{In}{incolor}{ }{\boxspacing}
\begin{Verbatim}[commandchars=\\\{\}]

\end{Verbatim}
\end{tcolorbox}

    \begin{tcolorbox}[breakable, size=fbox, boxrule=1pt, pad at break*=1mm,colback=cellbackground, colframe=cellborder]
\prompt{In}{incolor}{ }{\boxspacing}
\begin{Verbatim}[commandchars=\\\{\}]

\end{Verbatim}
\end{tcolorbox}


    % Add a bibliography block to the postdoc
    
    
    
\end{document}
