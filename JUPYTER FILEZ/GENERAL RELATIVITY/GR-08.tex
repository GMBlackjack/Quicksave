% Based on http://nbviewer.jupyter.org/github/ipython/nbconvert-examples/blob/master/citations/Tutorial.ipynb , authored by Brian E. Granger
    % Declare the document class
    \documentclass[landscape,letterpaper,10pt,english]{article}


    \usepackage[breakable]{tcolorbox}
    \usepackage{parskip} % Stop auto-indenting (to mimic markdown behaviour)
    

    % Basic figure setup, for now with no caption control since it's done
    % automatically by Pandoc (which extracts ![](path) syntax from Markdown).
    \usepackage{graphicx}
    % Maintain compatibility with old templates. Remove in nbconvert 6.0
    \let\Oldincludegraphics\includegraphics
    % Ensure that by default, figures have no caption (until we provide a
    % proper Figure object with a Caption API and a way to capture that
    % in the conversion process - todo).
    \usepackage{caption}
    \DeclareCaptionFormat{nocaption}{}
    \captionsetup{format=nocaption,aboveskip=0pt,belowskip=0pt}

    \usepackage{float}
    \floatplacement{figure}{H} % forces figures to be placed at the correct location
    \usepackage{xcolor} % Allow colors to be defined
    \usepackage{enumerate} % Needed for markdown enumerations to work
    \usepackage{geometry} % Used to adjust the document margins
    \usepackage{amsmath} % Equations
    \usepackage{amssymb} % Equations
    \usepackage{textcomp} % defines textquotesingle
    % Hack from http://tex.stackexchange.com/a/47451/13684:
    \AtBeginDocument{%
        \def\PYZsq{\textquotesingle}% Upright quotes in Pygmentized code
    }
    \usepackage{upquote} % Upright quotes for verbatim code
    \usepackage{eurosym} % defines \euro

    \usepackage{iftex}
    \ifPDFTeX
        \usepackage[T1]{fontenc}
        \IfFileExists{alphabeta.sty}{
              \usepackage{alphabeta}
          }{
              \usepackage[mathletters]{ucs}
              \usepackage[utf8x]{inputenc}
          }
    \else
        \usepackage{fontspec}
        \usepackage{unicode-math}
    \fi

    \usepackage{fancyvrb} % verbatim replacement that allows latex
    \usepackage{grffile} % extends the file name processing of package graphics
                         % to support a larger range
    \makeatletter % fix for old versions of grffile with XeLaTeX
    \@ifpackagelater{grffile}{2019/11/01}
    {
      % Do nothing on new versions
    }
    {
      \def\Gread@@xetex#1{%
        \IfFileExists{"\Gin@base".bb}%
        {\Gread@eps{\Gin@base.bb}}%
        {\Gread@@xetex@aux#1}%
      }
    }
    \makeatother
    \usepackage[Export]{adjustbox} % Used to constrain images to a maximum size
    \adjustboxset{max size={0.9\linewidth}{0.9\paperheight}}

    % The hyperref package gives us a pdf with properly built
    % internal navigation ('pdf bookmarks' for the table of contents,
    % internal cross-reference links, web links for URLs, etc.)
    \usepackage{hyperref}
    % The default LaTeX title has an obnoxious amount of whitespace. By default,
    % titling removes some of it. It also provides customization options.
    \usepackage{titling}
    \usepackage{longtable} % longtable support required by pandoc >1.10
    \usepackage{booktabs}  % table support for pandoc > 1.12.2
    \usepackage{array}     % table support for pandoc >= 2.11.3
    \usepackage{calc}      % table minipage width calculation for pandoc >= 2.11.1
    \usepackage[inline]{enumitem} % IRkernel/repr support (it uses the enumerate* environment)
    \usepackage[normalem]{ulem} % ulem is needed to support strikethroughs (\sout)
                                % normalem makes italics be italics, not underlines
    \usepackage{mathrsfs}
    

    
    % Colors for the hyperref package
    \definecolor{urlcolor}{rgb}{0,.145,.698}
    \definecolor{linkcolor}{rgb}{.71,0.21,0.01}
    \definecolor{citecolor}{rgb}{.12,.54,.11}

    % ANSI colors
    \definecolor{ansi-black}{HTML}{3E424D}
    \definecolor{ansi-black-intense}{HTML}{282C36}
    \definecolor{ansi-red}{HTML}{E75C58}
    \definecolor{ansi-red-intense}{HTML}{B22B31}
    \definecolor{ansi-green}{HTML}{00A250}
    \definecolor{ansi-green-intense}{HTML}{007427}
    \definecolor{ansi-yellow}{HTML}{DDB62B}
    \definecolor{ansi-yellow-intense}{HTML}{B27D12}
    \definecolor{ansi-blue}{HTML}{208FFB}
    \definecolor{ansi-blue-intense}{HTML}{0065CA}
    \definecolor{ansi-magenta}{HTML}{D160C4}
    \definecolor{ansi-magenta-intense}{HTML}{A03196}
    \definecolor{ansi-cyan}{HTML}{60C6C8}
    \definecolor{ansi-cyan-intense}{HTML}{258F8F}
    \definecolor{ansi-white}{HTML}{C5C1B4}
    \definecolor{ansi-white-intense}{HTML}{A1A6B2}
    \definecolor{ansi-default-inverse-fg}{HTML}{FFFFFF}
    \definecolor{ansi-default-inverse-bg}{HTML}{000000}

    % common color for the border for error outputs.
    \definecolor{outerrorbackground}{HTML}{FFDFDF}

    % commands and environments needed by pandoc snippets
    % extracted from the output of `pandoc -s`
    \providecommand{\tightlist}{%
      \setlength{\itemsep}{0pt}\setlength{\parskip}{0pt}}
    \DefineVerbatimEnvironment{Highlighting}{Verbatim}{commandchars=\\\{\}}
    % Add ',fontsize=\small' for more characters per line
    \newenvironment{Shaded}{}{}
    \newcommand{\KeywordTok}[1]{\textcolor[rgb]{0.00,0.44,0.13}{\textbf{{#1}}}}
    \newcommand{\DataTypeTok}[1]{\textcolor[rgb]{0.56,0.13,0.00}{{#1}}}
    \newcommand{\DecValTok}[1]{\textcolor[rgb]{0.25,0.63,0.44}{{#1}}}
    \newcommand{\BaseNTok}[1]{\textcolor[rgb]{0.25,0.63,0.44}{{#1}}}
    \newcommand{\FloatTok}[1]{\textcolor[rgb]{0.25,0.63,0.44}{{#1}}}
    \newcommand{\CharTok}[1]{\textcolor[rgb]{0.25,0.44,0.63}{{#1}}}
    \newcommand{\StringTok}[1]{\textcolor[rgb]{0.25,0.44,0.63}{{#1}}}
    \newcommand{\CommentTok}[1]{\textcolor[rgb]{0.38,0.63,0.69}{\textit{{#1}}}}
    \newcommand{\OtherTok}[1]{\textcolor[rgb]{0.00,0.44,0.13}{{#1}}}
    \newcommand{\AlertTok}[1]{\textcolor[rgb]{1.00,0.00,0.00}{\textbf{{#1}}}}
    \newcommand{\FunctionTok}[1]{\textcolor[rgb]{0.02,0.16,0.49}{{#1}}}
    \newcommand{\RegionMarkerTok}[1]{{#1}}
    \newcommand{\ErrorTok}[1]{\textcolor[rgb]{1.00,0.00,0.00}{\textbf{{#1}}}}
    \newcommand{\NormalTok}[1]{{#1}}

    % Additional commands for more recent versions of Pandoc
    \newcommand{\ConstantTok}[1]{\textcolor[rgb]{0.53,0.00,0.00}{{#1}}}
    \newcommand{\SpecialCharTok}[1]{\textcolor[rgb]{0.25,0.44,0.63}{{#1}}}
    \newcommand{\VerbatimStringTok}[1]{\textcolor[rgb]{0.25,0.44,0.63}{{#1}}}
    \newcommand{\SpecialStringTok}[1]{\textcolor[rgb]{0.73,0.40,0.53}{{#1}}}
    \newcommand{\ImportTok}[1]{{#1}}
    \newcommand{\DocumentationTok}[1]{\textcolor[rgb]{0.73,0.13,0.13}{\textit{{#1}}}}
    \newcommand{\AnnotationTok}[1]{\textcolor[rgb]{0.38,0.63,0.69}{\textbf{\textit{{#1}}}}}
    \newcommand{\CommentVarTok}[1]{\textcolor[rgb]{0.38,0.63,0.69}{\textbf{\textit{{#1}}}}}
    \newcommand{\VariableTok}[1]{\textcolor[rgb]{0.10,0.09,0.49}{{#1}}}
    \newcommand{\ControlFlowTok}[1]{\textcolor[rgb]{0.00,0.44,0.13}{\textbf{{#1}}}}
    \newcommand{\OperatorTok}[1]{\textcolor[rgb]{0.40,0.40,0.40}{{#1}}}
    \newcommand{\BuiltInTok}[1]{{#1}}
    \newcommand{\ExtensionTok}[1]{{#1}}
    \newcommand{\PreprocessorTok}[1]{\textcolor[rgb]{0.74,0.48,0.00}{{#1}}}
    \newcommand{\AttributeTok}[1]{\textcolor[rgb]{0.49,0.56,0.16}{{#1}}}
    \newcommand{\InformationTok}[1]{\textcolor[rgb]{0.38,0.63,0.69}{\textbf{\textit{{#1}}}}}
    \newcommand{\WarningTok}[1]{\textcolor[rgb]{0.38,0.63,0.69}{\textbf{\textit{{#1}}}}}


    % Define a nice break command that doesn't care if a line doesn't already
    % exist.
    \def\br{\hspace*{\fill} \\* }
    % Math Jax compatibility definitions
    \def\gt{>}
    \def\lt{<}
    \let\Oldtex\TeX
    \let\Oldlatex\LaTeX
    \renewcommand{\TeX}{\textrm{\Oldtex}}
    \renewcommand{\LaTeX}{\textrm{\Oldlatex}}
    % Document parameters
    % Document title
    \title{GR-08}
    
    
    
    
    
% Pygments definitions
\makeatletter
\def\PY@reset{\let\PY@it=\relax \let\PY@bf=\relax%
    \let\PY@ul=\relax \let\PY@tc=\relax%
    \let\PY@bc=\relax \let\PY@ff=\relax}
\def\PY@tok#1{\csname PY@tok@#1\endcsname}
\def\PY@toks#1+{\ifx\relax#1\empty\else%
    \PY@tok{#1}\expandafter\PY@toks\fi}
\def\PY@do#1{\PY@bc{\PY@tc{\PY@ul{%
    \PY@it{\PY@bf{\PY@ff{#1}}}}}}}
\def\PY#1#2{\PY@reset\PY@toks#1+\relax+\PY@do{#2}}

\@namedef{PY@tok@w}{\def\PY@tc##1{\textcolor[rgb]{0.73,0.73,0.73}{##1}}}
\@namedef{PY@tok@c}{\let\PY@it=\textit\def\PY@tc##1{\textcolor[rgb]{0.24,0.48,0.48}{##1}}}
\@namedef{PY@tok@cp}{\def\PY@tc##1{\textcolor[rgb]{0.61,0.40,0.00}{##1}}}
\@namedef{PY@tok@k}{\let\PY@bf=\textbf\def\PY@tc##1{\textcolor[rgb]{0.00,0.50,0.00}{##1}}}
\@namedef{PY@tok@kp}{\def\PY@tc##1{\textcolor[rgb]{0.00,0.50,0.00}{##1}}}
\@namedef{PY@tok@kt}{\def\PY@tc##1{\textcolor[rgb]{0.69,0.00,0.25}{##1}}}
\@namedef{PY@tok@o}{\def\PY@tc##1{\textcolor[rgb]{0.40,0.40,0.40}{##1}}}
\@namedef{PY@tok@ow}{\let\PY@bf=\textbf\def\PY@tc##1{\textcolor[rgb]{0.67,0.13,1.00}{##1}}}
\@namedef{PY@tok@nb}{\def\PY@tc##1{\textcolor[rgb]{0.00,0.50,0.00}{##1}}}
\@namedef{PY@tok@nf}{\def\PY@tc##1{\textcolor[rgb]{0.00,0.00,1.00}{##1}}}
\@namedef{PY@tok@nc}{\let\PY@bf=\textbf\def\PY@tc##1{\textcolor[rgb]{0.00,0.00,1.00}{##1}}}
\@namedef{PY@tok@nn}{\let\PY@bf=\textbf\def\PY@tc##1{\textcolor[rgb]{0.00,0.00,1.00}{##1}}}
\@namedef{PY@tok@ne}{\let\PY@bf=\textbf\def\PY@tc##1{\textcolor[rgb]{0.80,0.25,0.22}{##1}}}
\@namedef{PY@tok@nv}{\def\PY@tc##1{\textcolor[rgb]{0.10,0.09,0.49}{##1}}}
\@namedef{PY@tok@no}{\def\PY@tc##1{\textcolor[rgb]{0.53,0.00,0.00}{##1}}}
\@namedef{PY@tok@nl}{\def\PY@tc##1{\textcolor[rgb]{0.46,0.46,0.00}{##1}}}
\@namedef{PY@tok@ni}{\let\PY@bf=\textbf\def\PY@tc##1{\textcolor[rgb]{0.44,0.44,0.44}{##1}}}
\@namedef{PY@tok@na}{\def\PY@tc##1{\textcolor[rgb]{0.41,0.47,0.13}{##1}}}
\@namedef{PY@tok@nt}{\let\PY@bf=\textbf\def\PY@tc##1{\textcolor[rgb]{0.00,0.50,0.00}{##1}}}
\@namedef{PY@tok@nd}{\def\PY@tc##1{\textcolor[rgb]{0.67,0.13,1.00}{##1}}}
\@namedef{PY@tok@s}{\def\PY@tc##1{\textcolor[rgb]{0.73,0.13,0.13}{##1}}}
\@namedef{PY@tok@sd}{\let\PY@it=\textit\def\PY@tc##1{\textcolor[rgb]{0.73,0.13,0.13}{##1}}}
\@namedef{PY@tok@si}{\let\PY@bf=\textbf\def\PY@tc##1{\textcolor[rgb]{0.64,0.35,0.47}{##1}}}
\@namedef{PY@tok@se}{\let\PY@bf=\textbf\def\PY@tc##1{\textcolor[rgb]{0.67,0.36,0.12}{##1}}}
\@namedef{PY@tok@sr}{\def\PY@tc##1{\textcolor[rgb]{0.64,0.35,0.47}{##1}}}
\@namedef{PY@tok@ss}{\def\PY@tc##1{\textcolor[rgb]{0.10,0.09,0.49}{##1}}}
\@namedef{PY@tok@sx}{\def\PY@tc##1{\textcolor[rgb]{0.00,0.50,0.00}{##1}}}
\@namedef{PY@tok@m}{\def\PY@tc##1{\textcolor[rgb]{0.40,0.40,0.40}{##1}}}
\@namedef{PY@tok@gh}{\let\PY@bf=\textbf\def\PY@tc##1{\textcolor[rgb]{0.00,0.00,0.50}{##1}}}
\@namedef{PY@tok@gu}{\let\PY@bf=\textbf\def\PY@tc##1{\textcolor[rgb]{0.50,0.00,0.50}{##1}}}
\@namedef{PY@tok@gd}{\def\PY@tc##1{\textcolor[rgb]{0.63,0.00,0.00}{##1}}}
\@namedef{PY@tok@gi}{\def\PY@tc##1{\textcolor[rgb]{0.00,0.52,0.00}{##1}}}
\@namedef{PY@tok@gr}{\def\PY@tc##1{\textcolor[rgb]{0.89,0.00,0.00}{##1}}}
\@namedef{PY@tok@ge}{\let\PY@it=\textit}
\@namedef{PY@tok@gs}{\let\PY@bf=\textbf}
\@namedef{PY@tok@gp}{\let\PY@bf=\textbf\def\PY@tc##1{\textcolor[rgb]{0.00,0.00,0.50}{##1}}}
\@namedef{PY@tok@go}{\def\PY@tc##1{\textcolor[rgb]{0.44,0.44,0.44}{##1}}}
\@namedef{PY@tok@gt}{\def\PY@tc##1{\textcolor[rgb]{0.00,0.27,0.87}{##1}}}
\@namedef{PY@tok@err}{\def\PY@bc##1{{\setlength{\fboxsep}{\string -\fboxrule}\fcolorbox[rgb]{1.00,0.00,0.00}{1,1,1}{\strut ##1}}}}
\@namedef{PY@tok@kc}{\let\PY@bf=\textbf\def\PY@tc##1{\textcolor[rgb]{0.00,0.50,0.00}{##1}}}
\@namedef{PY@tok@kd}{\let\PY@bf=\textbf\def\PY@tc##1{\textcolor[rgb]{0.00,0.50,0.00}{##1}}}
\@namedef{PY@tok@kn}{\let\PY@bf=\textbf\def\PY@tc##1{\textcolor[rgb]{0.00,0.50,0.00}{##1}}}
\@namedef{PY@tok@kr}{\let\PY@bf=\textbf\def\PY@tc##1{\textcolor[rgb]{0.00,0.50,0.00}{##1}}}
\@namedef{PY@tok@bp}{\def\PY@tc##1{\textcolor[rgb]{0.00,0.50,0.00}{##1}}}
\@namedef{PY@tok@fm}{\def\PY@tc##1{\textcolor[rgb]{0.00,0.00,1.00}{##1}}}
\@namedef{PY@tok@vc}{\def\PY@tc##1{\textcolor[rgb]{0.10,0.09,0.49}{##1}}}
\@namedef{PY@tok@vg}{\def\PY@tc##1{\textcolor[rgb]{0.10,0.09,0.49}{##1}}}
\@namedef{PY@tok@vi}{\def\PY@tc##1{\textcolor[rgb]{0.10,0.09,0.49}{##1}}}
\@namedef{PY@tok@vm}{\def\PY@tc##1{\textcolor[rgb]{0.10,0.09,0.49}{##1}}}
\@namedef{PY@tok@sa}{\def\PY@tc##1{\textcolor[rgb]{0.73,0.13,0.13}{##1}}}
\@namedef{PY@tok@sb}{\def\PY@tc##1{\textcolor[rgb]{0.73,0.13,0.13}{##1}}}
\@namedef{PY@tok@sc}{\def\PY@tc##1{\textcolor[rgb]{0.73,0.13,0.13}{##1}}}
\@namedef{PY@tok@dl}{\def\PY@tc##1{\textcolor[rgb]{0.73,0.13,0.13}{##1}}}
\@namedef{PY@tok@s2}{\def\PY@tc##1{\textcolor[rgb]{0.73,0.13,0.13}{##1}}}
\@namedef{PY@tok@sh}{\def\PY@tc##1{\textcolor[rgb]{0.73,0.13,0.13}{##1}}}
\@namedef{PY@tok@s1}{\def\PY@tc##1{\textcolor[rgb]{0.73,0.13,0.13}{##1}}}
\@namedef{PY@tok@mb}{\def\PY@tc##1{\textcolor[rgb]{0.40,0.40,0.40}{##1}}}
\@namedef{PY@tok@mf}{\def\PY@tc##1{\textcolor[rgb]{0.40,0.40,0.40}{##1}}}
\@namedef{PY@tok@mh}{\def\PY@tc##1{\textcolor[rgb]{0.40,0.40,0.40}{##1}}}
\@namedef{PY@tok@mi}{\def\PY@tc##1{\textcolor[rgb]{0.40,0.40,0.40}{##1}}}
\@namedef{PY@tok@il}{\def\PY@tc##1{\textcolor[rgb]{0.40,0.40,0.40}{##1}}}
\@namedef{PY@tok@mo}{\def\PY@tc##1{\textcolor[rgb]{0.40,0.40,0.40}{##1}}}
\@namedef{PY@tok@ch}{\let\PY@it=\textit\def\PY@tc##1{\textcolor[rgb]{0.24,0.48,0.48}{##1}}}
\@namedef{PY@tok@cm}{\let\PY@it=\textit\def\PY@tc##1{\textcolor[rgb]{0.24,0.48,0.48}{##1}}}
\@namedef{PY@tok@cpf}{\let\PY@it=\textit\def\PY@tc##1{\textcolor[rgb]{0.24,0.48,0.48}{##1}}}
\@namedef{PY@tok@c1}{\let\PY@it=\textit\def\PY@tc##1{\textcolor[rgb]{0.24,0.48,0.48}{##1}}}
\@namedef{PY@tok@cs}{\let\PY@it=\textit\def\PY@tc##1{\textcolor[rgb]{0.24,0.48,0.48}{##1}}}

\def\PYZbs{\char`\\}
\def\PYZus{\char`\_}
\def\PYZob{\char`\{}
\def\PYZcb{\char`\}}
\def\PYZca{\char`\^}
\def\PYZam{\char`\&}
\def\PYZlt{\char`\<}
\def\PYZgt{\char`\>}
\def\PYZsh{\char`\#}
\def\PYZpc{\char`\%}
\def\PYZdl{\char`\$}
\def\PYZhy{\char`\-}
\def\PYZsq{\char`\'}
\def\PYZdq{\char`\"}
\def\PYZti{\char`\~}
% for compatibility with earlier versions
\def\PYZat{@}
\def\PYZlb{[}
\def\PYZrb{]}
\makeatother


    % For linebreaks inside Verbatim environment from package fancyvrb.
    \makeatletter
        \newbox\Wrappedcontinuationbox
        \newbox\Wrappedvisiblespacebox
        \newcommand*\Wrappedvisiblespace {\textcolor{red}{\textvisiblespace}}
        \newcommand*\Wrappedcontinuationsymbol {\textcolor{red}{\llap{\tiny$\m@th\hookrightarrow$}}}
        \newcommand*\Wrappedcontinuationindent {3ex }
        \newcommand*\Wrappedafterbreak {\kern\Wrappedcontinuationindent\copy\Wrappedcontinuationbox}
        % Take advantage of the already applied Pygments mark-up to insert
        % potential linebreaks for TeX processing.
        %        {, <, #, %, $, ' and ": go to next line.
        %        _, }, ^, &, >, - and ~: stay at end of broken line.
        % Use of \textquotesingle for straight quote.
        \newcommand*\Wrappedbreaksatspecials {%
            \def\PYGZus{\discretionary{\char`\_}{\Wrappedafterbreak}{\char`\_}}%
            \def\PYGZob{\discretionary{}{\Wrappedafterbreak\char`\{}{\char`\{}}%
            \def\PYGZcb{\discretionary{\char`\}}{\Wrappedafterbreak}{\char`\}}}%
            \def\PYGZca{\discretionary{\char`\^}{\Wrappedafterbreak}{\char`\^}}%
            \def\PYGZam{\discretionary{\char`\&}{\Wrappedafterbreak}{\char`\&}}%
            \def\PYGZlt{\discretionary{}{\Wrappedafterbreak\char`\<}{\char`\<}}%
            \def\PYGZgt{\discretionary{\char`\>}{\Wrappedafterbreak}{\char`\>}}%
            \def\PYGZsh{\discretionary{}{\Wrappedafterbreak\char`\#}{\char`\#}}%
            \def\PYGZpc{\discretionary{}{\Wrappedafterbreak\char`\%}{\char`\%}}%
            \def\PYGZdl{\discretionary{}{\Wrappedafterbreak\char`\$}{\char`\$}}%
            \def\PYGZhy{\discretionary{\char`\-}{\Wrappedafterbreak}{\char`\-}}%
            \def\PYGZsq{\discretionary{}{\Wrappedafterbreak\textquotesingle}{\textquotesingle}}%
            \def\PYGZdq{\discretionary{}{\Wrappedafterbreak\char`\"}{\char`\"}}%
            \def\PYGZti{\discretionary{\char`\~}{\Wrappedafterbreak}{\char`\~}}%
        }
        % Some characters . , ; ? ! / are not pygmentized.
        % This macro makes them "active" and they will insert potential linebreaks
        \newcommand*\Wrappedbreaksatpunct {%
            \lccode`\~`\.\lowercase{\def~}{\discretionary{\hbox{\char`\.}}{\Wrappedafterbreak}{\hbox{\char`\.}}}%
            \lccode`\~`\,\lowercase{\def~}{\discretionary{\hbox{\char`\,}}{\Wrappedafterbreak}{\hbox{\char`\,}}}%
            \lccode`\~`\;\lowercase{\def~}{\discretionary{\hbox{\char`\;}}{\Wrappedafterbreak}{\hbox{\char`\;}}}%
            \lccode`\~`\:\lowercase{\def~}{\discretionary{\hbox{\char`\:}}{\Wrappedafterbreak}{\hbox{\char`\:}}}%
            \lccode`\~`\?\lowercase{\def~}{\discretionary{\hbox{\char`\?}}{\Wrappedafterbreak}{\hbox{\char`\?}}}%
            \lccode`\~`\!\lowercase{\def~}{\discretionary{\hbox{\char`\!}}{\Wrappedafterbreak}{\hbox{\char`\!}}}%
            \lccode`\~`\/\lowercase{\def~}{\discretionary{\hbox{\char`\/}}{\Wrappedafterbreak}{\hbox{\char`\/}}}%
            \catcode`\.\active
            \catcode`\,\active
            \catcode`\;\active
            \catcode`\:\active
            \catcode`\?\active
            \catcode`\!\active
            \catcode`\/\active
            \lccode`\~`\~
        }
    \makeatother

    \let\OriginalVerbatim=\Verbatim
    \makeatletter
    \renewcommand{\Verbatim}[1][1]{%
        %\parskip\z@skip
        \sbox\Wrappedcontinuationbox {\Wrappedcontinuationsymbol}%
        \sbox\Wrappedvisiblespacebox {\FV@SetupFont\Wrappedvisiblespace}%
        \def\FancyVerbFormatLine ##1{\hsize\linewidth
            \vtop{\raggedright\hyphenpenalty\z@\exhyphenpenalty\z@
                \doublehyphendemerits\z@\finalhyphendemerits\z@
                \strut ##1\strut}%
        }%
        % If the linebreak is at a space, the latter will be displayed as visible
        % space at end of first line, and a continuation symbol starts next line.
        % Stretch/shrink are however usually zero for typewriter font.
        \def\FV@Space {%
            \nobreak\hskip\z@ plus\fontdimen3\font minus\fontdimen4\font
            \discretionary{\copy\Wrappedvisiblespacebox}{\Wrappedafterbreak}
            {\kern\fontdimen2\font}%
        }%

        % Allow breaks at special characters using \PYG... macros.
        \Wrappedbreaksatspecials
        % Breaks at punctuation characters . , ; ? ! and / need catcode=\active
        \OriginalVerbatim[#1,codes*=\Wrappedbreaksatpunct]%
    }
    \makeatother

    % Exact colors from NB
    \definecolor{incolor}{HTML}{303F9F}
    \definecolor{outcolor}{HTML}{D84315}
    \definecolor{cellborder}{HTML}{CFCFCF}
    \definecolor{cellbackground}{HTML}{F7F7F7}

    % prompt
    \makeatletter
    \newcommand{\boxspacing}{\kern\kvtcb@left@rule\kern\kvtcb@boxsep}
    \makeatother
    \newcommand{\prompt}[4]{
        {\ttfamily\llap{{\color{#2}[#3]:\hspace{3pt}#4}}\vspace{-\baselineskip}}
    }
    

    
% Start the section counter at -1, so the Table of Contents is Section 0
   \setcounter{section}{-2}
% Prevent overflowing lines due to hard-to-break entities
    \sloppy
    % Setup hyperref package
    \hypersetup{
      breaklinks=true,  % so long urls are correctly broken across lines
      colorlinks=true,
      urlcolor=urlcolor,
      linkcolor=linkcolor,
      citecolor=citecolor,
      }

    % Slightly bigger margins than the latex defaults
    \geometry{verbose,tmargin=0.5in,bmargin=0.5in,lmargin=0.5in,rmargin=0.5in}


\begin{document}
    
    \maketitle
    
    

    
    \hypertarget{general-relativity-problems-chapter-8-the-einstein-field-equations}{%
\section{General Relativity Problems Chapter 8: The Einstein Field
Equations}\label{general-relativity-problems-chapter-8-the-einstein-field-equations}}

\hypertarget{authors-gabriel-m-steward}{%
\subsection{Authors: Gabriel M
Steward}\label{authors-gabriel-m-steward}}

    https://github.com/zachetienne/nrpytutorial/blob/master/Tutorial-Template\_Style\_Guide.ipynb

Link to the Style Guide. Not internal in case something breaks.

    \hypertarget{nrpy-source-code-for-this-module}{%
\subsubsection{\texorpdfstring{ NRPy+ Source Code for this
module:}{ NRPy+ Source Code for this module:}}\label{nrpy-source-code-for-this-module}}

None!

\hypertarget{introduction}{%
\subsection{Introduction:}\label{introduction}}

Now maybe we can apply what we've learned to some actual physical
problems. Maybe. One can hope.

\hypertarget{other-optional}{%
\subsection{\texorpdfstring{ Other
(Optional):}{ Other (Optional):}}\label{other-optional}}

Placeholder.

\hypertarget{note-on-notation}{%
\subsubsection{Note on Notation:}\label{note-on-notation}}

Any new notation will be brought up in the notebook when it becomes
relevant.

\hypertarget{citations}{%
\subsubsection{Citations:}\label{citations}}

{[}1{]} https://en.wikipedia.org/wiki/Laplace\_operator (Laplacian
Operator \(\nabla^2\))

{[}2{]} https://www.vedantu.com/physics/derivation-of-orbital-velocity
(Because we were beings stupid.)

    \hypertarget{table-of-contents}{%
\section{Table of Contents}\label{table-of-contents}}

\[\label{toc}\]

\hyperref[p1]{Problem 1} (The Point Particle)

\hyperref[p2]{Problem 2} (Geometrized Values, Plank Units)

\hyperref[p3]{Problem 3} (Geometrized Units Calculations)

\hyperref[latex_pdf_output]{PDF} (turn this into a PDF)

    \hypertarget{problem-1-back-to-top}{%
\section{\texorpdfstring{Problem 1 {[}Back to
\hyperref[toc]{top}{]}}{Problem 1 {[}Back to {]}}}\label{problem-1-back-to-top}}

\[\label{P1}\]

\emph{Show that 8.2 is a solution of 8.1 by the following method. Assume
the point particle to be at the origin r=0 and to produce a spherically
symmetric field. Then use Gauss' law on a sphere of radius r to
conclude\ldots{}}

\[\frac{d\phi}{dr} = \frac{Gm}{r^2}\]

\emph{Deduce 8.2 from this. Consider the behavior at infinity.}

    Right well this seems obvious, but let's write it out in full,
especially since we don't remember Gauss' law off the top of our head.
The 4D version given by 4.57 is

\[ \int V^\alpha_{,\alpha} d^4x = \oint V^\alpha n_\alpha d^3S \]

Which can be reduced to three dimensions if needed without much issue.

We snote that we start with 8.1, that is, \$ \nabla\^{}2 \phi = 4\pi G
\rho \$.

ALWAYS REMMEBER: \(\nabla^2 = \nabla \cdot \nabla\) is the LAPLACIAN. Do
not think it's some kind of simple double derivative, there's a dot
product hiding in there. ALSO ALSO it's not exactly normal for SPHERICAL
COORDINATES. So let's go actually grab it and stop trying to be cute.
\hyperref[1]{1}

\begin{figure}
\centering
\includegraphics{attachment:Screenshot\%20from\%202022-06-07\%2009-48-22.png}
\caption{Screenshot\%20from\%202022-06-07\%2009-48-22.png}
\end{figure}

    (Note that they're using different notation for \(\nabla^2\)) Now, since
we're spherically symmetric, the angular terms all go away immediately,
leaving us with just the radial terms. This makes an admittedly somewhat
strange adjustment to 8.1

\[ \frac{1}{r^2} \frac{\partial}{\partial r} (r^2 \frac{\partial \phi}{\partial r}) = 4\pi G \rho \]

    Now one looks at this and says ``wait, that makes the left side zero''
which of course it does. The density \(\rho\) is not really something we
have the benefit of using since we havev a point particle in effect and
would end up with massive discontinuities. But the equation should still
hold, so we can still do that thing where we integrate both sides over
all space because hwy not.

\[ \int \frac{1}{r^2} \frac{\partial}{\partial r} (r^2 \frac{\partial \phi}{\partial r}) d^3x = \int 4\pi G \rho d^3x \]

    Now the nice thing about density integrated over all space is that it
equals m, and everything else on the right side is a cosntant, so
victory is sweet.

\[ \Rightarrow \int \frac{1}{r^2} \frac{\partial}{\partial r} (r^2 \frac{\partial \phi}{\partial r}) d^3x = 4\pi G m \]

We tried to apply Gauss' law here, but we have to rememver that the
fraction out front depends on r, so it's not that simple, as it's not
inside a derivative. So we just do it the hard way, it seems.
Fortunately due to spherical symmetry we do just \emph{know} the angles
have no influence on the integarl.

\[ \Rightarrow \int \frac{1}{r^2} \frac{\partial}{\partial r} (r^2 \frac{\partial \phi}{\partial r}) r^2sin\theta drd\theta d\phi = 4\pi G m \]

\[  \Rightarrow 4 \pi \int \frac{\partial}{\partial r} (r^2 \frac{\partial \phi}{\partial r}) dr = 4\pi G m \]

And THIS is just an integral of a derivative over the same variable, so
it all pops out.

\[ \Rightarrow  4 \pi r^2 \frac{\partial \phi}{\partial r} = 4\pi G m \]

\[ \Rightarrow  \frac{\partial \phi}{\partial r} = \frac{G m}{r^2} \]

Which is what we were looking for. We express annoyance, it was much
easier to solve this once we stopped trying to use Gauss' law.

    Anyway take the antiderivative of both sides, the result is obviuos.

\[ \Rightarrow \phi = -\frac{G m}{r} \]

    The behavior at infinity is a rather quick drop to near-zero that never
quite reaches it. Which from what we know is how gravity works.

    \hypertarget{problem-2-back-to-top}{%
\section{\texorpdfstring{Problem 2 {[}Back to
\hyperref[toc]{top}{]}}{Problem 2 {[}Back to {]}}}\label{problem-2-back-to-top}}

\[\label{P2}\]

\emph{a) Derive the following useful conversion factors from the SI
values of G and c:}

\[ G/c^2 = 7.425e-28  m/kg = 1 \] \[ c^5/G = 3.629e52 J/s = 1 \]

    Just\ldots{} take the SI units and divide them. The result comes out
trivially.

The units are \(\frac{m^3}{kg s^2} \frac{s^2}{m^2} = \frac{m}{kg}\)

And for the second \$\frac{m^5}{s^5} \frac{kg s^2}{m^3} =
\frac{kg m^2}{s^3} = \frac{J}{s} \$

    \emph{b) Derive the values in geometerized units of the constants in
table 8.1 from their given values in SI units.}

Well, at least I can know the answers this time. Even if it is a bit
bleh.

For planck's constant the conversion factor is \(G/c^3\) which produces
a s/kg and has a value of 2.477e-36. Multiplied by the SI for planck's
constant (or well the modified constant, such is the way with
\(\hbar\)'s bar.) is 2.613e-70 \(m^2\). The book says 2.612. We say
close enough.

For the masses, the \(G/c^2\) conversion given above works.

Luminosity is just \(c^5/G\), making all the others trivial, only
\(\hbar\) needed any actual new work. Remember to actually invert it
though.

    \emph{c) Express the following quantities in geometerized units}

\emph{i) A density (typical of neutroan stars) \(\rho\) = 1e17
kg/\(m^3\)}

    This can be multiplied by G directly to get something in 1/\(m^2\),
specifically 6,674,000 1/\(m^2\)

    \emph{ii) a pressure (also typical of neutron stars) p = 1e33
kg/m\(s^2\)}

This one evaluates in the exact same way, multiply by G, though the
results are in 1/\(s^2\) units instead. This is 6.674e22 1\$s\^{}2\$,
otherwise known as ABSURD.

    \emph{iii) The acceleration of gravity on Earth's surface g=9.8
m/\(s^2\)}

Apply c by itself. \ldots Feels like we've already done this. 3.269e-8
1/s

    \emph{iv) The luminosity of a supernova, L=1e41 J/s}

We have the direct conversion \(c^5/G\) worked out for this one already.
though we do have to ``invert'' it to actually apply it. 2.756e12.
Unitless.

    \emph{d) Three dimensioned constants in nature are regarded as
fundamental: c, G, and \(\hbar\). With c=G=1, \(\hbar\) has units
\(m^2\), so \(\hbar^{1/2}\) defines a fundamental unit of length called
the Planck length. From table 8.1, we calculate \(\hbar^{1/2}\) =
1.616e-35. Since this number invovles the fundamental constants of
relativity, gravitaiton, and quantum theory, many physicists feel that
this length will play an important role in quantum gravity. Express this
length in terms of the SI values of c, G, and \(\hbar\). Similarly, use
the conservation factors to calculate the Planck mass and Planck time,
fundamental numbers formed from c, G, and \(\hbar\) that have the units
of mass and time respectively. Compare these fundamental numbers with
characteristic masses, lengths, and time scales that are known from
elementary particle theory.}

    Well we ultimately arrived at the altered value of \(\hbar\) with
G/\(c^3\). So we can declare the value here to be

\[ \sqrt{\frac{G\hbar}{c^3}} \].

now, going from m to s involves a dividing by c.~This becomes:

\[ \sqrt{\frac{G\hbar}{c^5}} \].

now the MASS\ldots{} we need to get a kg out, which implies we need to
divide by \(G\), which changes our units results from simply m to
\(kgs^2/m^2\). So we need to multiply again by \(c^2\). Factoring this
into the square root provides:

\[ \sqrt{\frac{\hbar c}{G}} \].

Which in fact match a quick internet search.

    \hypertarget{problem-3-back-to-top}{%
\section{\texorpdfstring{Problem 3 {[}Back to
\hyperref[toc]{top}{]}}{Problem 3 {[}Back to {]}}}\label{problem-3-back-to-top}}

\[\label{P3}\]

\emph{a) Calculate in geometrized units:}

\emph{i) The Newtonian potential \(\phi\) of the Sun at the Sun's
surface, radius 6.960e8 m.}

    \(\phi = -\frac{Gm}{r}\)

Now the final units of this is \(m^2/s^2\), which can be reduced by the
conversion factor \(1/c^2\) to a completely dimensionless quantity. We
do need the solar mass though, 1.989e30 kg. \$-Gm/rc\^{}2 \$ is
2.122e-6. (unitless)

To calculate this directly from an equation m/r, we'd use the conversion
factor for mass, G/\(c^2\). This is effectively the exact same
calculation.

    \emph{ii) The Newtonian potential \(\phi\) of the Sun at the radius of
Earth's orbit, r=1AU = 1.496e11 m}

    Simlar calculation to above, 9.872e-9, unitless.

    \emph{iii) The Newtonian potential \(\phi\) of Earth at its surface,
radius 6.371e6 m}

    Earth's mass is 5.972e24 kg. By the same calculation we have\ldots{}

6.960e-10, unitless.

    \emph{iv) The velocity of Earth in its orbit around the sun.}

From memory we know this is about 30km/s or 30,000 m/s, which when
divided by lightspeed is 0.0001.

    \emph{b) You should have found that your answer to ii) was larger than
iii). Why, then, do we on Earth feel Earth's gravitational pull much
more than the Sun's?}

    REFERENCE FRAMES! We do actually move with the sun's gravity--because we
keep moving in the same path the Earth does around the Sun! It just
pulls on us at more-or-less the exact same strength it pulls on Earth
everywhere, so there's no distinct difference we notice from its
presence or if it was absent. We are free-falling around the Sun, but
there's nothing to stop us and thus no force we feel.

Earth's ground, meanwhile, is actively opposing our freefall, and thus
we feel it.

    \emph{c) Show that a circular orbit around a body of mass M has an
orbital velocity, in Newtonian theroy, of \(\textbf{v}^2 = -\phi\),
where \(\phi\) is the Newtonian potential.}

    First of all our speed for earth indicates we had an estimate
here\ldots{} or that Earth's not on a circular orbit, or that other
things influence it, whatever. Anyway, let's derive this rule for
circular orbits in general. In a circular orbit, \(\phi\) is constant in
the radially inward direction.

Consider the polar geodesic of a circle, which we can reprsent by
\((cos\theta, sin\theta)\), with \(\theta\) being our parameter. The
force always pulls toward the center, that is, precisely
\((-\phi cos\theta, -\phi sin\theta)\) Assuming clockwise orbit, the
velocity vector in a circular orbit always points at a 90 degree angle,
or \((vsin\theta, -vcos\theta)\)

We need to correlate these two things, somehow, but right now they could
be anything.

F magnitude is equal to the change in momentum magnitude, which is
\(m\delta v.\)

So we take the derivative of v and find the magnitude. The derivative is
\((vcos\theta, vsin\theta)\), and the magnitude of this is just flat-out
v. F=mv\ldots? That's not right\ldots{}

Feeling stupid we went and looked up the derivation for this ELEMENTARY
PROBLEM.

\ldots Yes we were being stupid, we were trying to use the information
we had in this chapter and not just use the math of circular motion
which would make everything so much easier. \hyperref[2]{2}

Anyway jsut correlate the known gravitaitonal force and centripedal
acceleration GMm/\(r^2\) = m\(v^2\)/r and get the wonderous result of
\(v = \sqrt{\frac{GM}{r}}\) which is exactly what we sought.

    \hypertarget{problem-4-back-to-top}{%
\section{\texorpdfstring{Problem 4 {[}Back to
\hyperref[toc]{top}{]}}{Problem 4 {[}Back to {]}}}\label{problem-4-back-to-top}}

\[\label{P4}\]

\emph{a) Let A be an nxn matrix whose entries are all very small,
\(|A_{ij} << 1/n|\), and let I be the unit matrix. Show that}

\[ (I+A)^{-1} = I - A + A^2 - A^3 + A^4 ... \]

\emph{by proving that i) the series on the right hand converges
absolutely for each of the \(n^2\) entries, and ii) (I+A) times the
right-hand side equals I.}

    The absolute convergence argument is sensible. We oscillate up and down,
but every time add a smaller and smaller magnitude number. (we're using
much less than 1/n, not 1, which is what makes this true. Every
multiplication also sees the addition of n of those multiplications, but
if everythign is less than 1/n, this is always a decrease.) Now what it
converges TO is heavily dependent on what exactly the terms ARE. Perhaps
we can set up an iteration to find out exactly what\ldots{} yeah no that
infinite sum kept increasing in dimensionality to absurd degrees. But it
does HAVE to converge since the entries are all very small.

Anyway with the small numbers approximation multiplying I+A times the
right side reduces to (I+A)(I-A) = I + A - A + smol. Small terms go
away, A cancels, I remains.

So this is at least accurate for very small entries, yes. However, the
\textless\textless{} 1/n is a definite requirement, otherwise the
matrices can and will grow in size.

    \emph{b) Use a) to establish 8.21 from 8.20}

    Oh this is so much easier than it seeemd. The inverse of I+A is just
I-A. Trivially true for 8.21 and 8.20, which are

8.20: \$ \Lambda\^{}\{\alpha'\}\emph{\beta = \delta\^{}\alpha}\beta +
\xi\^{}\alpha\_\{,\beta\} \$

8.21: \$ \Lambda\^{}\alpha\emph{\{\beta'\} = \delta\^{}\alpha}\beta -
\xi\^{}\alpha\_\{,\beta\} \$

    This naturally relies on the matrices being inverses of each other to
actually get this relation. Clearly they are, but why? The indeces don't
exactly match up quite right\ldots{} well, except it's all symmetric. I
is obviously symmetric, and

    \hypertarget{addendum-output-this-notebook-to-latex-formatted-pdf-file-back-to-top}{%
\section{\texorpdfstring{Addendum: Output this notebook to
\(\LaTeX\)-formatted PDF file {[}Back to
\hyperref[toc]{top}{]}}{Addendum: Output this notebook to \textbackslash LaTeX-formatted PDF file {[}Back to {]}}}\label{addendum-output-this-notebook-to-latex-formatted-pdf-file-back-to-top}}

\[\label{latex_pdf_output}\]

The following code cell converts this Jupyter notebook into a proper,
clickable \(\LaTeX\)-formatted PDF file. After the cell is successfully
run, the generated PDF may be found in the root NRPy+ tutorial
directory, with filename \url{GR-08.pdf} (Note that clicking on this
link may not work; you may need to open the PDF file through another
means.)

\textbf{Important Note}: Make sure that the file name is right in all
six locations, two here in the Markdown, four in the code below.

\begin{itemize}
\tightlist
\item
  GR-08.pdf
\item
  GR-08.ipynb
\item
  GR-08.tex
\end{itemize}

    \begin{tcolorbox}[breakable, size=fbox, boxrule=1pt, pad at break*=1mm,colback=cellbackground, colframe=cellborder]
\prompt{In}{incolor}{1}{\boxspacing}
\begin{Verbatim}[commandchars=\\\{\}]
\PY{k+kn}{import} \PY{n+nn}{cmdline\PYZus{}helper} \PY{k}{as} \PY{n+nn}{cmd}    \PY{c+c1}{\PYZsh{} NRPy+: Multi\PYZhy{}platform Python command\PYZhy{}line interface}
\PY{n}{cmd}\PY{o}{.}\PY{n}{output\PYZus{}Jupyter\PYZus{}notebook\PYZus{}to\PYZus{}LaTeXed\PYZus{}PDF}\PY{p}{(}\PY{l+s+s2}{\PYZdq{}}\PY{l+s+s2}{GR\PYZhy{}08}\PY{l+s+s2}{\PYZdq{}}\PY{p}{)}
\end{Verbatim}
\end{tcolorbox}

    \begin{Verbatim}[commandchars=\\\{\}]
Created GR-08.tex, and compiled LaTeX file to PDF file GR-08.pdf
    \end{Verbatim}

    \begin{tcolorbox}[breakable, size=fbox, boxrule=1pt, pad at break*=1mm,colback=cellbackground, colframe=cellborder]
\prompt{In}{incolor}{ }{\boxspacing}
\begin{Verbatim}[commandchars=\\\{\}]

\end{Verbatim}
\end{tcolorbox}

    \begin{tcolorbox}[breakable, size=fbox, boxrule=1pt, pad at break*=1mm,colback=cellbackground, colframe=cellborder]
\prompt{In}{incolor}{ }{\boxspacing}
\begin{Verbatim}[commandchars=\\\{\}]

\end{Verbatim}
\end{tcolorbox}


    % Add a bibliography block to the postdoc
    
    
    
\end{document}
