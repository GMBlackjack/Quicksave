% Based on http://nbviewer.jupyter.org/github/ipython/nbconvert-examples/blob/master/citations/Tutorial.ipynb , authored by Brian E. Granger
    % Declare the document class
    \documentclass[landscape,letterpaper,10pt,english]{article}


    \usepackage[breakable]{tcolorbox}
    \usepackage{parskip} % Stop auto-indenting (to mimic markdown behaviour)
    

    % Basic figure setup, for now with no caption control since it's done
    % automatically by Pandoc (which extracts ![](path) syntax from Markdown).
    \usepackage{graphicx}
    % Maintain compatibility with old templates. Remove in nbconvert 6.0
    \let\Oldincludegraphics\includegraphics
    % Ensure that by default, figures have no caption (until we provide a
    % proper Figure object with a Caption API and a way to capture that
    % in the conversion process - todo).
    \usepackage{caption}
    \DeclareCaptionFormat{nocaption}{}
    \captionsetup{format=nocaption,aboveskip=0pt,belowskip=0pt}

    \usepackage{float}
    \floatplacement{figure}{H} % forces figures to be placed at the correct location
    \usepackage{xcolor} % Allow colors to be defined
    \usepackage{enumerate} % Needed for markdown enumerations to work
    \usepackage{geometry} % Used to adjust the document margins
    \usepackage{amsmath} % Equations
    \usepackage{amssymb} % Equations
    \usepackage{textcomp} % defines textquotesingle
    % Hack from http://tex.stackexchange.com/a/47451/13684:
    \AtBeginDocument{%
        \def\PYZsq{\textquotesingle}% Upright quotes in Pygmentized code
    }
    \usepackage{upquote} % Upright quotes for verbatim code
    \usepackage{eurosym} % defines \euro

    \usepackage{iftex}
    \ifPDFTeX
        \usepackage[T1]{fontenc}
        \IfFileExists{alphabeta.sty}{
              \usepackage{alphabeta}
          }{
              \usepackage[mathletters]{ucs}
              \usepackage[utf8x]{inputenc}
          }
    \else
        \usepackage{fontspec}
        \usepackage{unicode-math}
    \fi

    \usepackage{fancyvrb} % verbatim replacement that allows latex
    \usepackage{grffile} % extends the file name processing of package graphics
                         % to support a larger range
    \makeatletter % fix for old versions of grffile with XeLaTeX
    \@ifpackagelater{grffile}{2019/11/01}
    {
      % Do nothing on new versions
    }
    {
      \def\Gread@@xetex#1{%
        \IfFileExists{"\Gin@base".bb}%
        {\Gread@eps{\Gin@base.bb}}%
        {\Gread@@xetex@aux#1}%
      }
    }
    \makeatother
    \usepackage[Export]{adjustbox} % Used to constrain images to a maximum size
    \adjustboxset{max size={0.9\linewidth}{0.9\paperheight}}

    % The hyperref package gives us a pdf with properly built
    % internal navigation ('pdf bookmarks' for the table of contents,
    % internal cross-reference links, web links for URLs, etc.)
    \usepackage{hyperref}
    % The default LaTeX title has an obnoxious amount of whitespace. By default,
    % titling removes some of it. It also provides customization options.
    \usepackage{titling}
    \usepackage{longtable} % longtable support required by pandoc >1.10
    \usepackage{booktabs}  % table support for pandoc > 1.12.2
    \usepackage{array}     % table support for pandoc >= 2.11.3
    \usepackage{calc}      % table minipage width calculation for pandoc >= 2.11.1
    \usepackage[inline]{enumitem} % IRkernel/repr support (it uses the enumerate* environment)
    \usepackage[normalem]{ulem} % ulem is needed to support strikethroughs (\sout)
                                % normalem makes italics be italics, not underlines
    \usepackage{mathrsfs}
    

    
    % Colors for the hyperref package
    \definecolor{urlcolor}{rgb}{0,.145,.698}
    \definecolor{linkcolor}{rgb}{.71,0.21,0.01}
    \definecolor{citecolor}{rgb}{.12,.54,.11}

    % ANSI colors
    \definecolor{ansi-black}{HTML}{3E424D}
    \definecolor{ansi-black-intense}{HTML}{282C36}
    \definecolor{ansi-red}{HTML}{E75C58}
    \definecolor{ansi-red-intense}{HTML}{B22B31}
    \definecolor{ansi-green}{HTML}{00A250}
    \definecolor{ansi-green-intense}{HTML}{007427}
    \definecolor{ansi-yellow}{HTML}{DDB62B}
    \definecolor{ansi-yellow-intense}{HTML}{B27D12}
    \definecolor{ansi-blue}{HTML}{208FFB}
    \definecolor{ansi-blue-intense}{HTML}{0065CA}
    \definecolor{ansi-magenta}{HTML}{D160C4}
    \definecolor{ansi-magenta-intense}{HTML}{A03196}
    \definecolor{ansi-cyan}{HTML}{60C6C8}
    \definecolor{ansi-cyan-intense}{HTML}{258F8F}
    \definecolor{ansi-white}{HTML}{C5C1B4}
    \definecolor{ansi-white-intense}{HTML}{A1A6B2}
    \definecolor{ansi-default-inverse-fg}{HTML}{FFFFFF}
    \definecolor{ansi-default-inverse-bg}{HTML}{000000}

    % common color for the border for error outputs.
    \definecolor{outerrorbackground}{HTML}{FFDFDF}

    % commands and environments needed by pandoc snippets
    % extracted from the output of `pandoc -s`
    \providecommand{\tightlist}{%
      \setlength{\itemsep}{0pt}\setlength{\parskip}{0pt}}
    \DefineVerbatimEnvironment{Highlighting}{Verbatim}{commandchars=\\\{\}}
    % Add ',fontsize=\small' for more characters per line
    \newenvironment{Shaded}{}{}
    \newcommand{\KeywordTok}[1]{\textcolor[rgb]{0.00,0.44,0.13}{\textbf{{#1}}}}
    \newcommand{\DataTypeTok}[1]{\textcolor[rgb]{0.56,0.13,0.00}{{#1}}}
    \newcommand{\DecValTok}[1]{\textcolor[rgb]{0.25,0.63,0.44}{{#1}}}
    \newcommand{\BaseNTok}[1]{\textcolor[rgb]{0.25,0.63,0.44}{{#1}}}
    \newcommand{\FloatTok}[1]{\textcolor[rgb]{0.25,0.63,0.44}{{#1}}}
    \newcommand{\CharTok}[1]{\textcolor[rgb]{0.25,0.44,0.63}{{#1}}}
    \newcommand{\StringTok}[1]{\textcolor[rgb]{0.25,0.44,0.63}{{#1}}}
    \newcommand{\CommentTok}[1]{\textcolor[rgb]{0.38,0.63,0.69}{\textit{{#1}}}}
    \newcommand{\OtherTok}[1]{\textcolor[rgb]{0.00,0.44,0.13}{{#1}}}
    \newcommand{\AlertTok}[1]{\textcolor[rgb]{1.00,0.00,0.00}{\textbf{{#1}}}}
    \newcommand{\FunctionTok}[1]{\textcolor[rgb]{0.02,0.16,0.49}{{#1}}}
    \newcommand{\RegionMarkerTok}[1]{{#1}}
    \newcommand{\ErrorTok}[1]{\textcolor[rgb]{1.00,0.00,0.00}{\textbf{{#1}}}}
    \newcommand{\NormalTok}[1]{{#1}}

    % Additional commands for more recent versions of Pandoc
    \newcommand{\ConstantTok}[1]{\textcolor[rgb]{0.53,0.00,0.00}{{#1}}}
    \newcommand{\SpecialCharTok}[1]{\textcolor[rgb]{0.25,0.44,0.63}{{#1}}}
    \newcommand{\VerbatimStringTok}[1]{\textcolor[rgb]{0.25,0.44,0.63}{{#1}}}
    \newcommand{\SpecialStringTok}[1]{\textcolor[rgb]{0.73,0.40,0.53}{{#1}}}
    \newcommand{\ImportTok}[1]{{#1}}
    \newcommand{\DocumentationTok}[1]{\textcolor[rgb]{0.73,0.13,0.13}{\textit{{#1}}}}
    \newcommand{\AnnotationTok}[1]{\textcolor[rgb]{0.38,0.63,0.69}{\textbf{\textit{{#1}}}}}
    \newcommand{\CommentVarTok}[1]{\textcolor[rgb]{0.38,0.63,0.69}{\textbf{\textit{{#1}}}}}
    \newcommand{\VariableTok}[1]{\textcolor[rgb]{0.10,0.09,0.49}{{#1}}}
    \newcommand{\ControlFlowTok}[1]{\textcolor[rgb]{0.00,0.44,0.13}{\textbf{{#1}}}}
    \newcommand{\OperatorTok}[1]{\textcolor[rgb]{0.40,0.40,0.40}{{#1}}}
    \newcommand{\BuiltInTok}[1]{{#1}}
    \newcommand{\ExtensionTok}[1]{{#1}}
    \newcommand{\PreprocessorTok}[1]{\textcolor[rgb]{0.74,0.48,0.00}{{#1}}}
    \newcommand{\AttributeTok}[1]{\textcolor[rgb]{0.49,0.56,0.16}{{#1}}}
    \newcommand{\InformationTok}[1]{\textcolor[rgb]{0.38,0.63,0.69}{\textbf{\textit{{#1}}}}}
    \newcommand{\WarningTok}[1]{\textcolor[rgb]{0.38,0.63,0.69}{\textbf{\textit{{#1}}}}}


    % Define a nice break command that doesn't care if a line doesn't already
    % exist.
    \def\br{\hspace*{\fill} \\* }
    % Math Jax compatibility definitions
    \def\gt{>}
    \def\lt{<}
    \let\Oldtex\TeX
    \let\Oldlatex\LaTeX
    \renewcommand{\TeX}{\textrm{\Oldtex}}
    \renewcommand{\LaTeX}{\textrm{\Oldlatex}}
    % Document parameters
    % Document title
    \title{GR-08}
    
    
    
    
    
% Pygments definitions
\makeatletter
\def\PY@reset{\let\PY@it=\relax \let\PY@bf=\relax%
    \let\PY@ul=\relax \let\PY@tc=\relax%
    \let\PY@bc=\relax \let\PY@ff=\relax}
\def\PY@tok#1{\csname PY@tok@#1\endcsname}
\def\PY@toks#1+{\ifx\relax#1\empty\else%
    \PY@tok{#1}\expandafter\PY@toks\fi}
\def\PY@do#1{\PY@bc{\PY@tc{\PY@ul{%
    \PY@it{\PY@bf{\PY@ff{#1}}}}}}}
\def\PY#1#2{\PY@reset\PY@toks#1+\relax+\PY@do{#2}}

\@namedef{PY@tok@w}{\def\PY@tc##1{\textcolor[rgb]{0.73,0.73,0.73}{##1}}}
\@namedef{PY@tok@c}{\let\PY@it=\textit\def\PY@tc##1{\textcolor[rgb]{0.24,0.48,0.48}{##1}}}
\@namedef{PY@tok@cp}{\def\PY@tc##1{\textcolor[rgb]{0.61,0.40,0.00}{##1}}}
\@namedef{PY@tok@k}{\let\PY@bf=\textbf\def\PY@tc##1{\textcolor[rgb]{0.00,0.50,0.00}{##1}}}
\@namedef{PY@tok@kp}{\def\PY@tc##1{\textcolor[rgb]{0.00,0.50,0.00}{##1}}}
\@namedef{PY@tok@kt}{\def\PY@tc##1{\textcolor[rgb]{0.69,0.00,0.25}{##1}}}
\@namedef{PY@tok@o}{\def\PY@tc##1{\textcolor[rgb]{0.40,0.40,0.40}{##1}}}
\@namedef{PY@tok@ow}{\let\PY@bf=\textbf\def\PY@tc##1{\textcolor[rgb]{0.67,0.13,1.00}{##1}}}
\@namedef{PY@tok@nb}{\def\PY@tc##1{\textcolor[rgb]{0.00,0.50,0.00}{##1}}}
\@namedef{PY@tok@nf}{\def\PY@tc##1{\textcolor[rgb]{0.00,0.00,1.00}{##1}}}
\@namedef{PY@tok@nc}{\let\PY@bf=\textbf\def\PY@tc##1{\textcolor[rgb]{0.00,0.00,1.00}{##1}}}
\@namedef{PY@tok@nn}{\let\PY@bf=\textbf\def\PY@tc##1{\textcolor[rgb]{0.00,0.00,1.00}{##1}}}
\@namedef{PY@tok@ne}{\let\PY@bf=\textbf\def\PY@tc##1{\textcolor[rgb]{0.80,0.25,0.22}{##1}}}
\@namedef{PY@tok@nv}{\def\PY@tc##1{\textcolor[rgb]{0.10,0.09,0.49}{##1}}}
\@namedef{PY@tok@no}{\def\PY@tc##1{\textcolor[rgb]{0.53,0.00,0.00}{##1}}}
\@namedef{PY@tok@nl}{\def\PY@tc##1{\textcolor[rgb]{0.46,0.46,0.00}{##1}}}
\@namedef{PY@tok@ni}{\let\PY@bf=\textbf\def\PY@tc##1{\textcolor[rgb]{0.44,0.44,0.44}{##1}}}
\@namedef{PY@tok@na}{\def\PY@tc##1{\textcolor[rgb]{0.41,0.47,0.13}{##1}}}
\@namedef{PY@tok@nt}{\let\PY@bf=\textbf\def\PY@tc##1{\textcolor[rgb]{0.00,0.50,0.00}{##1}}}
\@namedef{PY@tok@nd}{\def\PY@tc##1{\textcolor[rgb]{0.67,0.13,1.00}{##1}}}
\@namedef{PY@tok@s}{\def\PY@tc##1{\textcolor[rgb]{0.73,0.13,0.13}{##1}}}
\@namedef{PY@tok@sd}{\let\PY@it=\textit\def\PY@tc##1{\textcolor[rgb]{0.73,0.13,0.13}{##1}}}
\@namedef{PY@tok@si}{\let\PY@bf=\textbf\def\PY@tc##1{\textcolor[rgb]{0.64,0.35,0.47}{##1}}}
\@namedef{PY@tok@se}{\let\PY@bf=\textbf\def\PY@tc##1{\textcolor[rgb]{0.67,0.36,0.12}{##1}}}
\@namedef{PY@tok@sr}{\def\PY@tc##1{\textcolor[rgb]{0.64,0.35,0.47}{##1}}}
\@namedef{PY@tok@ss}{\def\PY@tc##1{\textcolor[rgb]{0.10,0.09,0.49}{##1}}}
\@namedef{PY@tok@sx}{\def\PY@tc##1{\textcolor[rgb]{0.00,0.50,0.00}{##1}}}
\@namedef{PY@tok@m}{\def\PY@tc##1{\textcolor[rgb]{0.40,0.40,0.40}{##1}}}
\@namedef{PY@tok@gh}{\let\PY@bf=\textbf\def\PY@tc##1{\textcolor[rgb]{0.00,0.00,0.50}{##1}}}
\@namedef{PY@tok@gu}{\let\PY@bf=\textbf\def\PY@tc##1{\textcolor[rgb]{0.50,0.00,0.50}{##1}}}
\@namedef{PY@tok@gd}{\def\PY@tc##1{\textcolor[rgb]{0.63,0.00,0.00}{##1}}}
\@namedef{PY@tok@gi}{\def\PY@tc##1{\textcolor[rgb]{0.00,0.52,0.00}{##1}}}
\@namedef{PY@tok@gr}{\def\PY@tc##1{\textcolor[rgb]{0.89,0.00,0.00}{##1}}}
\@namedef{PY@tok@ge}{\let\PY@it=\textit}
\@namedef{PY@tok@gs}{\let\PY@bf=\textbf}
\@namedef{PY@tok@gp}{\let\PY@bf=\textbf\def\PY@tc##1{\textcolor[rgb]{0.00,0.00,0.50}{##1}}}
\@namedef{PY@tok@go}{\def\PY@tc##1{\textcolor[rgb]{0.44,0.44,0.44}{##1}}}
\@namedef{PY@tok@gt}{\def\PY@tc##1{\textcolor[rgb]{0.00,0.27,0.87}{##1}}}
\@namedef{PY@tok@err}{\def\PY@bc##1{{\setlength{\fboxsep}{\string -\fboxrule}\fcolorbox[rgb]{1.00,0.00,0.00}{1,1,1}{\strut ##1}}}}
\@namedef{PY@tok@kc}{\let\PY@bf=\textbf\def\PY@tc##1{\textcolor[rgb]{0.00,0.50,0.00}{##1}}}
\@namedef{PY@tok@kd}{\let\PY@bf=\textbf\def\PY@tc##1{\textcolor[rgb]{0.00,0.50,0.00}{##1}}}
\@namedef{PY@tok@kn}{\let\PY@bf=\textbf\def\PY@tc##1{\textcolor[rgb]{0.00,0.50,0.00}{##1}}}
\@namedef{PY@tok@kr}{\let\PY@bf=\textbf\def\PY@tc##1{\textcolor[rgb]{0.00,0.50,0.00}{##1}}}
\@namedef{PY@tok@bp}{\def\PY@tc##1{\textcolor[rgb]{0.00,0.50,0.00}{##1}}}
\@namedef{PY@tok@fm}{\def\PY@tc##1{\textcolor[rgb]{0.00,0.00,1.00}{##1}}}
\@namedef{PY@tok@vc}{\def\PY@tc##1{\textcolor[rgb]{0.10,0.09,0.49}{##1}}}
\@namedef{PY@tok@vg}{\def\PY@tc##1{\textcolor[rgb]{0.10,0.09,0.49}{##1}}}
\@namedef{PY@tok@vi}{\def\PY@tc##1{\textcolor[rgb]{0.10,0.09,0.49}{##1}}}
\@namedef{PY@tok@vm}{\def\PY@tc##1{\textcolor[rgb]{0.10,0.09,0.49}{##1}}}
\@namedef{PY@tok@sa}{\def\PY@tc##1{\textcolor[rgb]{0.73,0.13,0.13}{##1}}}
\@namedef{PY@tok@sb}{\def\PY@tc##1{\textcolor[rgb]{0.73,0.13,0.13}{##1}}}
\@namedef{PY@tok@sc}{\def\PY@tc##1{\textcolor[rgb]{0.73,0.13,0.13}{##1}}}
\@namedef{PY@tok@dl}{\def\PY@tc##1{\textcolor[rgb]{0.73,0.13,0.13}{##1}}}
\@namedef{PY@tok@s2}{\def\PY@tc##1{\textcolor[rgb]{0.73,0.13,0.13}{##1}}}
\@namedef{PY@tok@sh}{\def\PY@tc##1{\textcolor[rgb]{0.73,0.13,0.13}{##1}}}
\@namedef{PY@tok@s1}{\def\PY@tc##1{\textcolor[rgb]{0.73,0.13,0.13}{##1}}}
\@namedef{PY@tok@mb}{\def\PY@tc##1{\textcolor[rgb]{0.40,0.40,0.40}{##1}}}
\@namedef{PY@tok@mf}{\def\PY@tc##1{\textcolor[rgb]{0.40,0.40,0.40}{##1}}}
\@namedef{PY@tok@mh}{\def\PY@tc##1{\textcolor[rgb]{0.40,0.40,0.40}{##1}}}
\@namedef{PY@tok@mi}{\def\PY@tc##1{\textcolor[rgb]{0.40,0.40,0.40}{##1}}}
\@namedef{PY@tok@il}{\def\PY@tc##1{\textcolor[rgb]{0.40,0.40,0.40}{##1}}}
\@namedef{PY@tok@mo}{\def\PY@tc##1{\textcolor[rgb]{0.40,0.40,0.40}{##1}}}
\@namedef{PY@tok@ch}{\let\PY@it=\textit\def\PY@tc##1{\textcolor[rgb]{0.24,0.48,0.48}{##1}}}
\@namedef{PY@tok@cm}{\let\PY@it=\textit\def\PY@tc##1{\textcolor[rgb]{0.24,0.48,0.48}{##1}}}
\@namedef{PY@tok@cpf}{\let\PY@it=\textit\def\PY@tc##1{\textcolor[rgb]{0.24,0.48,0.48}{##1}}}
\@namedef{PY@tok@c1}{\let\PY@it=\textit\def\PY@tc##1{\textcolor[rgb]{0.24,0.48,0.48}{##1}}}
\@namedef{PY@tok@cs}{\let\PY@it=\textit\def\PY@tc##1{\textcolor[rgb]{0.24,0.48,0.48}{##1}}}

\def\PYZbs{\char`\\}
\def\PYZus{\char`\_}
\def\PYZob{\char`\{}
\def\PYZcb{\char`\}}
\def\PYZca{\char`\^}
\def\PYZam{\char`\&}
\def\PYZlt{\char`\<}
\def\PYZgt{\char`\>}
\def\PYZsh{\char`\#}
\def\PYZpc{\char`\%}
\def\PYZdl{\char`\$}
\def\PYZhy{\char`\-}
\def\PYZsq{\char`\'}
\def\PYZdq{\char`\"}
\def\PYZti{\char`\~}
% for compatibility with earlier versions
\def\PYZat{@}
\def\PYZlb{[}
\def\PYZrb{]}
\makeatother


    % For linebreaks inside Verbatim environment from package fancyvrb.
    \makeatletter
        \newbox\Wrappedcontinuationbox
        \newbox\Wrappedvisiblespacebox
        \newcommand*\Wrappedvisiblespace {\textcolor{red}{\textvisiblespace}}
        \newcommand*\Wrappedcontinuationsymbol {\textcolor{red}{\llap{\tiny$\m@th\hookrightarrow$}}}
        \newcommand*\Wrappedcontinuationindent {3ex }
        \newcommand*\Wrappedafterbreak {\kern\Wrappedcontinuationindent\copy\Wrappedcontinuationbox}
        % Take advantage of the already applied Pygments mark-up to insert
        % potential linebreaks for TeX processing.
        %        {, <, #, %, $, ' and ": go to next line.
        %        _, }, ^, &, >, - and ~: stay at end of broken line.
        % Use of \textquotesingle for straight quote.
        \newcommand*\Wrappedbreaksatspecials {%
            \def\PYGZus{\discretionary{\char`\_}{\Wrappedafterbreak}{\char`\_}}%
            \def\PYGZob{\discretionary{}{\Wrappedafterbreak\char`\{}{\char`\{}}%
            \def\PYGZcb{\discretionary{\char`\}}{\Wrappedafterbreak}{\char`\}}}%
            \def\PYGZca{\discretionary{\char`\^}{\Wrappedafterbreak}{\char`\^}}%
            \def\PYGZam{\discretionary{\char`\&}{\Wrappedafterbreak}{\char`\&}}%
            \def\PYGZlt{\discretionary{}{\Wrappedafterbreak\char`\<}{\char`\<}}%
            \def\PYGZgt{\discretionary{\char`\>}{\Wrappedafterbreak}{\char`\>}}%
            \def\PYGZsh{\discretionary{}{\Wrappedafterbreak\char`\#}{\char`\#}}%
            \def\PYGZpc{\discretionary{}{\Wrappedafterbreak\char`\%}{\char`\%}}%
            \def\PYGZdl{\discretionary{}{\Wrappedafterbreak\char`\$}{\char`\$}}%
            \def\PYGZhy{\discretionary{\char`\-}{\Wrappedafterbreak}{\char`\-}}%
            \def\PYGZsq{\discretionary{}{\Wrappedafterbreak\textquotesingle}{\textquotesingle}}%
            \def\PYGZdq{\discretionary{}{\Wrappedafterbreak\char`\"}{\char`\"}}%
            \def\PYGZti{\discretionary{\char`\~}{\Wrappedafterbreak}{\char`\~}}%
        }
        % Some characters . , ; ? ! / are not pygmentized.
        % This macro makes them "active" and they will insert potential linebreaks
        \newcommand*\Wrappedbreaksatpunct {%
            \lccode`\~`\.\lowercase{\def~}{\discretionary{\hbox{\char`\.}}{\Wrappedafterbreak}{\hbox{\char`\.}}}%
            \lccode`\~`\,\lowercase{\def~}{\discretionary{\hbox{\char`\,}}{\Wrappedafterbreak}{\hbox{\char`\,}}}%
            \lccode`\~`\;\lowercase{\def~}{\discretionary{\hbox{\char`\;}}{\Wrappedafterbreak}{\hbox{\char`\;}}}%
            \lccode`\~`\:\lowercase{\def~}{\discretionary{\hbox{\char`\:}}{\Wrappedafterbreak}{\hbox{\char`\:}}}%
            \lccode`\~`\?\lowercase{\def~}{\discretionary{\hbox{\char`\?}}{\Wrappedafterbreak}{\hbox{\char`\?}}}%
            \lccode`\~`\!\lowercase{\def~}{\discretionary{\hbox{\char`\!}}{\Wrappedafterbreak}{\hbox{\char`\!}}}%
            \lccode`\~`\/\lowercase{\def~}{\discretionary{\hbox{\char`\/}}{\Wrappedafterbreak}{\hbox{\char`\/}}}%
            \catcode`\.\active
            \catcode`\,\active
            \catcode`\;\active
            \catcode`\:\active
            \catcode`\?\active
            \catcode`\!\active
            \catcode`\/\active
            \lccode`\~`\~
        }
    \makeatother

    \let\OriginalVerbatim=\Verbatim
    \makeatletter
    \renewcommand{\Verbatim}[1][1]{%
        %\parskip\z@skip
        \sbox\Wrappedcontinuationbox {\Wrappedcontinuationsymbol}%
        \sbox\Wrappedvisiblespacebox {\FV@SetupFont\Wrappedvisiblespace}%
        \def\FancyVerbFormatLine ##1{\hsize\linewidth
            \vtop{\raggedright\hyphenpenalty\z@\exhyphenpenalty\z@
                \doublehyphendemerits\z@\finalhyphendemerits\z@
                \strut ##1\strut}%
        }%
        % If the linebreak is at a space, the latter will be displayed as visible
        % space at end of first line, and a continuation symbol starts next line.
        % Stretch/shrink are however usually zero for typewriter font.
        \def\FV@Space {%
            \nobreak\hskip\z@ plus\fontdimen3\font minus\fontdimen4\font
            \discretionary{\copy\Wrappedvisiblespacebox}{\Wrappedafterbreak}
            {\kern\fontdimen2\font}%
        }%

        % Allow breaks at special characters using \PYG... macros.
        \Wrappedbreaksatspecials
        % Breaks at punctuation characters . , ; ? ! and / need catcode=\active
        \OriginalVerbatim[#1,codes*=\Wrappedbreaksatpunct]%
    }
    \makeatother

    % Exact colors from NB
    \definecolor{incolor}{HTML}{303F9F}
    \definecolor{outcolor}{HTML}{D84315}
    \definecolor{cellborder}{HTML}{CFCFCF}
    \definecolor{cellbackground}{HTML}{F7F7F7}

    % prompt
    \makeatletter
    \newcommand{\boxspacing}{\kern\kvtcb@left@rule\kern\kvtcb@boxsep}
    \makeatother
    \newcommand{\prompt}[4]{
        {\ttfamily\llap{{\color{#2}[#3]:\hspace{3pt}#4}}\vspace{-\baselineskip}}
    }
    

    
% Start the section counter at -1, so the Table of Contents is Section 0
   \setcounter{section}{-2}
% Prevent overflowing lines due to hard-to-break entities
    \sloppy
    % Setup hyperref package
    \hypersetup{
      breaklinks=true,  % so long urls are correctly broken across lines
      colorlinks=true,
      urlcolor=urlcolor,
      linkcolor=linkcolor,
      citecolor=citecolor,
      }

    % Slightly bigger margins than the latex defaults
    \geometry{verbose,tmargin=0.5in,bmargin=0.5in,lmargin=0.5in,rmargin=0.5in}


\begin{document}
    
    \maketitle
    
    

    
    \hypertarget{general-relativity-problems-chapter-7-physics-in-a-curved-spacetime}{%
\section{General Relativity Problems Chapter 7: Physics in a Curved
Spacetime}\label{general-relativity-problems-chapter-7-physics-in-a-curved-spacetime}}

\hypertarget{authors-gabriel-m-steward}{%
\subsection{Authors: Gabriel M
Steward}\label{authors-gabriel-m-steward}}

    https://github.com/zachetienne/nrpytutorial/blob/master/Tutorial-Template\_Style\_Guide.ipynb

Link to the Style Guide. Not internal in case something breaks.

    \hypertarget{nrpy-source-code-for-this-module}{%
\subsubsection{\texorpdfstring{ NRPy+ Source Code for this
module:}{ NRPy+ Source Code for this module:}}\label{nrpy-source-code-for-this-module}}

None!

\hypertarget{introduction}{%
\subsection{Introduction:}\label{introduction}}

Now maybe we can apply what we've learned to some actual physical
problems. Maybe. One can hope.

\hypertarget{other-optional}{%
\subsection{\texorpdfstring{ Other
(Optional):}{ Other (Optional):}}\label{other-optional}}

Placeholder.

\hypertarget{note-on-notation}{%
\subsubsection{Note on Notation:}\label{note-on-notation}}

Any new notation will be brought up in the notebook when it becomes
relevant.

\hypertarget{citations}{%
\subsubsection{Citations:}\label{citations}}

{[}1{]}
https://www.wolframalpha.com/input?i=y\%27\%28x\%29+\%3D+cosx+sinx+y+-+y\%5C\%28tanx\%29+-+1
(Wolfram likes differential equations.)

    \hypertarget{table-of-contents}{%
\section{Table of Contents}\label{table-of-contents}}

\[\label{toc}\]

\hyperref[p1]{Problem 1} (A thought experiment)

\hyperref[p2]{Problem 2} (Newtonian Metric)

\hyperref[p3]{Problem 3} (Newtonian Metric Christoffel Symbols)

\hyperref[p4]{Problem 4} (g00)

\hyperref[p5]{Problem 5} (The Static Fluid Problem, incomplete)

\hyperref[p6]{Problem 6} (Geodesic Momentum)

\hyperref[p7]{Problem 7} (Playing with Complicated Metrics and Their
Momentums)

\hyperref[p8]{Problem 8} (Independent Components, Stress-Energy Tensors,
and Integrals, incomplete)

\hyperref[p9]{Problem 9} (Calculating R and R properties, largely
skipped.)

\hyperref[p10]{Problem 10} (\textbf{KILLING} vector fields.)

\hyperref[latex_pdf_output]{PDF} (turn this into a PDF)

    \hypertarget{problem-1-back-to-top}{%
\section{\texorpdfstring{Problem 1 {[}Back to
\hyperref[toc]{top}{]}}{Problem 1 {[}Back to {]}}}\label{problem-1-back-to-top}}

\[\label{P1}\]

\emph{If 7.3 were the corret genrealization of 7.1 to a curved
spacetime, how would you interpret it? What would happen to the number
of particles in a comoving volume of the fluid, as time evolves? In
principle, can we distinguish experimetnally between 7.2 and 7.3?}

    7.1: \((nU^\alpha)_{,\alpha}=0\)

7.2: \((nU^\alpha)_{;\alpha}=0\)

7.3: \((nU^\alpha)_{;\alpha}=qR^2\)

The exact value of qR\^{}2 is irrelevant, besides the fact that it
vanishes in a Lorentz frame as R vanishes there.

\(nU^\alpha\) is the particle density times its actual ``speed'' at any
given reference frame. We know this quantity to be N. The first
component is the number density, the rest are the flux.

7.1 says many things, but one of the things it says is that the flux
values are constants with respect to all coordinates: they do not
change. (for something of constant density everywher,e which is what n
is). No matter where we are or when we are, everything is the same.

7.2 asserts this is true no matter your reference frame or curvature.

7.3, on the other hand, disagrees, saying that the fluxes do change with
curvature. This means that while locally it may seem that everything is
conserved, globally it is not, the values of the fluxes could be
different at large differences. Also the number density could change.

The number density is the big thing. Elsewhere, we have a different
density. This means there have to be \emph{more particles} over there.
But we established a constant density, so curvature must be
\emph{creating} particles.

This seems like nonsense, to be sure.

Depending on the curvature, the number of particles would go up or down
(we don't know the sign of q). R, however, is squared and always
positive. So this means curvature can only CREATE or DESTROY particles,
not both. There would be no ``going back'' afterward.

Which means we can definitely test it. Just gather a bunch of particles
and accelerate them to near light speeds and then count them along the
whole trip.

((we understand that virtual particles throw a wrench into this, but
that usually keeps particle number conserved over large times, so just
consider them noise.))

At this juncture all experiments show that particles are in fact
conserved. As well as all the other conservation laws. Still, though,
the gradual incerasing or decreasing of particles would be rather
interesting. Since it can't go back down or up, though, it sure seems
like that would violate energy conservation\ldots{} and entropy\ldots{}
and a whole slew of other things.

Now if there were an alternative 7.3 that could take both positive and
negative values, then that would be a different story. And something
like this kind of has to be true, as at any given moment particle number
in the universe is not conserved. The aforementioned virtual particles
at black hole horizons kind of throw a wrench into this. (Though that
theory would involve interactions between particles, which we are
ignoring right now, and is probably a million times more complicated
than we realize.)

    \hypertarget{problem-2-back-to-top}{%
\section{\texorpdfstring{Problem 2 {[}Back to
\hyperref[toc]{top}{]}}{Problem 2 {[}Back to {]}}}\label{problem-2-back-to-top}}

\[\label{P2}\]

\emph{To first order in \(\phi\), compute \(g^{\alpha\beta}\) for 7.8}

    Ah, 7.8! The ``Newtonian'' metric! 7.8 is rather helpful but we'll write
it out for reference:

\[ ds^2 = -(1+2\phi)dt^2 + (1-2\phi)(dx^2+dy^2+dz^2)  \]

Which gives us a metric with diagonal

\[(-(1+2\phi),1-2\phi,1-2\phi,1-2\phi)\]

Now we could calculat the inverse metric the long way\ldots{} or we
could just find this matrix's inverse that returns us to the (-1,1,1,1)
matrix. Which is actually rather easy, just invert everything.

\[(-\frac{1}{(1+2\phi)},\frac{1}{1-2\phi},\frac{1}{1-2\phi},\frac{1}{1-2\phi})\]

NOTE: The metrics are inverses of each other, they form the identity,
NOT the Lorentz metricwhen together. They do convert back and forth
between the Lorentz metric, but that's when STARTING from the Lorentz
metric. Be careful.

    \hypertarget{problem-3-back-to-top}{%
\section{\texorpdfstring{Problem 3 {[}Back to
\hyperref[toc]{top}{]}}{Problem 3 {[}Back to {]}}}\label{problem-3-back-to-top}}

\[\label{P3}\]

\emph{Calculate all the Christoffel symbols for the metric given by 7.8
to first order in \(\phi\). Assume \(\phi\) is a general function of
t,x,y,z.}

Ah, see, this is more challenging, since now we've got to take
derivatives of the metric we declared in \textbf{Problem 2} and since we
know nothing about the actual form of \(\phi\) this could get quite
ugly. Fortunately we can ignore anything squared or higher in our
approximation, which is neat.

Anyway, we need derivatives of the metric with respect to everything.
This is:

\[(-2\phi_{,t},-2\phi_{,x},-2\phi_{,y},-2\phi_{,z})\]

It actually doesn't matter which element is being used, they're all the
same in terms of derivative, bizarrely.

And fortunately for us we don't have to calculate derivatives of the
inverse metric. Yet.

The general formula for the Christoffel symbols are:

\[ \Gamma^\gamma_{\beta\mu} = \frac{1}{2} g^{\alpha\gamma} (g_{\alpha\beta,\mu} + g_{\alpha\mu,\beta} - g_{\beta\mu,\alpha})  \]

Which means we'll have\ldots{} a 4x4x4 result. This is gonna be a spicy
one.

    Fortunately the vast majority of these are just going to be zero because
the metric has a lot of zeroes. We only need concern ourselves with the
situations where the three inner metrics aren't zero, that is, their
indeces match. Unfortunately there are still a lot of those. This time
the bottom coordinates \(\beta\mu\) represent the row/column while the
top coordinate \(\gamma\) is the matrix itself. Coordinates are in order
(txyz) or (0123) depending on what system is being used.

\[ \Gamma^{\gamma}_{\beta\mu} = \begin{bmatrix}
\frac{\phi_{,t}}{1+2\phi} & \frac{\phi_{,x}}{1+2\phi} & \frac{\phi_{,y}}{1+2\phi} & \frac{\phi_{,z}}{1+2\phi} \\ 
\frac{\phi_{,x}}{1+2\phi} & -\frac{\phi_{,t}}{1+2\phi} & 0 & 0 \\
\frac{\phi_{,y}}{1+2\phi} & 0 & -\frac{\phi_{,t}}{1+2\phi} & 0 \\
\frac{\phi_{,z}}{1+2\phi} & 0 & 0 & -\frac{\phi_{,t}}{1+2\phi}
\end{bmatrix},\begin{bmatrix}
\frac{\phi_{,x}}{1-2\phi} & -\frac{\phi_{,t}}{1-2\phi} & 0 & 0 \\ 
-\frac{\phi_{,t}}{1-2\phi} & -\frac{\phi_{,x}}{1-2\phi} & -\frac{\phi_{,y}}{1-2\phi} & -\frac{\phi_{,z}}{1-2\phi} \\
0 & -\frac{\phi_{,y}}{1-2\phi} & \frac{\phi_{,x}}{1-2\phi} & 0 \\
0 & -\frac{\phi_{,z}}{1-2\phi} & 0 & \frac{\phi_{,x}}{1-2\phi}
\end{bmatrix},\begin{bmatrix}
\frac{\phi_{,y}}{1-2\phi} & 0 & -\frac{\phi_{,t}}{1-2\phi} & 0 \\ 
0 & \frac{\phi_{,y}}{1-2\phi} & -\frac{\phi_{,x}}{1-2\phi} & 0 \\
-\frac{\phi_{,t}}{1-2\phi} & -\frac{\phi_{,x}}{1-2\phi} & -\frac{\phi_{,y}}{1-2\phi} & -\frac{\phi_{,z}}{1-2\phi} \\
0 & 0 & -\frac{\phi_{,z}}{1-2\phi} & \frac{\phi_{,y}}{1-2\phi}
\end{bmatrix},\begin{bmatrix}
\frac{\phi_{,z}}{1-2\phi} & 0 & 0 & -\frac{\phi_{,t}}{1-2\phi} \\ 
0 & \frac{\phi_{,z}}{1-2\phi} & 0 & -\frac{\phi_{,x}}{1-2\phi} \\
0 & 0 & \frac{\phi_{,z}}{1-2\phi} & -\frac{\phi_{,y}}{1-2\phi} \\
-\frac{\phi_{,t}}{1-2\phi} & -\frac{\phi_{,x}}{1-2\phi} & -\frac{\phi_{,y}}{1-2\phi} & -\frac{\phi_{,z}}{1-2\phi}
\end{bmatrix}
\]

    And there they all. All of them. We suspect that in a known field many
of the derivatives will vanish.

Apparently the book considers ``to first order'' approximations to
include removing the fraction since phi is small. Which is fine, 1
dominates, but methinks that's not a very strict definition of ``to
first order.''

\[ \Gamma^{\gamma}_{\beta\mu} = \begin{bmatrix}
\phi_{,t} & \phi_{,x} & \phi_{,y} & \phi_{,z} \\ 
\phi_{,x} & -\phi_{,t} & 0 & 0 \\
\phi_{,y} & 0 & -\phi_{,t} & 0 \\
\phi_{,z} & 0 & 0 & -\phi_{,t}
\end{bmatrix},\begin{bmatrix}
\phi_{,x} & -\phi_{,t} & 0 & 0 \\ 
-\phi_{,t} & -\phi_{,x} & -\phi_{,y} & -\phi_{,z} \\
0 & -\phi_{,y} & \phi_{,x} & 0 \\
0 & -\phi_{,z} & 0 & \phi_{,x}
\end{bmatrix},\begin{bmatrix}
\phi_{,y} & 0 & -\phi_{,t} & 0 \\ 
0 & \phi_{,y} & -\phi_{,x} & 0 \\
-\phi_{,t} & -\phi_{,x} & -\phi_{,y} & -\phi_{,z} \\
0 & 0 & -\phi_{,z} & \phi_{,y}
\end{bmatrix},\begin{bmatrix}
\phi_{,z} & 0 & 0 & -\phi_{,t} \\ 
0 & \phi_{,z} & 0 & -\phi_{,x} \\
0 & 0 & \phi_{,z} & -\phi_{,y} \\
-\phi_{,t} & -\phi_{,x} & -\phi_{,y} & -\phi_{,z}
\end{bmatrix}
\]

    Random factoid: determinant g is equal to \(1-8\phi\).

    \hypertarget{problem-4-back-to-top}{%
\section{\texorpdfstring{Problem 4 {[}Back to
\hyperref[toc]{top}{]}}{Problem 4 {[}Back to {]}}}\label{problem-4-back-to-top}}

\[\label{P4}\]

\emph{Verify that hte results 7.15 and 7.24 depend only on \(g_{00}\),
the form of \(g_{xx}\) doesn't affect them, as long as it is
1+O(\(\phi\)).}

    This is essentially the same as asking for the dependence of the
Christoffel symbols. For instance\ldots{}

7.15: \(\frac{d}{d\tau} p^0 = -m \frac{\partial \phi}{\partial \tau}\)

This was derived from a Christoffel symbol of \(\Gamma^0_00\). As we saw
in \textbf{Problem 3}, this is calculated from

\[ \Gamma^\gamma_{\beta\mu} = \frac{1}{2} g^{\alpha\gamma} (g_{\alpha\beta,\mu} + g_{\alpha\mu,\beta} - g_{\beta\mu,\alpha})  \]

\[ \Rightarrow \Gamma^0_{0} = \frac{1}{2} g^{00} (g_{00,0} + g_{00,0} - g_{00,0})  \]

Which obviously only includes the metric's 00. (The inverse metric can
be calculated from it directly).

7.24 relies instead on \(\Gamma^i_{00}\). It's rather self evident that
the normal g-metrics all cancel unless they are 00. HOWEVER\ldots{} the
inverse metric \emph{does} matter. Now, its term cannot be derived from
\(g_00\), and is in fact the inverse of whatever xx, yy, zz. While the
inverse metric does appear to vanish, it still carries with it hte
\emph{sign} of those components.

In fact this is even directly shown in equation 7.20, where the inverse
metric is calculated. It provides a positive sign, while for the t-value
it would be negative.

\ldots That said in the actual limit we're discussion, that of the first
order, then we're actually fine, seeing as \emph{all} the matrix values
an be argued to depend only on \(g_00\) as all their values are
correlated to derivatives of it. But that seems to be more a trick of
fate than anything.

    \hypertarget{problem-5-back-to-top}{%
\section{\texorpdfstring{Problem 5 {[}Back to
\hyperref[toc]{top}{]}}{Problem 5 {[}Back to {]}}}\label{problem-5-back-to-top}}

\[\label{P5}\]

\emph{a) For a perfect fluid, verify that the spatial components of 7.6
in the Newtonian limit reduce to
\(\textbf{v}_{,t} + (\textbf{v}\cdot \nabla)\textbf{v} + \nabla p/\rho + \nabla \phi = 0\)
for the metric 7.8. This is known as Euler's equation for
nonrelativistic fluid flow in a gravitaitonal field. You will need to
use 7.2 to get this result.}

    7.6 is the statement of the conservation of four-monentum, given by

\[ T^{\mu\nu}_{;\nu}=0 \]

\[ T^{\mu\nu} = (\rho + p)U^\mu U^\nu + pg^{\mu\nu} \]

A hint on page 177 points out that in the nonrelativistic limit, v has
to be small, and p \textless\textless{} \(\rho\) since the individual
constituent particles also have to be moving slowly in relation to each
other.

7.2 is \((nU^\alpha)_{;\alpha}=0\)

Let's not assume that n is a constant, at least not at first.

\[ n_{;\alpha}U^\alpha + nU^\alpha_{;\alpha} =0 \]

For this part, we only care about the spatial components, however we do
end up with a time derivative so we shall not ignore the variety of
potential sums here.

The inverse metric is already known from previous problems, time
component \(-\frac{1}{1+2\phi}\) and space components
\(\frac{1}{1-2\phi}\). (along the diagonal, all else is zero).

    So now we take the covariant derivative. Funny thing, the covariant
derivative of any metric component is guaranteed to be zero, but because
we have a p out front it doesn't just simply vanish. We also set
\(\mu = i\) to remind ourselves we are only computing the physical
section. So the covariant derivative of our term becomes\ldots{}

\[ T^{i\nu}_{\nu} = \left[ (\rho + p)U^i U^\nu + pg^{i\nu} \right]_{,\nu} + \left[ (\rho + p)U^\alpha U^\nu + pg^{\alpha\nu} \right]\Gamma^i_{\alpha\nu} + \left[ (\rho + p)U^i U^\alpha + pg^{i\alpha} \right]\Gamma^\nu_{\alpha\nu} \]

\[  = \left[ (\rho + p)U^i U^\nu \right]_{,\nu} + \left[ (\rho + p)U^\alpha U^\nu \right]\Gamma^i_{\alpha\nu} + \left[ (\rho + p)U^i U^\alpha \right]\Gamma^\nu_{\alpha\nu} + \left[pg^{i\nu} \right]_{,\nu} + \left[pg^{\alpha\nu} \right]\Gamma^i_{\alpha\nu} + \left[pg^{i\alpha} \right]\Gamma^\nu_{\alpha\nu} \]

Split up everything using the product rule.

\[ = (\rho_{,\nu} + p_{,\nu})U^i U^\nu + (\rho + p)U^i_{,\nu} U^\nu  + (\rho + p)U^i U^\nu_{,\nu}  + \left[ (\rho + p)U^\alpha U^\nu \right]\Gamma^i_{\alpha\nu} + \left[ (\rho + p)U^i U^\alpha \right]\Gamma^\nu_{\alpha\nu} + p_{,\nu}g^{i\nu} + pg^{i\nu}_{,\nu} + \left[pg^{\alpha\nu} \right]\Gamma^i_{\alpha\nu} + \left[pg^{i\alpha} \right]\Gamma^\nu_{\alpha\nu} \]

The last three terms are just the expansion of a covariant derivative of
a metric, it has to be zero, cut all of them off.

\[ = (\rho_{,\nu} + p_{,\nu})U^i U^\nu + (\rho + p)U^i_{,\nu} U^\nu  + (\rho + p)U^i U^\nu_{,\nu}  + \left[ (\rho + p)U^\alpha U^\nu \right]\Gamma^i_{\alpha\nu} + \left[ (\rho + p)U^i U^\alpha \right]\Gamma^\nu_{\alpha\nu} + p_{,\nu}g^{i\nu}\]

    Now at this point we have to remind ourselves what \(\rho\) and p
represent. p = mU (the magnitude of U, not the components, we want to be
clear), and \(\rho = mn\). The n is important, since it's part of the
7.2 relation we have. n and U may not be constant, however, m is. m is
the mass per particle and as we're assuming a perfect fluid each
particle has to have the same amount, so we can pull this m out of
everything we might want. (Thus it is perfectly fine not to split up
\(\rho\) and p into components). Because of this, 7.2 implies:

\[ \rho_{;\alpha}U^\alpha + \rho U^\alpha_{;\alpha} =0 \]

\(\rho\) is a scalar so there's no need to apply the derivative
expansion to it. (?). However, U is a vector, so it can be split here.

\[ \rho_{,\alpha}U^\alpha + \rho U^\alpha_{,\alpha} + \Gamma^\alpha_{\mu\alpha}U^\mu =0 \]

    Now we do actually have this form in existence in our above relation,
but it isn't exactly obvious where it is. So let's split everything up.

\[ = \rho_{,\nu}U^i U^\nu + p_{,\nu}U^i U^\nu + \rho U^i_{,\nu} U^\nu + pU^i_{,\nu} U^\nu  + \rho U^i U^\nu_{,\nu} + pU^i U^\nu_{,\nu}  + \rho U^\alpha U^\nu \Gamma^i_{\alpha\nu} + pU^\alpha U^\nu \Gamma^i_{\alpha\nu} + \rho U^i U^\alpha \Gamma^\nu_{\alpha\nu} + pU^i U^\alpha \Gamma^\nu_{\alpha\nu} + p_{,\nu}g^{i\nu}\]

    With \(\alpha = \nu\) we can find every term we want multiplied by a
\(U^i\). Remove them.

\[ = p_{,\nu}U^i U^\nu + \rho U^i_{,\nu} U^\nu + pU^i_{,\nu} U^\nu + pU^i U^\nu_{,\nu}  + \rho U^\alpha U^\nu \Gamma^i_{\alpha\nu} + pU^\alpha U^\nu \Gamma^i_{\alpha\nu} + pU^i U^\alpha \Gamma^\nu_{\alpha\nu} + p_{,\nu}g^{i\nu}\]

    Now we can actually extract two of the terms we want out of this if we
just divide through by \(\rho\).

\[ = \frac1\rho p_{,\nu}U^i U^\nu + U^i_{,\nu} U^\nu + \frac1\rho pU^i_{,\nu} U^\nu + \frac1\rho pU^i U^\nu_{,\nu}  + U^\alpha U^\nu \Gamma^i_{\alpha\nu} + \frac1\rho pU^\alpha U^\nu \Gamma^i_{\alpha\nu} + \frac1\rho pU^i U^\alpha \Gamma^\nu_{\alpha\nu} + \frac{p_{,\nu}}{\rho}g^{i\nu}\]

Note that the two terms without fractions in front can be combined into
a covariant derivative.

\[ = \frac1\rho p_{,\nu}U^i U^\nu + U^i_{;\nu} U^\nu + \frac1\rho pU^i_{,\nu} U^\nu + \frac1\rho pU^i U^\nu_{,\nu}  + \frac1\rho pU^\alpha U^\nu \Gamma^i_{\alpha\nu} + \frac1\rho pU^i U^\alpha \Gamma^\nu_{\alpha\nu} + \frac{p_{,\nu}}{\rho}g^{i\nu}\]

The second term is, if we apply the spatial limitation, \$ v\_\{,t\} +
(v\cdot \nabla)v\$, and if we apply the spatial limitation on the last
term it becomes simply \(\nabla p/\rho\) since the metric is locally
\(\eta\). This is almost everything in the relation we need, we're just
missing the peasky \(\nabla \phi\) term.

However, we've got a lot of extra terms. We note that p
\textless\textless{} \(\rho\) and we have a lot of p/\(\rho\) terms,
which naturally will be \emph{really} small compared to everything else.
So let's just remove all those terms.

\[ = \frac1\rho p_{,\nu}U^i U^\nu + U^i_{;\nu} U^\nu + \frac{p_{,\nu}}{\rho}g^{i\nu}\]

    Now, all our hopes rest on pulling a \(\phi\) out of that first term,
somehow. However, we do know \(\phi\) is inherently tied to change in
momentum, so we have a chance. We can actually associate this all with
the definition of the geodesic and arrive at:

\[ = \frac1\rho U^i \frac{dp^\alpha}{d\tau} + U^i_{;\nu} U^\nu + \frac{p_{,\nu}}{\rho}g^{i\nu}\]

Which means we probably should have treated p as a vector this entire
time\ldots{} but changing our derivatives to covariant derivatives on p
doesn't acutally change anything about our algebra up to this point.
Regardless, this becomes via 7.24:

\[ = - \frac1\rho U^i m \phi_{,\alpha} + U^i_{;\nu} U^\nu + \frac{p_{,\nu}}{\rho}g^{i\nu}\]

\[ = - \frac1n U^i \phi_{,\alpha} + U^i_{;\nu} U^\nu + \frac{p_{,\nu}}{\rho}g^{i\nu}\]

Which\ldots{} yeah not quite right. Maybe p isn't supposed to be treated
as a vector, which leads to some ratehr unnerving questions as to how to
correlate it to \(\phi\)\ldots{}

Return to this agian tomorrow.

    \emph{b) Examine the time component of 7.6 under the same assumptions
and interpret each term.}

    \emph{c) 7.38 implies that a static fluid (v=0) in a static Newtonian
gravitaitonal field obeys the equation of hydrostatic equlibrium.
\(\nabla p + \rho \nabla \phi = 0\). A metric tensor is said to be
static if there exists coordisnates in which \(\vec e_0\) is timelike,
\(g_{i0}=0\), and \(g_{\alpha\beta,0}=0\). Deduce from 7.6 that a static
fluid \((U^i=0, p_{,0}=0, etc)\) obeys the relativistic equation of
hydrostatic equilibrium}

\[ p_{,i} + (\rho + p) \left[ \frac12 ln(-g_{00}) \right]_{,i} = 0 \]

    

    \emph{d) This suggests that, at least for static situations, there is a
close relation between \(g_{00}\) and \(-exp(2\phi)\), where \(\phi\) is
the Newtonian potential for a similar physical situation. Show that 7.8
and \textbf{Problem 4} are consistent with this.}

    \hypertarget{problem-6-back-to-top}{%
\section{\texorpdfstring{Problem 6 {[}Back to
\hyperref[toc]{top}{]}}{Problem 6 {[}Back to {]}}}\label{problem-6-back-to-top}}

\[\label{P6}\]

\emph{Deduce 7.25 from 7.10}

    7.10: \(\nabla_{\vec p} \vec p = 0\)

7.25: \(p^\alpha p_{\beta;\alpha} = 0\)

Here we go!

\[ \nabla_{\vec p} \vec p \] \[ = p^\beta p^\alpha_{;\beta} \]

Ooooh, the index is in the wrong spot. This suddenly became interesting.

While it looks complicated, it's actually not. Just transform both sides
of the equality using a metric.

\[  g_{\alpha\nu} p^\beta p^\alpha_{;\beta} = g_{\alpha\nu}(0)\]

\[  p^\beta p_{\alpha;\beta} = 0 \]

Tah-dah! While they definitely aren't equal to each other in an exact
sense (one is a zero one-form and one is a zero vector) they do both
evaluate to zero.

    \hypertarget{problem-7-back-to-top}{%
\section{\texorpdfstring{Problem 7 {[}Back to
\hyperref[toc]{top}{]}}{Problem 7 {[}Back to {]}}}\label{problem-7-back-to-top}}

\[\label{P7}\]

\emph{Consider the following four different metrics, as given by their
line elements.}

\emph{i) \(ds^2 = -dt^2+dx^2+dy^2+dz^2\)}

\emph{ii)
\(ds^2 = -(1-2M/r)dt^2+(1-2M/r)^{-1}dr^2+r^2(d\theta^2+ sin^2\theta d\phi^2)\),
where M is a constant.}

\emph{iii)
\(ds^2 = -\frac{\Delta - a^2sin^2\theta}{\rho^2}dt^2 - 2a\frac{2Mrsin^2\theta}{\rho^2}dtd\phi + \frac{(r^2+a^2)^2-a^2\Delta sin^2\theta}{\rho^2}sin^2\theta d\phi^2 + \frac{\rho^2}{\Delta}dr^2 + \rho^2 d\theta^2\),
where M and a are constants and we have introduced the shorthand
notation \(\Delta = r^2-2Mr+a^2 ; \rho^2 = r^2+a^2cos^2\theta\)}

\emph{iv)
\(ds^2 = -dt^2 + R^2(t)[(1-kr)^{-1}dr^2 + r^2(d\theta^2 + sin^2\theta d\phi^2)]\)
where k is a cosntant and R(t) is an arbitrary function of t alone.}

\emph{The first one should be familiar by now. We shall encounter the
other three in later chapters. Their names are, respectively, the
Schwarzschild, Kerr, and Robertson-Walker metrics.}

\emph{a) For each metric find as many conserved components
\(\rho_\alpha\) of a freely falling particle's four momentum as
possible.}

    This relies entirely on the evaluation of 7.29

\[ m \frac{dp_\beta}{d\tau} = \frac12 g_{\nu\alpha,\beta} p^\nu p^\alpha\]

That is, the momentum is conserved if the term on the right is zero.
Which means that if every term of the metric is independent of that
variable.

In the case of i), every single term depends on no variables whatsoever.
Thus, all four terms of the four momentum are conserved. Which it BETTER
be since energy conservation is a pretty important part of the standard
theory.

Note--for latter options, be careful, just because not all of them are
dependent doesn't mean the metrics don't always cancel. Check for
canceling.

    In the case of ii), as we are time independent in every case, \(p_t\) is
conserved, meaning we can apply a conservation of energy. However,
r-momentum and \(\theta\)-momentum are not conserved. Curiously,
\(\phi\)-momentum IS conserved, none of the components depend on it.
Fascinating. No opportunity for canceling as everything is unique and
along the diagonal.

\begin{enumerate}
\def\labelenumi{\roman{enumi})}
\setcounter{enumi}{2}
\tightlist
\item
  Likes to hide in complexity, but it might help to assign dependencies
  to the ``compressed'' objects. \(\Delta\) only has dependence on r.
  \(\rho\) has dependence on both r and \(\theta\). Even though some
  terms are on the off-diagonal, we do have to consider them as we use
  all of them here. Nothing depends on t, so energy conservation
  applies. everything depends on r, so that's not conserved. Everything
  also depends on \(\theta\). Once again, though, \(\phi\)-momentum is
  conserved.
\end{enumerate}

Now, we consider canceling, as we do in fact have some off-diagonal
terms. However, the diagonal terms are still terrible and kill both r
and \(\theta\) each and every time.

\begin{enumerate}
\def\labelenumi{\roman{enumi})}
\setcounter{enumi}{3}
\tightlist
\item
  for once, energy is not conserved, as we have a pretty clear time
  dependence. And an r dependence. And there's \(\theta\) dependence.
  But again there is NOT \(\phi\) dependence! \(\phi\) just likes to be
  conserved, it seems.
\end{enumerate}

Cancelation \emph{almost} occurs for r, and if k = \(1+sin^2\theta\) it
actually would. But sadly it does not.

    \emph{b) Use the result of \textbf{Problem 6-28} to put i) in the form
\(i') ds^2 = -dt^2 + dr^2 + r^2(d\theta^2 + sin^2\theta d\phi^2)\). From
this, argue that ii) and iv) are spherically symmetric. Does this
increase the number of conserved components \(p^\alpha\)?}

    Well we know the cartesian spherical metric is
\((1,r^2,r^2sin^2\theta)\). Naturally the portion of it in the time
metric must retain the transformation, so it becomes
\((-1,1,r^2,r^2sin^2\theta)\).

If we used the full metric we would have to add t-dependency to the
relations in the matrix, but in Minowski spacetime the components are
independent of time, so this produces a bunch of zeroes. t=t so the
transformation for the time component is just 1, so it leaves -1.

This obviously indicates that if we ignored the t and r components fir
ii) and iv) they reduce to the spherical shell metric; absolutely in the
case of ii), and with an added time dependence R(t) in the case of iv).
This would arguably demand that \(\theta\)-momentum be conserved as well
for both of them. Does this mean that 7.29 is not a guarantee of finding
\emph{all} conserved quantities?

Unfortunately this couldn't be true, for a nonzero derivative indicates
change, and the restricted-to-shell coordinates do not have any other
\(\theta\) dependencies to reduce it to zero. Perhaps, then, the
momentum components at that point themselves cancel to zero?
\(p^\phi p^\phi = 0\)? But this would only be true if the particle
wasn't moving in the \(\phi\) direction, and we're looking for a general
case. Thus, even though the potentials are spherically symmetric,
momentum is not always conserved along said shell.

This implies it is also true for Minkowski space, that
\(\theta\)-momentum is not conserved.

We can actually visually think of why this is. Momentum is conserved in
a circular orbit, that we know without question. A circular orbit is
effectively a geodesic along a spherical shell space. Howeever, if we
were to model that orbit at a 45 degree angle, it has to \emph{change
direction} in \(\theta\) to complete its orbit. Meanwhile in \(\phi\) it
keeps going the same direction at the same rate. (Imagine viewing a
clock from above, then tilting it left or right. it still goes around
the center at the same rate in terms of \(\phi\).)

So what actually happens is we \emph{Decreased} the conserved components
of i)! In fact, r-momentum is no longer conserved either. This is a
little less obvious as to why, for a circular geodesic maintains zero
r-momentum, and a line going directly away also maintains r-momentum.
Our method of ``Defining a well known geodesic that clarly violates the
result'' may not work here. It does make sense, though, as ostensibly
eliptical orbits are geodesics that clearly increase and decrease r,
obviously not keepin the component conserved.

One things that Kepler's Second Law (the one about sweeping out equal
areas) might be related to \(\phi\)-momentum conservation.

    \emph{c) It can be shown that for i') and ii)-iv), a geodesic that
begins with \(\theta = \pi/2\) and \(p^\theta = 0\) - i.e., one which
begins tangent to the equatorial plane - always has \(\theta = \pi/2\)
and \(p^\theta = 0\). For cases i'), ii), and iii), use the equation
\(\vec p \cdot \vec p = -m^2\) to solve for p\^{}r in terms of m, other
conserved quantities, and known functions of position.}

    The dot product uses the metric! Since we don't have to worry about
\(\theta\) let's just use the general metric and we get:

\[g_{tt} (p^t)^2 + g_{rr} (p^r)^2 + g{\phi\phi} (p^\phi)^2 = m^2\]
\[\Rightarrow g_{rr} (p^r)^2 +  = m^2 - g{\phi\phi} (p^\phi)^2 - g_{tt} (p^t)^2 \]
\[\Rightarrow (p^r)^2 +  = \frac{m^2 - g{\phi\phi} (p^\phi)^2 - g_{tt} (p^t)^2}{g_{rr}} \]

Now this is the general solution for i'), ii), and it would work for iv)
as well if t were conserved.

However, iii) throws its wrench into it all since it's all ``hey I have
off diagonal terms!'' So we need to treat it separately.

Fortunately the form is exactly the same, we just need to add an extra
term.

\[(p^r)^2 +  = \frac{m^2 - g{\phi\phi} (p^\phi)^2 - g_{tt} (p^t)^2 - 2g_{t\phi}p^tp^\phi}{g_{rr}} \]

Notably there are two copies because all metrics are symmetric.

Yes, these could be simplified. Yes, they coudl be fully substituted.
But that would be tedius and not lead to much in the way of insights,
since the only metric we have context for is i'). Which, by the way,
becomes:

\[\Rightarrow (p^r)^2 +  = m^2 - r^2sin^2\theta (p^\phi)^2 + (p^t)^2\]

    \emph{d) For iv) spherical ysmmetry implies that if a geodesic begins
with \(p^\theta = p^\phi = 0,\) these remain zero. Use this to show from
7.29 that when k=0, \(p_r\) is a conserved quantity.}

    Now this is trivial. With two components of p being zero, the only parts
of 7.29 that matter are the tt and rr terms, -1 and
\(R^2(t)(1-kr^2)^{-1}\). If k=0, the r dependence vanishes, and thus r
is conserved.

    \hypertarget{problem-8-back-to-top}{%
\section{\texorpdfstring{Problem 8 {[}Back to
\hyperref[toc]{top}{]}}{Problem 8 {[}Back to {]}}}\label{problem-8-back-to-top}}

\[\label{P8}\]

\emph{Suppose that in some coordinate system the components of the
metric \(g_{\alpha\beta}\) are independent of some coordinate \(x^\mu\)}

\emph{a) Show that the conservation law \(T^\nu_{\mu;\nu}\) for any
stress-energy tensor becomes}

\[ \frac{(\sqrt{-g} T^\nu_\mu)_{,\nu}}{\sqrt{-g}} = 0 \]

    First of all, we have this exact relation already proven for from
\textbf{Problem 6-34c} for antisymmetric ``double up'' Tensors. (aka,
double ``vector'' tensor, or just \(2\choose0\).) Does the argument
still hold for symmetric \(1\choose1\) tensors? (The stress-energy
tensor is in fact symmetric)

The answer is\ldots{} kinda! Because the covariant derivative of a
\(1\choose1\) tensor is of the form \(B' + \Gamma B - \Gamma B\), while
\(2\choose0\) is \(B' + \Gamma B + \Gamma B\). Which means that all the
``by symmetry'' arguments come out to be ``by antisymmetry''
EXCEPT\ldots{} the argument was that one term self-canceled. So now we
have to worry about that instead.

So let's write it out:
\(T^{\nu}_{\mu,\nu} - \Gamma^\alpha_{\mu\nu}T^\nu_\alpha + \Gamma^\nu_{\alpha\nu}T^\alpha_{\mu} = T^{\nu}_{\mu;\nu} = 0\).

    And it is rather evident that with the different locations of indeces
the symmetry in the lower-indeces of the Chrsitoffel symbols does not
apply. The third term can rather easily be set into the form we want:

\[T^{\nu}_{\mu,\nu} - \Gamma^\alpha_{\mu\nu}T^\nu_\alpha + \frac{\sqrt{-g}_{,\alpha}}{\sqrt{-g}}T^\alpha_{\mu} = T^{\nu}_{\mu;\nu} = 0\]

Which we can combine with the first term after some index shuffling and
inverse product rule to get:

\[ \frac{(\sqrt{-g}T^\nu_\mu)_{,\nu}}{\sqrt{-g}} - \Gamma^\alpha_{\mu\nu}T^\nu_\alpha = T^{\nu}_{\mu;\nu} = 0\]

    Which implies that the term subtracted off here has to self cancel.
Except, there's no way to guarantee that. It is easy to imagine a
symmetric tensor where that is not the case. Simply have there only be
positive values and put them all on the diagonal in the radial metric,
where only -r is along the diagonals! Which would restrict the elements
of the tensor to be related via a linear equation which is not true for
all symmetric tensors.

    Antisymmetry time! the middle term is 0. All flipped terms cancel, and
all diagonal terms are zero. Thus\ldots{}

\(\Rightarrow F^{\alpha\beta}_{,\beta} + \Gamma^\beta_{\mu\beta}F^{\alpha\mu} = F^{\alpha\beta}_{;\beta}\).

\(\Rightarrow F^{\alpha\beta}_{,\beta} + \frac{\sqrt{-g}_{,\mu}}{\sqrt{-g}}F^{\alpha\mu} = F^{\alpha\beta}_{;\beta}\).

\(F^{\alpha\beta}_{,\beta} + \Gamma^\alpha_{\mu\beta}F^{\mu\beta} + \Gamma^\beta_{\mu\beta}F^{\alpha\mu} = F^{\alpha\beta}_{;\beta}\)

    \emph{b) Suppose that in these coordinates \(T^{\alpha\beta}\neq 0\)
only in some bounded region of each spacelike hypersurface
\(x^0 = const\). Show that 7.41 (the above) implies}

\[ \int_{x_0 = const} T^\nu_\mu \sqrt{-g} n_\nu d^3x \]

\emph{is independent of \(x^0\) if \(n_\nu\) is the unit normal to the
hypersurface. This is the generalization to continua of the conservation
law stated after 7.29.}

    \emph{c) Consider flat Minkowski space in a global inertial frame with
spherical polar coordinates \((t,r,\theta,\phi)\). Show from b) that}

\[ J = \int_{t=const} T^0_\phi r^2sin\theta drd\theta d\phi\]

\emph{is independent of t. This is the total angular momentum of the
system.}

    \emph{d) Express the integral in c) in terms of the components
\(T^{\alpha\beta}\) on the cartesian basis (t,x,y,z) showing that}

\[ J = \int_{t=const} (xT^{y0}-yT^{x0}) dxdydz\]

\emph{This is the continuum version of the nonrelativistic expression
\((\textbf{r} \times \textbf{p})_z\) for a particle's angular momentum
about the z axis.}

    \hypertarget{problem-9-back-to-top}{%
\section{\texorpdfstring{Problem 9 {[}Back to
\hyperref[toc]{top}{]}}{Problem 9 {[}Back to {]}}}\label{problem-9-back-to-top}}

\[\label{P9}\]

\emph{a) Find the components of the Reimann tensor
\(R_{\alpha\beta\mu\nu}\)} for the metric 7.8 to first order in
\(\phi\)*

If we don't find a trick quickly this will be dismissed as needlessly
tedious.

Actually we decide it isn't really, since we have all the christoffel
coefficients and they're all really simple! Though naturally we find
\(R^{\alpha}_{\beta\mu\nu}\) instead since that one uses the
coefficients directly. We can just straight-up write them out maybe!
6.63 says:

\[ R^\alpha_{\beta\mu\nu} = \Gamma^\alpha_{\beta\nu,\mu} - \Gamma^\alpha_{\beta\mu,\nu} + \Gamma^\alpha_{\sigma\mu}\Gamma^\sigma_{\beta\mu,\nu} - \Gamma^\alpha_{\sigma\nu}\Gamma^\sigma_{\beta\mu} \]

and here are the Christoff symbols we already found:

\[ \Gamma^{\gamma}_{\beta\mu} = \begin{bmatrix}
\phi_{,t} & \phi_{,x} & \phi_{,y} & \phi_{,z} \\ 
\phi_{,x} & -\phi_{,t} & 0 & 0 \\
\phi_{,y} & 0 & -\phi_{,t} & 0 \\
\phi_{,z} & 0 & 0 & -\phi_{,t}
\end{bmatrix},\begin{bmatrix}
\phi_{,x} & -\phi_{,t} & 0 & 0 \\ 
-\phi_{,t} & -\phi_{,x} & -\phi_{,y} & -\phi_{,z} \\
0 & -\phi_{,y} & \phi_{,x} & 0 \\
0 & -\phi_{,z} & 0 & \phi_{,x}
\end{bmatrix},\begin{bmatrix}
\phi_{,y} & 0 & -\phi_{,t} & 0 \\ 
0 & \phi_{,y} & -\phi_{,x} & 0 \\
-\phi_{,t} & -\phi_{,x} & -\phi_{,y} & -\phi_{,z} \\
0 & 0 & -\phi_{,z} & \phi_{,y}
\end{bmatrix},\begin{bmatrix}
\phi_{,z} & 0 & 0 & -\phi_{,t} \\ 
0 & \phi_{,z} & 0 & -\phi_{,x} \\
0 & 0 & \phi_{,z} & -\phi_{,y} \\
-\phi_{,t} & -\phi_{,x} & -\phi_{,y} & -\phi_{,z}
\end{bmatrix}
\]

Second derivatives would just add another comma term to any of these if
they end up being used.

Now, we know that there are only 20 distinct terms in R, so we'll look
at those. \textbf{Problem 6-18} lists those.

Even with just 20 of them\ldots{} we have deemed this to be tedous. All
the tools are above, it's essentially just a lookup table at this point.

    \emph{b) Show that the equation of geodesic deviation, 6.87, implies (to
lowest order in \(\phi\) and velocities)}

\[ \frac{d^2\xi^i}{dt^2} = -\phi_{,ij}\xi^j \]

    6.87: \$ \nabla\_V \nabla\emph{V \xi\^{}\alpha =
R\^{}\alpha}\{\mu\nu\beta\} V\textsuperscript{\mu V}\nu \xi\^{}\beta\$

\ldots Obviously this would require us to fully evaluate a). To avoid
tedium, it is pretty clear from the form of the actual Christoffel
symbolst hat such an answer is reasonable, as it would just be second
derivatives.

Using a trick from \textbf{Problem 10} might give us some greater
insight though, sicne the double covariant derivatives will actually
become something in addition to the actual time up there via 6.85:

    \[ \frac{d^2}{d\tau^2} \xi^\alpha + \Gamma^\alpha_{\beta 0,0}\xi^\beta \]

Which provides

\[  \frac{d^2}{d\tau^2} \xi^\alpha = R^\alpha_{\mu\nu\beta} V^\mu V^\nu \xi^\beta - \Gamma^\alpha_{\beta 0,0}\xi^\beta \]

\[ \Rightarrow \frac{d^2}{d\tau^2} \xi^\alpha = (R^\alpha_{\mu\nu\beta} V^\mu V^\nu - \Gamma^\alpha_{\beta 0,0} ) \xi^\beta \]

    Which is essentially just a huge sum of Christoffel symbols, lots of
them. it's not at all unreasonable to expect it to turn out like the
equation above.

    \emph{c) Interpret this equation when the geodesics are world lines of
freely falling particles which begin from rest at nearby poitns in a
Newtonian gravitaitonal field.}

    Okay, so we need to remember that \(\vec \xi\) is the distance between
two lines that started out paralell. \(\phi\) is the strength of the
Newtonian field (often just considered the force).

Thus, what we are looking at is relative acceleration between the two
particles. Normally when we think of dropping things in gravity, they
drop together at the same rate--they accelerate with respect to the
ground, but not each other.

Since this term is celarly nonzero, the point is that this doesn't
happen. They will eventually diverge, that is, no longer be paralell.
Notably, if one imagines a uniform gravitaitonal field, then this goes
away: the distance between them isn't increasing at all and they truly
are falling in tandem. Such a gravity field does not really exist,
though.

It is interesting that the exact relation of how the distance relative
acceleration changes depends on entirely different components,
indicating some kind of rotational effect in play.

    \hypertarget{problem-10-back-to-top}{%
\section{\texorpdfstring{Problem 10 {[}Back to
\hyperref[toc]{top}{]}}{Problem 10 {[}Back to {]}}}\label{problem-10-back-to-top}}

\[\label{P10}\]

\emph{a) Show that if a vector field \(\xi^\alpha\) satisfies
\textbf{Killing's Equation},
\(\nabla_\alpha \xi_\beta + \nabla_\beta \xi_\alpha = 0\). Then along a
geodesic, \(p^\alpha\xi_\alpha\) = const. This is a coordinate invariant
way of characterizing the cosnervation law we deduced from 7.29. We only
have to know whether a metric admits Killing Fields}

    An important thing to note is that the equation is acting on one-forms,
not the vectors themselves.

OKAY so this is long and convoluted so BEAR WITH ME.

Start with \[ p^\alpha \xi_\alpha \]

Take the derivative with respect to proper time, the product rule makes
it split.

\[ \frac{dp^\alpha}{d\tau} \xi_\alpha + p^\alpha \frac{d \xi_\alpha}{d\tau} \]

Now both of these derivatives can be substituted, but each one is a bit
odd. Start with the momentum.

\[ \frac{dp^\alpha}{d\tau} \Rightarrow U^\beta p^\alpha_{,\beta} \]

This is part of the expansion of the covariant derivative. Perhaps more
important, though, is that p=mU, and so\ldots.

\[ = m U^\beta U^\alpha_{,\beta} \]

Which is part of the \emph{geodesic}, times m. The sum of the geodesic
is zero, so the other half of the geodesic can be equated to this one.
Precisely:

\[ m U^\beta U^\alpha_{,\beta} = - m \Gamma^\alpha_{\mu\beta} U^\beta U^\alpha = - \Gamma^\alpha_{\mu\beta} U^\beta p^\alpha \]

Which we will substitute back into the start when we get back to it.

Now we turn to the other half, but we don't know as much about it. Like
previously, though, we can identify it with a covariant derivative
expansion, although this one is a one-form.

\[ \frac{d\xi_\alpha}{d\tau} = U^\beta\xi_{\alpha,\beta} = U^\beta \xi_{\alpha;\beta} + \Gamma^\mu_{\alpha\beta} U^\beta \xi_\mu \]

Note that the covariant derivative remains instead of being able to be
set to zero. Now, if we make this substitution in the original:

\[ \frac{dp^\alpha}{d\tau} \xi_\alpha + p^\alpha \frac{d \xi_\alpha}{d\tau} \]

\[ = - \Gamma^\alpha_{\mu\beta} U^\beta p^\alpha \xi_\alpha + p^\alpha  U^\beta \xi_{\alpha;\beta} + p^\alpha \Gamma^\mu_{\alpha\beta} U^\beta \xi_\mu \]

With index shuffling the first and last terms cancel, leaving only:

\[ =  p^\alpha  U^\beta \xi_{\alpha;\beta} \]

Now we finally bring in the \textbf{KILLING} property, which basically
says the covariant derivatives are antisymmetric.

\[ = - p^\alpha  U^\beta \xi_{\beta;\alpha} \]

Which wouldn't tell us much\ldots{} but once again, p=mU. so we can pull
the m out of the first p and move it to the second.

\[ = - U^\alpha  p^\beta \xi_{\beta;\alpha} \]

So this allows us to do this trick that's so clever it looks like
cheating.

\[ p^\alpha  U^\beta \xi_{\alpha;\beta} = - U^\alpha  p^\beta \xi_{\beta;\alpha} \]

\[ \Rightarrow p^\alpha  U^\beta \xi_{\alpha;\beta} = - U^\beta  p^\alpha \xi_{\alpha;\beta} \]

The magic of INDEXING. Suddenly, it is equal to its own negative which
means both sides have to be zero which MEANS the derivative is zero and
thus THE VALUE IS CONSTANT.

    \emph{b) Find ten Killing fields of Minkowski spacetime.}

    So one-form and vector fields are identical in Minkowski spacetime, so
we don't have to worry much about changing everything.

Obviously the null field works (0,0,0,0).

Then any constant field (a,b,c,d). Naturally one could cheat and just
make more out of this one, but we'll only count it as one distinct one
(though the zero was considered a special case.)

The simplest non-trivial one is (x,-t,0,0)

One can also play on the other side (0,0,z,-y).

And they can be combined due to obvious independence (x,-t,z,-y).

Or we can fixate on one coordinate: (x+y+z, -t, -t, -t)

Notably we can add any of these together to make even more. (2x+y+z,
-2t, -t, -t) and (2x+y+z,-2t,z-t,-y-t) for example.

I think that's enough to get a general idea of how this works, though
there may be a few odd-derivative cases here and there that might be
able to be made to work.

NOTE: aha, yes, there was a note we missed: the SUMMATION over the
indeces. Which means something like t=t can work, as everything gets
added. Say (x-t,t-x,0,0). That works. What we're actually doing is
trying every combination of derivatives, ensuring that they equal zero
in the end.

    \emph{c) Show that if \(\vec \xi\) and \(\vec \eta\) are Killing fields,
then so is \(\alpha\vec\xi+\beta\vec\eta\) for constant \(\alpha\) and
\(\beta\)}

    We \emph{demonstrated} this above, but did nothing about showing it was
always true for all possible killing fields. Furthermore, b) was only in
Minkowski space, this will be in \emph{other} space and thus technically
the \textbf{KILLING} property is determined by one-forms rather than the
vectors themselves.

Fortunately showing that the property holds no matter which way one goes
is trivial.

\[\nabla_\alpha \xi_\beta + \nabla_\beta \xi_\alpha = 0\]

Apply a raising metric to both sides. They can move in and out of
covariant derivatives at will and do nothing to 0-one-forms aside from
turn them into 0-vectors. Which means the flipped form holds as well:

\[\nabla_\alpha g^{\beta\alpha} \xi_\beta + \nabla_\beta g^{\beta\alpha} \xi_\alpha = 0\]

\[\nabla_\alpha \xi^\alpha + \nabla_\beta \xi^\beta = 0\]

Look at how the indeces have changed, fascinating. Now the requirement
is that self-covariant-derivatives of the vectors themselves eventually
add to zero.

    Since this a bit suspicious, we note that adding vectors created a new
vector, but adding the one-forms the original vectors correlate to
creates a one-form that correlates to the new vector. Which means we can
also see if we can prove the one-form fields on their own maintain the
property.

Which\ldots{} is rahter easy to show. Let a one-form be a combination of
two one-forms\ldots{}

\[\nabla_\alpha \xi_\beta + \nabla_\beta \xi_\alpha \]
\[ = \nabla_\alpha (\omega_\beta + \iota_\beta) + \nabla_\beta (\omega_\alpha + \iota_\alpha) \]
\[ = \nabla_\alpha \omega_\beta + \nabla_\alpha \iota_\beta + \nabla_\beta \omega_\alpha + \nabla_\beta \iota_\alpha \]

Which just contains two copies of the original relation, both equal
zero, so it holds.

    \emph{d) Show that Lorentz transformations of the fields in b) simply
produce linear combinations as in c).}

    Lorentz transformations can always be represented as
\[(\gamma, -v\gamma) (-v\gamma, \gamma)\] with enough coordinate
rotations. Let's use some two-fields to make this easier: (x,-t) is our
example, the result becomes
\((x\gamma + v\gamma t, -vx\gamma - t\gamma)\). Which yes is a linear
combination of (x,-t) and (t,-x). Very simple.

Note: the metric does not change.

    \emph{e) If you did \textbf{Problem 7} use the results of 7-a to find
Killing vectors of metrics ii)-iv)}

    Will you take the zero vector? No? Well in that case, conserved
quantities will help us here. The conserved quantities can refer to each
other easily with no issue, so having a \[\phi , -t\] swap will work for
ii) and iii). However, iv) only has one conserved quantity\ldots{}

But, as we showed also in \textbf{Problem 7}, it's analagous to
spherical shell coordinates. Since we don't wanna waste time on this the
method is rather obvious: take the covariant derivative of either
\(\theta\) or \(\phi\), find out what the result is, and then use that
result on the other coordinate to make it equal the negative of the
first. The Chrsistoffel symbol calculation (and typing) will take up
most of the time here.

    Edit: all right, the day is new, might as well actually calculate it. Go
ahead and use the r=1 metric, which has diagonal \((1,sin^2\theta)\).

Since it might be helpful to write them out, here's the relations in
shorthand:

\[ \theta\theta\theta = 1/2 \]
\[ \theta\theta\phi = \theta\phi\theta = \phi\phi\phi = \phi\theta\theta = 0 \]
\[ \theta\phi\phi = -cos\theta sin\theta \]
\[ \phi\theta\phi = \phi\phi\theta = 1/tan\theta \]

    Let's let the \(\theta\) component be \(\phi\), and then differentiate
it with respect to \(\phi\).

\[ V^\theta_{;\phi} = V^\theta_{,\phi} + \Gamma^\theta_{\mu\phi} V^\mu \]
\[ = 1 + \Gamma^\theta_{\theta\phi} V^\theta + \Gamma^\theta_{\phi\phi} V^\phi \]
\[ = 1 + 0 - cos\theta sin\theta V^\phi \]

We're going to end up with a differential equation, since:

\[ V^\phi_{;\theta} = -V^\theta_{;\phi} = cos\theta sin\theta V^\phi - 1 \]
\[ V^\phi_{;\theta} = V^\phi_{,\theta} + \Gamma^\phi_{\phi\theta} V^\phi + \Gamma^\phi_{\theta\theta} V^\theta \]
\[ = V^\phi_{,\theta}\]

Which brings us to:

\[ cos\theta sin\theta V^\phi - 1 = V^\phi_{,\theta} + \frac{1}{tan\theta} V^\phi\]

Solve for \(V^\phi\)

    \[ cos\theta sin\theta V^\phi - 1 - \frac{1}{tan\theta} V^\phi = V^\phi_{,\theta} \]

Let's just have a calculator do this.

\ldots Hmm well. At least we have a answer.

\begin{figure}
\centering
\includegraphics{attachment:Screenshot\%20from\%202022-06-07\%2008-42-16.png}
\caption{Screenshot\%20from\%202022-06-07\%2008-42-16.png}
\end{figure}

There's probably a wiser and better choice somewhere that will be more
evident, but this proves the principle.

Derived using \hyperref[1]{1}

    \hypertarget{addendum-output-this-notebook-to-latex-formatted-pdf-file-back-to-top}{%
\section{\texorpdfstring{Addendum: Output this notebook to
\(\LaTeX\)-formatted PDF file {[}Back to
\hyperref[toc]{top}{]}}{Addendum: Output this notebook to \textbackslash LaTeX-formatted PDF file {[}Back to {]}}}\label{addendum-output-this-notebook-to-latex-formatted-pdf-file-back-to-top}}

\[\label{latex_pdf_output}\]

The following code cell converts this Jupyter notebook into a proper,
clickable \(\LaTeX\)-formatted PDF file. After the cell is successfully
run, the generated PDF may be found in the root NRPy+ tutorial
directory, with filename \url{GR-07.pdf} (Note that clicking on this
link may not work; you may need to open the PDF file through another
means.)

\textbf{Important Note}: Make sure that the file name is right in all
six locations, two here in the Markdown, four in the code below.

\begin{itemize}
\tightlist
\item
  GR-07.pdf
\item
  GR-07.ipynb
\item
  GR-07.tex
\end{itemize}

    \begin{tcolorbox}[breakable, size=fbox, boxrule=1pt, pad at break*=1mm,colback=cellbackground, colframe=cellborder]
\prompt{In}{incolor}{1}{\boxspacing}
\begin{Verbatim}[commandchars=\\\{\}]
\PY{k+kn}{import} \PY{n+nn}{cmdline\PYZus{}helper} \PY{k}{as} \PY{n+nn}{cmd}    \PY{c+c1}{\PYZsh{} NRPy+: Multi\PYZhy{}platform Python command\PYZhy{}line interface}
\PY{n}{cmd}\PY{o}{.}\PY{n}{output\PYZus{}Jupyter\PYZus{}notebook\PYZus{}to\PYZus{}LaTeXed\PYZus{}PDF}\PY{p}{(}\PY{l+s+s2}{\PYZdq{}}\PY{l+s+s2}{GR\PYZhy{}07}\PY{l+s+s2}{\PYZdq{}}\PY{p}{)}
\end{Verbatim}
\end{tcolorbox}

    \begin{Verbatim}[commandchars=\\\{\}]
Created GR-07.tex, and compiled LaTeX file to PDF file GR-07.pdf
    \end{Verbatim}

    \begin{tcolorbox}[breakable, size=fbox, boxrule=1pt, pad at break*=1mm,colback=cellbackground, colframe=cellborder]
\prompt{In}{incolor}{ }{\boxspacing}
\begin{Verbatim}[commandchars=\\\{\}]

\end{Verbatim}
\end{tcolorbox}

    \begin{tcolorbox}[breakable, size=fbox, boxrule=1pt, pad at break*=1mm,colback=cellbackground, colframe=cellborder]
\prompt{In}{incolor}{ }{\boxspacing}
\begin{Verbatim}[commandchars=\\\{\}]

\end{Verbatim}
\end{tcolorbox}


    % Add a bibliography block to the postdoc
    
    
    
\end{document}
